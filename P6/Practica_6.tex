\documentclass[12pt,twoside]{article}
\usepackage{amsmath, amssymb}
\usepackage{amsmath}
\usepackage[active]{srcltx}
\usepackage{amssymb}
\usepackage{amscd}
\usepackage{makeidx}
\usepackage{amsthm}
\usepackage{algpseudocode}
\usepackage{algorithm}
\usepackage{listings}
\usepackage{graphicx}
\usepackage[spanish,es-noshorthands]{babel}
\usepackage{wrapfig}
\usepackage{tikz}
\renewcommand{\baselinestretch}{1}
\setcounter{page}{1}
\setlength{\textheight}{21.6cm}
\setlength{\textwidth}{14cm}
\setlength{\oddsidemargin}{1cm}
\setlength{\evensidemargin}{1cm}
\pagestyle{myheadings}
\thispagestyle{empty}
\markboth{\small{Pr\'actica 6. Eduardo, Daniel.}}{\small{.}}
\date{}
\begin{document}
\centerline{\bf An\'alisis de Algoritmos, Sem: 2020-1, 3CV2, Pr\'actica 6, 1 de Junio de 2020}
\centerline{}
\centerline{}
\begin{center}
\Large{\textsc{Práctica 6: Programación Dinámica.}}
\end{center}
\centerline{}
\centerline{\bf {Mendoza Mart\'inez Eduardo, Aguilar Gonzalez Daniel.}}
\centerline{}
\centerline{Escuela Superior de C\'omputo}
\centerline{Instituto Polit\'ecnico Nacional, M\'exico}
\centerline{$edoomm8@gmail.com, daguilarglz97@gmail.com$}
\newtheorem{Theorem}{\quad Theorem}[section]
\newtheorem{Definition}[Theorem]{\quad Definition}
\newtheorem{Corollary}[Theorem]{\quad Corollary}
\newtheorem{Lemma}[Theorem]{\quad Lemma}
\newtheorem{Example}[Theorem]{\quad Example}
\bigskip
\textbf{Resumen:} La siguiente práctica pretende mostrar el comportamiento de 3 algoritmos Algoritmo de fibonacci con el enfoque tup-down y buttom-up, Problema de la mochila entera y Problema de lineas de produccion estos implementados mediante el metodo de Programación dinamica, comparando sus tiempos de ejecuci\´on, as\´i como la complejidad de estos. \newline
{\bf Palabras Clave:} Solución óptima, Enfoque top-down y bottom-up, Algoritmo de Fibonacci, Problema de la mochila entera y Algoritmo de lineas de producción.

\section{Introducci\'on}
A lo largo del curso hemos desarrollado diferentes algoritmos los cuales pueden ser atacados con más de un metodo de solución de algoritmos como lo puede ser el algoritmo de fibonacci el cual se trato en una practica anterior dandole solución con dos algoritmos diferentes uno iterativo y otro recursivo. En este caso trataremos este algoritmo mediante el metodo de programación dinámica. Al resolver un algoritmo con métodos diferentes no debe haber algún cambio al momento de mostrar su resultado ya que este debe ser el mismo debido a que el algoritmo da solución a un problema específico como observaremos en el desarrollo de esta práctica.
En esta practica mostraremos el como funciona la programación dinámica al momento de implementar un algoritmo como los 3 antes mencionados y observaremos a continuación.

\section{Conceptos B\'asicos}
Se requiere conocer algunos conceptos relacionados
a la implementaci\´on de los algoritmos estudiados en esta pr\´actica y los
m\´etodos a trav\´es de los cuales se desarrollaron. Se presenta a continuaci\´on la informaci\´on
\subsection{Programación dinámica}
La programación dinámica es una técnica matemática que
se utiliza para la solución de problemas matemáticos
seleccionados, en los cuales se toma un serie de decisiones en forma secuencial.
Proporciona un procedimiento sistemático para encontrar la combinación de decisiones que maximice la efectividad total, al descomponer el problema en etapas, las que pueden ser completadas por una o más formas (estados), y enlazando cada etapa a través de cálculos recursivos.
La programación dinámica es básicamente un algoritmo de optimización. Significa que podemos resolver cualquier problema sin usar programación dinámica, pero podemos resolverlo de una mejor manera u optimizarlo usando programación dinámica.
La idea básica de la programación dinámica es almacenar el resultado de un problema después de resolverlo. Entonces, cuando tenemos la necesidad de usar la solución del problema, entonces no tenemos que resolver el problema nuevamente y solo usar la solución almacenada.
\subsection{Enfoques de programación Dinamica ( top-down y bottom-up)}\newline
Top down. \newline
El enfoque top-down enfatiza la planificación y conocimiento completo del sistema. Se entiende que la codificación no puede comenzar hasta que no se haya alcanzado un nivel de detalle suficiente, al menos en alguna parte del sistema. Esto retrasa las pruebas de las unidades funcionales del sistema hasta que gran parte del diseño se ha completado.
También conocida como de arriba-abajo y consiste en establecer una serie de niveles de mayor a menor complejidad (arriba-abajo) que den solución al problema. Consiste en efectuar una relación entre las etapas de la estructuración de forma que una etapa jerárquica y su inmediato inferior se relacionen mediante entradas y salidas de información. Este diseño consiste en una serie de descomposiciones sucesivas del problema inicial, que recibe el refinamiento progresivo del repertorio de instrucciones que van a formar parte del programa.\newline
Bottom-up. \newline
Bottom-up hace énfasis en la programación y pruebas tempranas, que pueden comenzar tan pronto se ha especificado el primer módulo. Este enfoque tiene el riesgo de programar cosas sin saber como se van a conectar al resto del sistema, y esta conexión puede no ser tan fácil como se creyó al comienzo. La reutilización del código es uno de los mayores beneficios del enfoque bottom-up.
El diseño ascendente se refiere a la identificación de aquellos procesos que necesitan computarizarse con forme vayan apareciendo, su análisis como sistema y su codificación, o bien, la adquisición de paquetes de software para satisfacer el problema inmediato.
\subsection{Serie de Fibonacci}
La serie Fibonacci, es una sucesión que comienza  con los números 0 y 1, y a partir de estos, cada termino siguiente es la suma de los dos anteriores.\newline
Ejemplo:\newline
    0, 1, 1, 2, 3, 5, 8, 13, 21, 34, 55, 89, 144, …
\subsection{Problema de la mochila entera}
El problema de la mochila (KP) puede ser definido con un conjunto de n artículos donde cada artículo es identificado por nx, con un valor entero px, y un peso wx. El problema consiste en elegir un subconjunto de n artículos maximizando el beneficio obtenido considerando el peso total de los artículos seleccionados, sin exceder la capacidad c de la mochila.
Se dispone de una mochila de capacidad C y de un conjunto de N objetos, donde los objetos son indivisibles. Describen a un objeto k que tiene un beneficio bk y un peso pk, para k = 1,2,…, N. Para los autores, el problema consiste en averiguar qué objetos se pueden insertar en la mochila sin exceder la capacidad total de la misma, obteniendo el máximo beneficio.

\section{Experimentaci\'on y Resultados}
En esta sección se presentan la experimentación de cada algoritmo y los resultados que nos arrojan cada uno de ellos así como sus gráficas y explicaciones breves.

\subsection{Algoritmo de Fibonacci}
A continuación mostraremos la implementación del algoritmo de Fibonacci a través de dos enfoques que pertenecen a la programacion dinámica
\subsubsection{Pseudocodigo Algoritmo fibonacci Enfoque top-down}
\begin{lstlisting}
1-- fibo = [100]
2-- for i = 0 to i<100
3--    fibo[i]=-1
4-- fibo[0]=0
5-- fibo[1]=1
6-- fibonacci_desc(n[0,...,n],fibo:tabla)
7--    if (fibo[n] !=-1)
8--        return fibo[n]
9--    fibo[n]= fibonacci_desc(n-2) + fibonacci_desc(n-1)
10--     return fibo[n]
\end{lstlisting}
\subsubsection{Algoritmo de Fibonacci Enfoque  top-down}
Consiste simplemente en comenzar resolviendo el problema de manera natural y almacenando las soluciones de los subproblemas en el camino. \newline
Buscamos encontrar el n-esimo termino de la sucesión es decir si n=5 buscaremos que numero es el que se encuentra en la posición numero 5 de la serie de fibonacci.\newline
Mencionado lo anterior comenzamos haciendo pruebas del algoritmo con este enfoque, nuestra primer prueba a realizar es con el valor de n= 50 es decir buscar el valor que toma la posición 50. En la figura 1 se muestra el tiempo que tardo en realizar esta instrucción y en la figura 2 la gráfica que nos arroja n contra el tiempo.
\begin{center}
    \includegraphics[width = 10cm]{Fibonacci/td/4096c.png}\\
    Figura 1 - Tiempo de ejecución para n=50 enfoque top down.
\end{center}

\begin{center}
    \includegraphics[width = 10cm]{Fibonacci/td/50g.png}\\
    Figura 2 - Gráfica del algoritmo para n=50 enfoque top down.
\end{center}
\newline
Al observar la fígura 2 nos damos cuenta que hay variaciones entre las primeras 10 posiciones, a partir de esa posición las variaciones son más pequeñas pero esta informacion no nos basta para poder buscar una función que acote de manera correcta nuestra grafica, asi que seguiremos aumentando el valor de n buscando otro comportamiento de la grafica. Nuestra siguiente prueba es aumentar el valor a n=1000 como se muestra a continuación. En la figura 3 podemos observar el tiempo que tarda en buscar el 1000 esimo termino de la serie y en la figura 4 la grafica obtenida.
\begin{center}
    \includegraphics[width = 10cm]{Fibonacci/td/1000c.png}\\
    Figura 3 - Tiempo de ejecución para n=1000 enfoque top down.
\end{center}

\begin{center}
    \includegraphics[width = 10cm]{Fibonacci/td/1000g.png}\\
    Figura 4 - Gráfica del algoritmo para n=1000 enfoque top down.
\end{center}
\newline
Aumentando el valor a 1000 seguimos viendo que nuestra grafica tiene variaciones en determinar el tiempo de cada posición sin aún poder observar la forma que tomara nuestra gráfica, asi que como lo venimos haciendo seguiremos incrementando nuestro valor hasta poder observar de mejor manera el comportamiento que buscamos.
Nuestro siguiente experimento es aumentar el valor a   n = 3125 y analizar su grafica que produce la cual se muetsra en la figura 6 y en la figura 5 se muestra el tiempo que tarda el programa en ejecutar esa cantidad.
\begin{center}
    \includegraphics[width = 10cm]{Fibonacci/td/3125_5c.png}\\
    Figura 5 - Tiempo de ejecución para n=3125 enfoque top down.
\end{center}

\begin{center}
    \includegraphics[width = 10cm]{Fibonacci/td/3125_5g.png}\\
    Figura 6 - Gráfica del algoritmo para n=3125 enfoque top down.
\end{center}
\newline
En la figura 6 podemos observar que la gráfica comienza a tomar una forma líneal despues de una cantidad mayor a 700 dandonos asi más información del comportamiento que tomara para poder darnos una idea de que forma podra ser la funcion que la acote de manera adecuada nuestra grafica que en esta ocacion parece ser de forma lineal.
Haremos una última prueba incrementando el valor a n= 4096 para poder cerciorarnos que nuestra grafica continua tomando esta forma como se muestra en la figura 8 y en la figura 9 observamos su tiempo de ejecución.
\begin{center}
    \includegraphics[width = 10cm]{Fibonacci/td/4096c.png}\\
    Figura 7 - Tiempo de ejecución para n=4096 enfoque top down.
\end{center}

\begin{center}
    \includegraphics[width = 10cm]{Fibonacci/td/4096g.png}\\
    Figura 8 - Gráfica del algoritmo para n=4096 enfoque top down.
\end{center}
\newline
Como podemos observar efectivamente la grafica sigue tomando el mismo comportamiento que mencionamos anteriormente asi que ahora procederemos a calcular una función que acote nuestra grafica. Como se menciono anteriormente nuestra grafica va tomando una forma lineal por lo que podemos buscar otra de la misma índole. \\
La funci\'on obtenida nos queda de la siguiente manera:
\begin{center}
    $f(n) = \frac{1}{10000}n$
\end{center}\\
En la figura 10 podemos observar una grafica con valor de 10,000 acotada de manera correcta por nuestra función propuesta y en la figura 9 el tiempo que tardo en ejecutarse.
\begin{center}
    \includegraphics[width = 10cm]{Fibonacci/td/10000_10c.png}\\
    Figura 9 - Tiempo de ejecución para n=10000 enfoque top down acotada por f(n).
\end{center}

\begin{center}
    \includegraphics[width = 10cm]{Fibonacci/td/10000_10g.png}\\
    Figura 10 - Gráfica del algoritmo para n=10000 enfoque top down acotada por f(n).
\end{center}
\newline
Gracias a todas las pruebas anteriores y a sus graficas que estas arrojaron pudimos encontrar la función que las acota llegando a la conclusión que el algoritmo  Fibonacci con Enfoque top down de Programación dinámica  tiene complejidad de orden Lineal $\theta{(n)}$.

\subsubsection{Pseudocodigo Algoritmo fibonacci Enfoque bottom-up}
\begin{lstlisting}
1-- fibo = [100]
2-- fibonacci_asc(n[0,...,n], fibo:tabla)
3--     if n <= 1
4--         return 1
5--     else
6--         fibo[0] = 0
7--         fibo[1] = 1
8--         for i=2 to i<=n do
9--             fibo[i] = fibo[i-1] + fibo[i-2]
10--     return fibo[n]
\end{lstlisting}
\subsubsection{Algoritmo de Fibonacci Enfoque  bottom-up}
Este enfoque consiste en crear una tabla de abajo hacia arriba y devuelve la última entrada de la tabla.
\newline \newline
Comenzamos haciendo pruebas del algoritmo con este enfoque, iniciando con n= 80. En la figura 11 mostramos el tiempo que tardo en buscar el valor de esta posición y el valor del numero que esta en esa posición y en la figura 12 la grafica que resulta haciendo la comparacion de la posición buscada contra el tiempo que tardó .
\begin{center}
    \includegraphics[width = 10cm]{Fibonacci/bu/fbu80t.PNG}\\
    Figura 11 - Tiempo de ejecución para n=80.
\end{center}

\begin{center}
    \includegraphics[width = 10cm]{Fibonacci/bu/fbu80g.PNG}\\
    Figura 12 - Gráfica del algoritmo para n=80.
\end{center}
\newline
En la grafica anterior que no toma una forma en la cual podamos observar el comportamiento de esta asi que lo siguiente que haremos es aumentar el valor de n a 800 y analizaremos la forma que toma la gráfica con este valor y determinaremos si es la adecuada o seguimos aumentando el valor de n.
En la figura 13 observamos el tiempo que tomo en encontrar el valor de la posición 800 y en la figura 14 su gráfica.
\begin{center}
    \includegraphics[width = 10cm]{Fibonacci/bu/fbu800t.PNG}\\
    Figura 13 - Tiempo de ejecución para n=800.
\end{center}

\begin{center}
    \includegraphics[width = 10cm]{Fibonacci/bu/fbu800g.PNG}\\
    Figura 14 - Gráfica del algoritmo para n=800.
\end{center}
\newline
Al aumentar el valor vemos que nuestra grafica sigue sin tomar un comportamiento que nos brinde información para poder encontrar una funcion y acotarla, así que nuevamente aumentaremos el valor de n a 4 cifras y seguiremos analizando.
\begin{center}
    \includegraphics[width = 10cm]{Fibonacci/bu/fbu5070t.PNG}\\
    Figura 15 - Tiempo de ejecución para n=5070.
\end{center}

\begin{center}
    \includegraphics[width = 10cm]{Fibonacci/bu/fbu5070g.PNG}\\
    Figura 16 - Gráfica del algoritmo para n=5070.
\end{center}
\newline
En la figura 16 observamos que la gráfica comienza a tomar forma aunque aún no se ve claramente como es, así que continuaremos aumentando el valor de n a 8000 como se muestra en las siguientes figuras.
\begin{center}
    \includegraphics[width = 10cm]{Fibonacci/bu/fbu8000t.PNG}\\
    Figura 17 - Tiempo de ejecución para n=8000.
\end{center}

\begin{center}
    \includegraphics[width = 10cm]{Fibonacci/bu/fbu8000g.PNG}\\
    Figura 18 - Gráfica del algoritmo para n=8000.
\end{center}
\newline
En la figura 18 nos damos cuenta que con el valor de 8000 la gráfica ya muestra una forma más clara con la que el siguiente paso es buscar una función que pueda acotar de manera corresta esta gráfica. \newline
Como hemos visto anteriormente entre más incrementamos el valor de n la gráfica se va viendo más clara cada vez asi que aunque con el valor de 8000 ya tenemos una mejor grafica, aumentaremos el valor y buscaremos una funcion. \newline \newline
Lo siguiente es encontrar una funci\'on que acote nuestra grafica y nos determine el orden de complejidad de este algoritmo todo basado en graficas que hemos mostrado. La funci\'on obtenida nos queda de la siguiente manera:
\begin{center}
    $f(n) = \frac{1}{60000}n$
\end{center}
En la figura 19 mostramos como nuestra f(n) obtenida acota de manera correcta la grafica del n-esimo de la serie de Fibonacci el cual tiene el valor de 12300 y en la figura 20 el tiempo que tard\'o en realizarse.
\begin{center}
    \includegraphics[width = 10cm]{Fibonacci/bu/fbu12300g.PNG}\\
    Figura 19 - Gráfica con valor 12300 acotada por f(n).
\end{center}
\begin{center}
    \includegraphics[width = 10cm]{Fibonacci/bu/fbu12300t.PNG}\\
    Figura 20 - Tiempo de ejecución para n=12300.
\end{center}
\newline
Como prueba extra incrementamos elvalor de n manteniendo los mismos valores de nuestra función para observar que la función acota correctamente nuestra gráfica como se muestra en la figura 21.
\begin{center}
    \includegraphics[width = 10cm]{Fibonacci/bu/fbu17300g.PNG}\\
    Figura 21 - Gráfica con valor 17300 acotada por f(n).
\end{center}
Concluyendo asi que graficamente el algoritmo Fibonacci con el Enfoque \newline bottom-up de Programación dinámica  tiene complejidad de orden Lineal $\theta{(n)}$.

\subsubsection{Top-down vs Bottom-up}
En esta sección mostramos que algoritmo de fibonacci se puede resolver de varias formas, aquí lo resolvimos mediante los dos enfoques que tiene el método programacion dinámica y gracias a los resultados experimentales mostrados anteriormente pudimos darnos cuenta que aunque los dos enfoques se comoponen y actuan de manera diferente ambos tienen un Orden de complejidad Lineal es decir $\theta{(n)}$, nos dimos cuenta de esto gracias a las gráficas aqui mostradas las cuales se obtenian gracias al tiempo que tardaba el programa en encontrar el valor de una posición.\\
Ambos tienen complejidad líneal pero cada uno de ellos actúa de manera distinta aunque no lo podamos ver de manera directa, ejemplo de esto es el tiempo en que cada uno de ellos calcula cada numero. \\
El enfoque que mejor resultados nos dió segun las gráficas fue el Enfoque Top-down ya que este enfoque comenzó a tener variaciones tiempo más notables a partir del valor 1000 fue en este punto donde comenzó a incrementarse de tal manera que nos daba una aparente forma lineal a demas las variaciones eran muy leves esto quiere decir que el tiempo en calcular una posición es muy corto esto debido a que es más amable con el rendimiento ya que nos ayuda a evitar cálculos innecesarios debido a que cada que cálcula un numero nuevo lo almacena esto con motivo que si más adelante requiere calcularlo no lo hará de nuevo, solo lo mandará a llamar debido a que usa llamadas a funciones de recursividad. Esto puede poner en desventaja este enfoque ya que  las llamadas a funciones recursivas requieren memoria de una fuente la cual tiene límites fijos y si la profundidad de recursión es demasiado profunda el programa se puede bloquear. \\
El principal beneficio que nos brinda el enfoque top-down es ejecutar de manera más rápida nuestro programa y ser más amable con su rendimiento al evitar realizar cálculos innecesarios es por este motivo que con el Enfoque Top-down se obtienen mejores resultados.
\newpage
\subsection{Algoritmo de la Mochila Entera}
En prácticas previas estudiamos el algoritmo de la mochila fraccionaria, donde se maxificaba el beneficio que una mochila puede transportar con ciertos objetos dados, en esta ocasión veremos el algoritmo de la mochila entera. La principal diferencia entre estos 2 es, como su nombre lo indica, que en una se pueden ingresar fracciones de ciertas cosas, mientras que la mochila entera no podemos ingresar fracciones de los objetos que estarán dentro de la mochila, en otras palabras, podríamos decir que los objetos a ingresar son indivisibles.

\subsubsection{Pseudocódigos del algoritmo de la mochila entera}
Para resolver la problemática de la mochila entera es necesario implementar dos funciones. La primera de ella la denominamos como \textit{generarTabla} y su pseudocódigo se muestra a continuación.
\begin{lstlisting}
generarTabla(w, b:[1,..., n], P:int)
1-  for c = 0 to c <= P do
2-      g[0,c] = 0
3-  for j = 1 to j <= n do
4-      g[j,0] = 0
5-  for j = 1 to j <= n do
6-      for c = 1 to c <= P to
7-          if c < w[j]
8-              g[j,c] = g[j-1,c]
9-          else
10-             if g[j-1,c] >= g[j-1,c-w[j]] + b[j]
11-                 g[j,c] = g[j-1,c]
12-             else
13-                 g[j,c] = g[j-1, c-w[j]] + b[j]
14- return g
\end{lstlisting}
La anterior función generará una tabla y con esta salida, podremos resolver el problema de la mochila entera a través del pseudocódigo siguiente.
\begin{lstlisting}
test(w, b[1,...,n],P:int,g:[0,...,n][0,...,P])
1-  if j > 0 then
2-      if c < w[j] then
3-          test(j-1,c)
3-      else
4-          if g[j-1, c-w[j]] + b[j] > g[j-1,c] then
5-              test(j-1, c-w[j])
6-              print("guardar objeto", j)
7-          else
8-              test(j-1, c)
\end{lstlisting}
\subsubsection{Ejemplo de la diapositiva}
Posterior a implementar los pseudocódigos en Python sin ningún tipo de error al ejecutarlo, analizamos que estuviese trabajando correctamente y nos basamos en el ejemplo de la diapositiva, por lo que debíamos obtener los mismos resultados.\\\\
Primeramente, declaramos los arreglos de los beneficios y pesos y el peso \textit{P} de la mochila para después generar la tabla con la que posteriormente nos apoyaremos. Todo esto se muestra en la figura 22, y vemos que nuestra tabla \textit{g} se genera tal cual fue presentado en las diapositivas.
\begin{center}
    \includegraphics[width = 10cm]{MochilaEntera/1.png}\\
    Figura 22 - Generación de tabla auxiliar del ejemplo de la diapositiva
\end{center}
Seguidamente, llamamos a nuestra función \textit{test} que nos permitirá ver que objetos guardar. Obtuvimos los mismos resultados que se obtuvieron en la diapositiva, así como se muestra en la figura 23, por lo que nuestra implementación fue correcta.
\begin{center}
    \includegraphics[width = 10cm]{MochilaEntera/2.png}\\
    Figura 23 - Resultado en consola del algoritmo de la mochila entera con el ejemplo de la diapositiva.
\end{center}
Ya teniendo nuestro algoritmo funcionando correctamente y antes de continuar con la siguiente sección, donde veremos 10 ejemplos distintos usando este algoritmo, necesitarémos crear algunas funciones que nos permitan crear datos aleatoriamente, tales como arreglos. La función será \textit{crearLista}, donde le pasaremos el tamaño de la lista y el rango que querramos que tengan los elementos. En la figura 24 vemos efectivamente que nos regresa arreglos completamente aleatorios.
\begin{center}
    \includegraphics[width = 10cm]{MochilaEntera/3.png}\\
    Figura 24 - Resultado en consola de la función \textit{crearLista}
\end{center}
Después, creamos una función denominada \textit{mochilaEntera} que nos servirá como apoyo para poder mostrar mejor la información que nos regresan las dos funciones anteriores. A su vez, modificamos la función \textit{test} de modo que aceptase un nuevo parametro denominado \textit{terna}, un arreglo en el cual se irán almacenando los índices de los objetos a guardar para así obtener la terna de interés y poder calcular sus pesos y beneficios de estos objetos. En la figura 25 vemos su funcionamiento con el ejemplo de la diapositiva.
\begin{center}
    \includegraphics[width = 10cm]{MochilaEntera/4.png}\\
    Figura 25 - Resultado en consola de la función \textit{mochilaEntera}
\end{center}
\subsubsection{10 ejemplos del algoritmo de la mochila entera}
Para el primer ejemplo decidimos aumentar un poco el tamaño de los arreglos de las dos listas a 4 elementos, mientras que para el peso de la mochila solo se aumento un poco. La figura 26 nos muestra que el algoritmo funciona acorde a se planteo con un nuevo ejemplo completamente distinto y si lo analizamos observándolo nosotros, vemos que efectivamente dió con la solución óptima donde se obtiene el máximo beneficio posible.
\begin{center}
    \includegraphics[width = 10cm]{MochilaEntera/5.png}\\
    Figura 26 - Primer ejemplo de la mochila entera
\end{center}
Para el segundo ejemplo, aumentamos más nuestros valores, en lo que respecta al tamaño de los arreglos, los beneficios y la capacidad de la mochila, mientras que en los pesos de los objetos utilizamos valores relativamente pequeños y nuevamente en la figura 27 se muestra la solución óptima de la mochila entera.
\begin{center}
    \includegraphics[width = 10cm]{MochilaEntera/6.png}\\
    Figura 27 - Segundo ejemplo de la mochila entera
\end{center}
En el tercer ejemplo decidimos utilizar valores similares, y lo único que variamos fue el tamaño de los pesos de los objetos, donde utilizamos un rango que pudiese pasar la capacidad de la mochila, y como se ve en la figura 28, el algoritmo toma el objeto con mayor beneficio, por lo que continua obteniendo la solución óptima.
\begin{center}
    \includegraphics[width = 10cm]{MochilaEntera/7.png}\\
    Figura 28 - Tercer ejemplo de la mochila entera
\end{center}
Para el cuarto ejemplo decidimos bajar la cantidad de elementos de nuestras listas para observar el funcionamiento del algoritmo más a fondo, es decir, para poder observar como se crea la tabla con un ejemplo distinto al de la diapositiva. Primeramente creamos y asignamos listas aleatorias así como es mostrado en la figura 29
\begin{center}
    \includegraphics[width = 6cm]{MochilaEntera/8.png}\\
    Figura 29 - Cuarto ejemplo de la mochila entera: Asignación de listas aleatorias
\end{center}
Y posterior a esto generamos nuestra tabla \textit{g}, que en la figura 30 se muestra y notamos que es bastante similar a la tabla obtenida en el ejemplo de la diapositiva.
\begin{center}
    \includegraphics[width = 14cm]{MochilaEntera/9.png}\\
    Figura 30 - Cuarto ejemplo de la mochila entera: Generación de tabla auxiliar
\end{center}
Y una vez con todo esto, nuestra función \textit{test} que creamos inicialmente y con apoyo de la tabla auxiliar generada, es capaz de decirnos que objetos guardar para tener una solución óptima al problema de la mochila entera, y como lo mencionamos anteriormente, agregamos un nuevo parametro a esta función, por lo que antes necesitamos crear nuestra terna de ceros. En la figura 31 vemos los objetos a guardar con ayuda de la función \textit{test}
\begin{center}
    \includegraphics[width = 6cm]{MochilaEntera/10.png}\\
    Figura 31 - Cuarto ejemplo de la mochila entera: Función \textit{test}
\end{center}
Nuestra función \textit{mochilaEntera} engloba todo esto y nos da un mejor resumen, por lo que de igual manera llamamos y en la figura 32 vemos más claramente que efectivamente damos con la solución óptima.
\begin{center}
    \includegraphics[width = 8cm]{MochilaEntera/11.png}\\
    Figura 32 - Cuarto ejemplo de la mochila entera: Función \textit{mochilaEntera}
\end{center}
En nuestro quinto ejemplo continuamos observando las soluciones óptimas que nuestro algoritmo da, establecimos 10 elementos en nuestros arreglos y variamos significativamente los valores de los beneficios de los objetos, mientras que los pesos los mantuvimos más cerrados unos con otros y el peso de la mochila lo establecimos en 60. En la figura 33 vemos el resultado de este ejemplo.
\begin{center}
    \includegraphics[width = 10cm]{MochilaEntera/12.png}\\
    Figura 33 - Quinto ejemplo de la mochila entera
\end{center}
Posteriormente duplicamos nuestros valores, para el tamaño de arreglos, mantuvimos un rango considerablemente amplio para los beneficios de las mochilas y de los pesos. Obtuvimos el peso exacto y el beneficio máximo en nuestra solución, esto es mostrado en la figura 34.
\begin{center}
    \includegraphics[width = 14cm]{MochilaEntera/13.png}\\
    Figura 34 - Sexto ejemplo de la mochila entera
\end{center}
Ya que tuvimos un número pequeño de objetos, comparado con el número de elementos, decidimos variar más el rango de nuestros pesos y duplicar la capacidad de la mochila y obtuvimos efectivamente más objetos y por ende más beneficios, como se muestra en la figura 35.
\begin{center}
    \includegraphics[width = 14cm]{MochilaEntera/14.png}\\
    Figura 35 - Séptimo ejemplo de la mochila entera
\end{center}
En nuestro octavo ejemplo decidimos retomar una pregunta cuestionada en la práctica 5, lo que ocurre cuando los beneficios y pesos tienen el mismo valor. Primero, creamos nuestras listas con valores iguales y al peso le asignamos un $10$ así como es mostrado en la figura 36.
\begin{center}
    \includegraphics[width = 7cm]{MochilaEntera/15.png}\\
    Figura 36 - Octavo ejemplo de la mochila entera: Asignación de listas con elementos iguales
\end{center}
Y luego pasamos a observar como es que se genera la tabla auxiliar con mismos beneficios y pesos, en donde podemos observar algo bastante peculiar, la fila 0 como en cualquier otro caso esta llena de ceros, luego en la fila vemos que se llena la tabla del $0$ al $1$ hasta la columna $10$, empezando por el $1$ en la columna $1$, para la fila $2$ tenemos del $0$ al $2$ hasta la columna $2$ y las demás columnas se llenan con $2$, para la fila $3$ se tiene del $0$ al $3$ hasta la columna $3$ y las demás columnas se llenan con $3$ y asi sucesivamente hasta llegar a la fila $11$, donde de ahí hasta la fila $15$ (que es igual al peso) se llena del $0$ al $10$, si tuvieramos más peso, por ejemplo $12$, este fenomeno ocurriría hasta la fila $12$ y así para distintos pesos o números de elementos, lo cual resulta bastante distintivo a las demás tablas que habíamos estado generando, donde solíamos ver muchos ceros. Todo esto se muestra en la figura 37.
\begin{center}
    \includegraphics[width = 5cm]{MochilaEntera/16.png}\\
    Figura 37 - Octavo ejemplo de la mochila entera: Generación de la tabla \textit{g}
\end{center}
En seguida, vemos que el algoritmo sigue realizando lo que se le pide, encontrar la solución óptima, aún teniendo mismos valores en los elementos de los arreglos de pesos y beneficios, y toma los primeros de ellos, tal como se muestra en la figura 38.
\begin{center}
    \includegraphics[width = 9cm]{MochilaEntera/17.png}\\
    Figura 38 - Octavo ejemplo de la mochila entera: Función \textit{mochilaEntera}
\end{center}
En los 2 últimos ejemplos aumentarémos el tamaño del número de elementos de nuestras listas así como los valores que tenemos dentro de ellos y de igual forma la capacidad de la mochila para observar como se comporta nuestro algoritmo. Para el ejemplo número 9 tenemos un arreglo de beneficios que varía de los $200$-$300$ en su valor, mientras que para los pesos varían de los $50$-$150$ con 25 de tamaño en cada arreglo y la capacidad de la mochila está dada por $700$, los resultados se muestran en la figura 39
\begin{center}
    \includegraphics[width = 14cm]{MochilaEntera/18.png}\\
    Figura 39 - Noveno ejemplo de la mochila entera
\end{center}
Por último incrementamos a 30 el tamaño de nuestros arreglos y el rango de los vlaores de los beneficios van del $100$-$900$, para los pesos van del $25$-$250$, la capacidad de la mochila fue de $1000$. Con todos estos ejemplos vimos que el beneficio no importa mucho que tanto varíe, pero los pesos son donde se determina que tantos objetos puede llevar la mochila y es donde entra la utilidad de este algoritmo, que en todos los ejemplos dados, siempre nos arrojo la solución óptima. La figura 40 nos muestra el último ejemplo de este algoritmo de la mochila
\begin{center}
    \includegraphics[width = 14cm]{MochilaEntera/18.png}\\
    Figura 40 - Décimo ejemplo de la mochila entera
\end{center}
\newpage
\subsection{Algoritmo de Lineas de Producción}
Por último, en esta práctica, estudiaremos el algoritmo que nos permite resolver el problema de obtener el mejor tiempo en una línea de producción. Se mostrarán algunos ejemplos y observaremos como se resuelven estas problemáticas mediante programación dinámica
\subsubsection{Pseudocódigos del algoritmo de líneas de producción}
Primero que todo, mostraremos el pseudocódigo en el que nos basaremos para posteriormente implementar el código en Python.\\\\
El primer pseudocódigo nos permite generar 2 tablas, en una de ellas podemos saber que recorrido es el mejor acorde a los tiempos que se tienen en todas las estaciones de trabajo y en aquellas que también nos permiten cambiarnos de una línea a la otra; mientras que en la otra tabla se nos dan los indices de las estaciones de donde provino para obtener esa solución óptima. La denominamos \textit{generarTablas} y se muestra a continuación.
\begin{lstlisting}
generarTablas(n, a[1,2][1,..,n],t[1,2][1,..,n],e[1,2],x[1,2])
1-  f[1][1] = e[1] + a[1][1]
2-  f[2][1] = e[2] + a[2][1]
3-  for j = 2 to n do
4-      if f[1][j-1]+a[1][j]<=f[2][j-1]+t[2][j-1]+a[1][j]
5-          f[1][j] = f[1][j-1] + a[1][j]
6-          I[1][j] = 1
7-      else
8-          f[1][j] = f[2][j-1] + t[2][j-1] + a[1][j]
9-          I[1][j] = 2
10-     if f[2][j-1]+a[2][j]<=f[1][j-1]+t[2][j-1]+a[2][j]
11-         f[2][j] = f[2][j-1] + a[2][j]
12-         I[2][j] = 2
13-     else
14-         f[2][j] = f[1][j-1] + t[1][j-1] + a[2][j]
15-         I[2][j] = 1
16- if f[1][n] + x[1] <= f[2][n] + x[2]
17-     f* = f[1][n] + x[1]
18-     I* = 1
19- else
20-     f* = f[2][n] + x[2]
21-     I* = 2
23- return f, f*, I, I*
\end{lstlisting}
Por otro lado, el otro algoritmo es para imprimir el recorrido que se debe hacer para tener nuestra solución óptima. A esta función la denominamos \textit{imprimirLineaProduccion}.
\begin{lstlisting}
imprimirLineaProduccion(n, I, I*)
1-  i = I*
2-  print "línea", i, "estación", n
3-  for j = n downto 2
4-      i = I[i][j]
5-      print "línea", i, "estación", j-1
\end{lstlisting}
\subsubsection{Implementación y ejemplo de la diapositiva}
A continuación implementamos el pseudocódigo antes mencionado y verificamos que estuviese dando la solución óptima con apoyo del ejemplo dado en las diapositivas.\\\\
Primeramente empezamos por implementar la función de \textit{generarTablas}, donde solo tuvimos que acomodar los índices de nuestro pseudocódigo para que pudiese trabajar correctamente este algoritmo. Declaramos las variables para las estaciones, que en este caso fue una matriz de $2\times6$ y del mismo modo la variable para el tiempo de traslado de línea a línea que en este caso fue una matriz de $2\times5$, también se declararon los arreglos de tamaño 2 para los tiempos de inicio y los tiempos finales, todas estas variables sacadas del ejemplo de la diapositiva. En la figura 41 mostramos que en efecto, esta función nos genera las tablas que son mostradas en las diapositivas, por lo que nuestro código fue implementado correctamente.
\begin{center}
    \includegraphics[width = 7cm]{LineasProduccion/1.png}\\
    Figura 41 - Resultado en consola de la función \textit{generarTablas}
\end{center}
En seguida, implementamos la función para imprimir las líneas de producción, pero hicimos un pequeño cambio a lo propuesto en el pseudocódigo, en lugar de usar un ciclo \textit{for}, hicimos la función recursiva, de modo que al momento de imprimir el recorrido que se tiene que hacer lo imprima de inicio a fin y no de fin a inicio como se muestra en la diapositiva. La figura 42 nos muestra el funcionamiento de \textit{imprimirLineaProduccion}
\begin{center}
    \includegraphics[width = 7cm]{LineasProduccion/2.png}\\
    Figura 42 - Resultado en consola de la función \textit{imprimirLineaProduccion}
\end{center}
\subsubsection{Ejemplos}
Para realizar ejemplos lo más arbitrarios que se puedan y a su vez mostrar el completo funcionamiento de este algoritmo, tendremos que implementar algunas funciones que nos permitan crear listas y matrices $2\times n$ aleatorias y así como darles un rango a los elementos que aparezcan dentro de estas.\\\\
La primera de ellas ya la hemos implementado en multiples ocasiones, la denominamos \textit{crearLista} y recibe como parametros: el tamaño de la lista, límite inferior del rango, límite superior del rango. Su funcionamiento se muestra en la figura 43.
\begin{center}
    \includegraphics[width = 7cm]{LineasProduccion/3.png}\\
    Figura 43 - Resultado en consola de la función \textit{crearLista}
\end{center}
Seguidamente, creamos la función \textit{crearMatriz} y se recibe los parametros del número de filas y el número de columnas y el rango de los números que llevará dentro. En la figura 44 vemos que nos regresa arreglos dentro de otro arreglo, que es otra manera en la que se puede interpretar una matriz.
\begin{center}
    \includegraphics[width = 10cm]{LineasProduccion/4.png}\\
    Figura 44 - Resultado en consola de la función \textit{crearMatriz}
\end{center}
Posterior a esto, para visualizar mejor nuestros ejemplos, desarrollamos una función que solamente se encarga de mostrar los valores de nuestras matrices y listas de manera parecida a lo que vimos en las diapositivas, la denominamos \textit{imprimirFigura} y la figura 45 nos muestra como funciona con el ejemplo de la diapositiva.
\begin{center}
    \includegraphics[width = 9.5cm]{LineasProduccion/5.png}\\
    Figura 45 - Resultado en consola de la función \textit{imprimirFigura}
\end{center}
Continuando con funciones que permitan una mejor visualización, implementamos \textit{imprimirTabla} que nos imprime de una mejor manera las matrices, esta función únicamente recibe la matriz, el incremento de los indices de las cabeceras de las columnas y el caracter de las cabeceras de las filas, en la figura 46 vemos como funciona más claramente.
\begin{center}
    \includegraphics[width = 9.5cm]{LineasProduccion/6.png}\\
    Figura 46 - Resultado en consola de la función \textit{imprimirTabla}
\end{center}
Todas estas funciones que se describieron las juntamos en una última función denominada \textit{lineaprod} que recibe un número \textit{n} que representa el número de estaciones, un número \textit{x} para el límite inferior de los tiempos de estación y un número \textit{y} para el límite superior de los tiempos de estación. Esta función será la que nos permita crear ejemplos lo más arbitrarios posibles y visualizar la información de la salida de nuestros algoritmos de una mejor manera. De igual manera implementamos una función \textit{lineaprod2} que permite hacer lo mismo que \textit{lineaprod} pero recibiendo las matrices y listas. En la figura 47 vemos el funcionamiento de \textit{lineaprod2} con el ejemplo de la diapositiva\\\\
\begin{center}
    \includegraphics[width = 6cm]{LineasProduccion/7.png}\\
    Figura 47 - Resultado en consola de la función \textit{lineaprod2}
\end{center}
Por lo que para el primer ejemplo, decidimos usar una \textit{n} no tan grande y los tiempos de igual manera fueron con un rango no tan amplio, esto para observar que efectivamente este calculando la solución óptima. El ejemplo que se generó, mostrado en la figura 48, nos muestra que se debe recorrer toda la línea 2 y si analizamos un poco, en efecto, esta es la solución óptima, porque al parecer la línea 1 también tiene tiempos cortos, pero la línea 2 es la que tiene el tiempo mínimo.
\begin{center}
    \includegraphics[width = 5cm]{LineasProduccion/8.png}\\
    Figura 48 - Primer ejemplo del algoritmo de las líneas de producción
\end{center}
A partir de lo obtenido en consola, facilmente podemos realizar una figura más gráfica que nos muestre la ruta óptima, tal como lo muestra la figura 49.
\begin{center}


\tikzset{every picture/.style={line width=0.75pt}} %set default line width to 0.75pt

\begin{tikzpicture}[x=0.75pt,y=0.75pt,yscale=-1,xscale=1]
%uncomment if require: \path (0,300); %set diagram left start at 0, and has height of 300

%Shape: Rectangle [id:dp13069199363835082]
\draw   (126,20) -- (339.5,20) -- (339.5,60) -- (126,60) -- cycle ;
%Flowchart: Connector [id:dp5512299288853035]
\draw   (159.5,39.75) .. controls (159.5,31.05) and (166.55,24) .. (175.25,24) .. controls (183.95,24) and (191,31.05) .. (191,39.75) .. controls (191,48.45) and (183.95,55.5) .. (175.25,55.5) .. controls (166.55,55.5) and (159.5,48.45) .. (159.5,39.75) -- cycle ;
%Flowchart: Connector [id:dp2189849636881922]
\draw   (227.5,40.75) .. controls (227.5,32.05) and (234.55,25) .. (243.25,25) .. controls (251.95,25) and (259,32.05) .. (259,40.75) .. controls (259,49.45) and (251.95,56.5) .. (243.25,56.5) .. controls (234.55,56.5) and (227.5,49.45) .. (227.5,40.75) -- cycle ;
%Flowchart: Connector [id:dp3456708564232924]
\draw   (278.25,40) .. controls (278.25,31.3) and (285.3,24.25) .. (294,24.25) .. controls (302.7,24.25) and (309.75,31.3) .. (309.75,40) .. controls (309.75,48.7) and (302.7,55.75) .. (294,55.75) .. controls (285.3,55.75) and (278.25,48.7) .. (278.25,40) -- cycle ;
%Straight Lines [id:da18807705600360936]
\draw    (193.5,10) -- (193.5,212.25) ;
%Flowchart: Connector [id:dp5652329118158956]
\draw   (195,89.13) .. controls (195,82.15) and (200.65,76.5) .. (207.63,76.5) .. controls (214.6,76.5) and (220.25,82.15) .. (220.25,89.13) .. controls (220.25,96.1) and (214.6,101.75) .. (207.63,101.75) .. controls (200.65,101.75) and (195,96.1) .. (195,89.13) -- cycle ;
%Flowchart: Connector [id:dp8746979631472311]
\draw   (195.5,123.88) .. controls (195.5,117.04) and (201.04,111.5) .. (207.88,111.5) .. controls (214.71,111.5) and (220.25,117.04) .. (220.25,123.88) .. controls (220.25,130.71) and (214.71,136.25) .. (207.88,136.25) .. controls (201.04,136.25) and (195.5,130.71) .. (195.5,123.88) -- cycle ;
%Shape: Rectangle [id:dp24933975270395403]
\draw   (128,152.5) -- (341.5,152.5) -- (341.5,192.5) -- (128,192.5) -- cycle ;
%Flowchart: Connector [id:dp5275645111903269]
\draw  [color={rgb, 255:red, 255; green, 0; blue, 0 }  ,draw opacity=1 ] (160,172.25) .. controls (160,163.55) and (167.05,156.5) .. (175.75,156.5) .. controls (184.45,156.5) and (191.5,163.55) .. (191.5,172.25) .. controls (191.5,180.95) and (184.45,188) .. (175.75,188) .. controls (167.05,188) and (160,180.95) .. (160,172.25) -- cycle ;
%Flowchart: Connector [id:dp22923642511463815]
\draw  [color={rgb, 255:red, 254; green, 0; blue, 0 }  ,draw opacity=1 ] (228,173.25) .. controls (228,164.55) and (235.05,157.5) .. (243.75,157.5) .. controls (252.45,157.5) and (259.5,164.55) .. (259.5,173.25) .. controls (259.5,181.95) and (252.45,189) .. (243.75,189) .. controls (235.05,189) and (228,181.95) .. (228,173.25) -- cycle ;
%Flowchart: Connector [id:dp21544296039486022]
\draw  [color={rgb, 255:red, 255; green, 0; blue, 0 }  ,draw opacity=1 ] (280.25,172.5) .. controls (280.25,163.8) and (287.3,156.75) .. (296,156.75) .. controls (304.7,156.75) and (311.75,163.8) .. (311.75,172.5) .. controls (311.75,181.2) and (304.7,188.25) .. (296,188.25) .. controls (287.3,188.25) and (280.25,181.2) .. (280.25,172.5) -- cycle ;
%Straight Lines [id:da730217629466098]
\draw    (261,10.5) -- (261,212.75) ;
%Flowchart: Connector [id:dp5337912594520517]
\draw   (262.5,89.13) .. controls (262.5,82.15) and (268.15,76.5) .. (275.13,76.5) .. controls (282.1,76.5) and (287.75,82.15) .. (287.75,89.13) .. controls (287.75,96.1) and (282.1,101.75) .. (275.13,101.75) .. controls (268.15,101.75) and (262.5,96.1) .. (262.5,89.13) -- cycle ;
%Flowchart: Connector [id:dp7664055959089207]
\draw   (263,123.88) .. controls (263,117.04) and (268.54,111.5) .. (275.38,111.5) .. controls (282.21,111.5) and (287.75,117.04) .. (287.75,123.88) .. controls (287.75,130.71) and (282.21,136.25) .. (275.38,136.25) .. controls (268.54,136.25) and (263,130.71) .. (263,123.88) -- cycle ;
%Flowchart: Connector [id:dp013345235493830376]
\draw   (90,83.75) .. controls (90,75.88) and (96.38,69.5) .. (104.25,69.5) .. controls (112.12,69.5) and (118.5,75.88) .. (118.5,83.75) .. controls (118.5,91.62) and (112.12,98) .. (104.25,98) .. controls (96.38,98) and (90,91.62) .. (90,83.75) -- cycle ;
%Flowchart: Connector [id:dp5977114953484333]
\draw  [color={rgb, 255:red, 255; green, 0; blue, 0 }  ,draw opacity=1 ] (90.5,123.75) .. controls (90.5,115.88) and (96.88,109.5) .. (104.75,109.5) .. controls (112.62,109.5) and (119,115.88) .. (119,123.75) .. controls (119,131.62) and (112.62,138) .. (104.75,138) .. controls (96.88,138) and (90.5,131.62) .. (90.5,123.75) -- cycle ;
%Flowchart: Connector [id:dp6565608977902386]
\draw   (348.5,86.25) .. controls (348.5,78.38) and (354.88,72) .. (362.75,72) .. controls (370.62,72) and (377,78.38) .. (377,86.25) .. controls (377,94.12) and (370.62,100.5) .. (362.75,100.5) .. controls (354.88,100.5) and (348.5,94.12) .. (348.5,86.25) -- cycle ;
%Flowchart: Connector [id:dp6077894932878212]
\draw  [color={rgb, 255:red, 254; green, 0; blue, 0 }  ,draw opacity=1 ] (349,125.25) .. controls (349,117.38) and (355.38,111) .. (363.25,111) .. controls (371.12,111) and (377.5,117.38) .. (377.5,125.25) .. controls (377.5,133.12) and (371.12,139.5) .. (363.25,139.5) .. controls (355.38,139.5) and (349,133.12) .. (349,125.25) -- cycle ;
%Straight Lines [id:da5513658599215276]
\draw [color={rgb, 255:red, 251; green, 1; blue, 1 }  ,draw opacity=1 ]   (120,127.25) -- (160.52,163.91) ;
\draw [shift={(162,165.25)}, rotate = 222.14] [color={rgb, 255:red, 251; green, 1; blue, 1 }  ,draw opacity=1 ][line width=0.75]    (10.93,-3.29) .. controls (6.95,-1.4) and (3.31,-0.3) .. (0,0) .. controls (3.31,0.3) and (6.95,1.4) .. (10.93,3.29)   ;
%Straight Lines [id:da9301826708585905]
\draw [color={rgb, 255:red, 251; green, 1; blue, 1 }  ,draw opacity=1 ]   (191.5,170.75) -- (225,170.75) ;
\draw [shift={(227,170.75)}, rotate = 180] [color={rgb, 255:red, 251; green, 1; blue, 1 }  ,draw opacity=1 ][line width=0.75]    (10.93,-3.29) .. controls (6.95,-1.4) and (3.31,-0.3) .. (0,0) .. controls (3.31,0.3) and (6.95,1.4) .. (10.93,3.29)   ;
%Straight Lines [id:da7269885512779708]
\draw [color={rgb, 255:red, 251; green, 1; blue, 1 }  ,draw opacity=1 ]   (260.5,172.5) -- (278.25,172.5) ;
\draw [shift={(280.25,172.5)}, rotate = 180] [color={rgb, 255:red, 251; green, 1; blue, 1 }  ,draw opacity=1 ][line width=0.75]    (10.93,-3.29) .. controls (6.95,-1.4) and (3.31,-0.3) .. (0,0) .. controls (3.31,0.3) and (6.95,1.4) .. (10.93,3.29)   ;
%Straight Lines [id:da941206114231756]
\draw [color={rgb, 255:red, 251; green, 1; blue, 1 }  ,draw opacity=1 ]   (312,167.25) -- (350.95,135.51) ;
\draw [shift={(352.5,134.25)}, rotate = 500.83] [color={rgb, 255:red, 251; green, 1; blue, 1 }  ,draw opacity=1 ][line width=0.75]    (10.93,-3.29) .. controls (6.95,-1.4) and (3.31,-0.3) .. (0,0) .. controls (3.31,0.3) and (6.95,1.4) .. (10.93,3.29)   ;

% Text Node
\draw (170,32) node [anchor=north west][inner sep=0.75pt]   [align=left] {2};
% Text Node
\draw (237,32) node [anchor=north west][inner sep=0.75pt]   [align=left] {1};
% Text Node
\draw (289,32) node [anchor=north west][inner sep=0.75pt]   [align=left] {2};
% Text Node
\draw (170,163.5) node [anchor=north west][inner sep=0.75pt]   [align=left] {2};
% Text Node
\draw (237,164.5) node [anchor=north west][inner sep=0.75pt]   [align=left] {1};
% Text Node
\draw (291,164) node [anchor=north west][inner sep=0.75pt]   [align=left] {3};
% Text Node
\draw (358,118) node [anchor=north west][inner sep=0.75pt]   [align=left] {2};
% Text Node
\draw (358,77.5) node [anchor=north west][inner sep=0.75pt]   [align=left] {4};
% Text Node
\draw (269.5,81) node [anchor=north west][inner sep=0.75pt]   [align=left] {5};
% Text Node
\draw (270.5,114.5) node [anchor=north west][inner sep=0.75pt]   [align=left] {3};
% Text Node
\draw (203,80) node [anchor=north west][inner sep=0.75pt]   [align=left] {5};
% Text Node
\draw (201.5,116) node [anchor=north west][inner sep=0.75pt]   [align=left] {1};
% Text Node
\draw (98,77) node [anchor=north west][inner sep=0.75pt]   [align=left] {1};
% Text Node
\draw (99,115) node [anchor=north west][inner sep=0.75pt]   [align=left] {2};


\end{tikzpicture}

    \\Figura 49 - Ruta óptima del primer ejemplo
\end{center}
En el segundo ejemplo aumentamos un poco el número de estaciones y el rango de tiempo de estas mismas. Coincidio que en este ejemplo la ruta a seguir se da por solo una línea, pero en este caso es por la línea 1 donde obtenemos el mejor tiempo y nuevamente si observamos a detalle esta solución es la óptima. Este ejemplo es mostrado en la figura 50.
\begin{center}
    \includegraphics[width = 8.5cm]{LineasProduccion/9.png}\\
    Figura 50 - Segundo ejemplo del algoritmo de las líneas de producción
\end{center}
Con lo anterior, podemos concluir lo que se muestra en la figura 51.
\begin{center}


\tikzset{every picture/.style={line width=0.75pt}} %set default line width to 0.75pt

\begin{tikzpicture}[x=0.75pt,y=0.75pt,yscale=-1,xscale=1]
%uncomment if require: \path (0,432); %set diagram left start at 0, and has height of 432

%Shape: Rectangle [id:dp3685700062544466]
\draw   (119,41.77) -- (509.5,41.77) -- (509.5,101.2) -- (119,101.2) -- cycle ;
%Shape: Ellipse [id:dp9649156214061496]
\draw  [color={rgb, 255:red, 255; green, 0; blue, 0 }  ,draw opacity=1 ] (131,72.97) .. controls (131,62.31) and (139.73,53.66) .. (150.5,53.66) .. controls (161.27,53.66) and (170,62.31) .. (170,72.97) .. controls (170,83.64) and (161.27,92.29) .. (150.5,92.29) .. controls (139.73,92.29) and (131,83.64) .. (131,72.97) -- cycle ;
%Shape: Ellipse [id:dp6329021506746515]
\draw  [color={rgb, 255:red, 255; green, 0; blue, 0 }  ,draw opacity=1 ] (215,72.97) .. controls (215,62.31) and (223.73,53.66) .. (234.5,53.66) .. controls (245.27,53.66) and (254,62.31) .. (254,72.97) .. controls (254,83.64) and (245.27,92.29) .. (234.5,92.29) .. controls (223.73,92.29) and (215,83.64) .. (215,72.97) -- cycle ;
%Shape: Ellipse [id:dp46085585699592846]
\draw  [color={rgb, 255:red, 255; green, 0; blue, 0 }  ,draw opacity=1 ] (300,70.99) .. controls (300,60.32) and (308.73,51.68) .. (319.5,51.68) .. controls (330.27,51.68) and (339,60.32) .. (339,70.99) .. controls (339,81.66) and (330.27,90.31) .. (319.5,90.31) .. controls (308.73,90.31) and (300,81.66) .. (300,70.99) -- cycle ;
%Shape: Ellipse [id:dp6724104265963096]
\draw  [color={rgb, 255:red, 250; green, 0; blue, 0 }  ,draw opacity=1 ] (383,70.99) .. controls (383,60.32) and (391.73,51.68) .. (402.5,51.68) .. controls (413.27,51.68) and (422,60.32) .. (422,70.99) .. controls (422,81.66) and (413.27,90.31) .. (402.5,90.31) .. controls (391.73,90.31) and (383,81.66) .. (383,70.99) -- cycle ;
%Shape: Ellipse [id:dp51290516302854]
\draw  [color={rgb, 255:red, 255; green, 0; blue, 0 }  ,draw opacity=1 ] (47,142.31) .. controls (47,131.64) and (55.73,122.99) .. (66.5,122.99) .. controls (77.27,122.99) and (86,131.64) .. (86,142.31) .. controls (86,152.98) and (77.27,161.62) .. (66.5,161.62) .. controls (55.73,161.62) and (47,152.98) .. (47,142.31) -- cycle ;
%Shape: Ellipse [id:dp6625964077851083]
\draw   (48,214.62) .. controls (48,203.95) and (56.73,195.3) .. (67.5,195.3) .. controls (78.27,195.3) and (87,203.95) .. (87,214.62) .. controls (87,225.28) and (78.27,233.93) .. (67.5,233.93) .. controls (56.73,233.93) and (48,225.28) .. (48,214.62) -- cycle ;
%Shape: Ellipse [id:dp6236617218136584]
\draw  [color={rgb, 255:red, 255; green, 0; blue, 0 }  ,draw opacity=1 ] (536,143.3) .. controls (536,132.63) and (544.73,123.98) .. (555.5,123.98) .. controls (566.27,123.98) and (575,132.63) .. (575,143.3) .. controls (575,153.97) and (566.27,162.61) .. (555.5,162.61) .. controls (544.73,162.61) and (536,153.97) .. (536,143.3) -- cycle ;
%Shape: Ellipse [id:dp9293782542736218]
\draw   (535,214.62) .. controls (535,203.95) and (543.73,195.3) .. (554.5,195.3) .. controls (565.27,195.3) and (574,203.95) .. (574,214.62) .. controls (574,225.28) and (565.27,233.93) .. (554.5,233.93) .. controls (543.73,233.93) and (535,225.28) .. (535,214.62) -- cycle ;
%Shape: Rectangle [id:dp43119014619460905]
\draw   (119,249.78) -- (508.5,249.78) -- (508.5,309.21) -- (119,309.21) -- cycle ;
%Shape: Ellipse [id:dp5845055683378779]
\draw   (131,280.98) .. controls (131,270.31) and (139.73,261.66) .. (150.5,261.66) .. controls (161.27,261.66) and (170,270.31) .. (170,280.98) .. controls (170,291.65) and (161.27,300.29) .. (150.5,300.29) .. controls (139.73,300.29) and (131,291.65) .. (131,280.98) -- cycle ;
%Shape: Ellipse [id:dp48987990927874736]
\draw   (215,280.98) .. controls (215,270.31) and (223.73,261.66) .. (234.5,261.66) .. controls (245.27,261.66) and (254,270.31) .. (254,280.98) .. controls (254,291.65) and (245.27,300.29) .. (234.5,300.29) .. controls (223.73,300.29) and (215,291.65) .. (215,280.98) -- cycle ;
%Shape: Ellipse [id:dp15606055372527372]
\draw   (300,279.99) .. controls (300,269.32) and (308.73,260.67) .. (319.5,260.67) .. controls (330.27,260.67) and (339,269.32) .. (339,279.99) .. controls (339,290.66) and (330.27,299.3) .. (319.5,299.3) .. controls (308.73,299.3) and (300,290.66) .. (300,279.99) -- cycle ;
%Shape: Ellipse [id:dp032602110302615284]
\draw   (383,279.99) .. controls (383,269.32) and (391.73,260.67) .. (402.5,260.67) .. controls (413.27,260.67) and (422,269.32) .. (422,279.99) .. controls (422,290.66) and (413.27,299.3) .. (402.5,299.3) .. controls (391.73,299.3) and (383,290.66) .. (383,279.99) -- cycle ;
%Shape: Ellipse [id:dp7718255418535962]
\draw   (178,147.01) .. controls (178,139.77) and (183.93,133.89) .. (191.25,133.89) .. controls (198.57,133.89) and (204.5,139.77) .. (204.5,147.01) .. controls (204.5,154.26) and (198.57,160.14) .. (191.25,160.14) .. controls (183.93,160.14) and (178,154.26) .. (178,147.01) -- cycle ;
%Shape: Ellipse [id:dp4964089034365795]
\draw   (178,208.42) .. controls (178,201.18) and (183.93,195.3) .. (191.25,195.3) .. controls (198.57,195.3) and (204.5,201.18) .. (204.5,208.42) .. controls (204.5,215.67) and (198.57,221.55) .. (191.25,221.55) .. controls (183.93,221.55) and (178,215.67) .. (178,208.42) -- cycle ;
%Shape: Ellipse [id:dp17642671971575474]
\draw   (264,147.01) .. controls (264,139.77) and (269.93,133.89) .. (277.25,133.89) .. controls (284.57,133.89) and (290.5,139.77) .. (290.5,147.01) .. controls (290.5,154.26) and (284.57,160.14) .. (277.25,160.14) .. controls (269.93,160.14) and (264,154.26) .. (264,147.01) -- cycle ;
%Shape: Ellipse [id:dp9659876401787248]
\draw   (264,208.42) .. controls (264,201.18) and (269.93,195.3) .. (277.25,195.3) .. controls (284.57,195.3) and (290.5,201.18) .. (290.5,208.42) .. controls (290.5,215.67) and (284.57,221.55) .. (277.25,221.55) .. controls (269.93,221.55) and (264,215.67) .. (264,208.42) -- cycle ;
%Shape: Ellipse [id:dp440659244864692]
\draw   (347,147.01) .. controls (347,139.77) and (352.93,133.89) .. (360.25,133.89) .. controls (367.57,133.89) and (373.5,139.77) .. (373.5,147.01) .. controls (373.5,154.26) and (367.57,160.14) .. (360.25,160.14) .. controls (352.93,160.14) and (347,154.26) .. (347,147.01) -- cycle ;
%Shape: Ellipse [id:dp9286960858057189]
\draw   (348,208.42) .. controls (348,201.18) and (353.93,195.3) .. (361.25,195.3) .. controls (368.57,195.3) and (374.5,201.18) .. (374.5,208.42) .. controls (374.5,215.67) and (368.57,221.55) .. (361.25,221.55) .. controls (353.93,221.55) and (348,215.67) .. (348,208.42) -- cycle ;
%Straight Lines [id:da662792588053928]
\draw    (174.5,32.86) -- (174.5,328.03) ;
%Straight Lines [id:da5941487383274984]
\draw    (259.5,36.86) -- (259.5,332.03) ;
%Straight Lines [id:da003425025072895105]
\draw    (344.5,37.86) -- (344.5,333.03) ;
%Straight Lines [id:da9077715205961372]
\draw [color={rgb, 255:red, 255; green, 0; blue, 0 }  ,draw opacity=1 ]   (79,128) -- (129.63,74.43) ;
\draw [shift={(131,72.97)}, rotate = 493.38] [color={rgb, 255:red, 255; green, 0; blue, 0 }  ,draw opacity=1 ][line width=0.75]    (10.93,-3.29) .. controls (6.95,-1.4) and (3.31,-0.3) .. (0,0) .. controls (3.31,0.3) and (6.95,1.4) .. (10.93,3.29)   ;
%Straight Lines [id:da9581084390032171]
\draw [color={rgb, 255:red, 255; green, 0; blue, 0 }  ,draw opacity=1 ]   (170,72.97) -- (213,72.97) ;
\draw [shift={(215,72.97)}, rotate = 180] [color={rgb, 255:red, 255; green, 0; blue, 0 }  ,draw opacity=1 ][line width=0.75]    (10.93,-3.29) .. controls (6.95,-1.4) and (3.31,-0.3) .. (0,0) .. controls (3.31,0.3) and (6.95,1.4) .. (10.93,3.29)   ;
%Straight Lines [id:da7871923244680363]
\draw [color={rgb, 255:red, 255; green, 0; blue, 0 }  ,draw opacity=1 ]   (254,72.97) -- (298,71.08) ;
\draw [shift={(300,70.99)}, rotate = 537.53] [color={rgb, 255:red, 255; green, 0; blue, 0 }  ,draw opacity=1 ][line width=0.75]    (10.93,-3.29) .. controls (6.95,-1.4) and (3.31,-0.3) .. (0,0) .. controls (3.31,0.3) and (6.95,1.4) .. (10.93,3.29)   ;
%Straight Lines [id:da5112394979869259]
\draw [color={rgb, 255:red, 255; green, 6; blue, 37 }  ,draw opacity=1 ]   (339,70.99) -- (381,70.99) ;
\draw [shift={(383,70.99)}, rotate = 180] [color={rgb, 255:red, 255; green, 6; blue, 37 }  ,draw opacity=1 ][line width=0.75]    (10.93,-3.29) .. controls (6.95,-1.4) and (3.31,-0.3) .. (0,0) .. controls (3.31,0.3) and (6.95,1.4) .. (10.93,3.29)   ;
%Straight Lines [id:da9126995185441131]
\draw    (426.5,36.86) -- (426.5,332.03) ;
%Shape: Ellipse [id:dp6768805467894146]
\draw  [color={rgb, 255:red, 250; green, 0; blue, 0 }  ,draw opacity=1 ] (460,71.99) .. controls (460,61.32) and (468.73,52.68) .. (479.5,52.68) .. controls (490.27,52.68) and (499,61.32) .. (499,71.99) .. controls (499,82.66) and (490.27,91.31) .. (479.5,91.31) .. controls (468.73,91.31) and (460,82.66) .. (460,71.99) -- cycle ;
%Shape: Ellipse [id:dp2809071485474479]
\draw  [color={rgb, 255:red, 0; green, 0; blue, 0 }  ,draw opacity=1 ] (460,279.99) .. controls (460,269.32) and (468.73,260.68) .. (479.5,260.68) .. controls (490.27,260.68) and (499,269.32) .. (499,279.99) .. controls (499,290.66) and (490.27,299.31) .. (479.5,299.31) .. controls (468.73,299.31) and (460,290.66) .. (460,279.99) -- cycle ;
%Straight Lines [id:da32198275011906907]
\draw [color={rgb, 255:red, 255; green, 6; blue, 37 }  ,draw opacity=1 ]   (422,70.99) -- (458,71.94) ;
\draw [shift={(460,71.99)}, rotate = 181.51] [color={rgb, 255:red, 255; green, 6; blue, 37 }  ,draw opacity=1 ][line width=0.75]    (10.93,-3.29) .. controls (6.95,-1.4) and (3.31,-0.3) .. (0,0) .. controls (3.31,0.3) and (6.95,1.4) .. (10.93,3.29)   ;
%Shape: Ellipse [id:dp24980388801748288]
\draw   (431,147.01) .. controls (431,139.77) and (436.93,133.89) .. (444.25,133.89) .. controls (451.57,133.89) and (457.5,139.77) .. (457.5,147.01) .. controls (457.5,154.26) and (451.57,160.14) .. (444.25,160.14) .. controls (436.93,160.14) and (431,154.26) .. (431,147.01) -- cycle ;
%Shape: Ellipse [id:dp10909438427230511]
\draw   (431,209.01) .. controls (431,201.77) and (436.93,195.89) .. (444.25,195.89) .. controls (451.57,195.89) and (457.5,201.77) .. (457.5,209.01) .. controls (457.5,216.26) and (451.57,222.14) .. (444.25,222.14) .. controls (436.93,222.14) and (431,216.26) .. (431,209.01) -- cycle ;
%Straight Lines [id:da8723757289521352]
\draw [color={rgb, 255:red, 255; green, 6; blue, 37 }  ,draw opacity=1 ]   (499,71.99) -- (554.03,122.63) ;
\draw [shift={(555.5,123.98)}, rotate = 222.62] [color={rgb, 255:red, 255; green, 6; blue, 37 }  ,draw opacity=1 ][line width=0.75]    (10.93,-3.29) .. controls (6.95,-1.4) and (3.31,-0.3) .. (0,0) .. controls (3.31,0.3) and (6.95,1.4) .. (10.93,3.29)   ;

% Text Node
\draw (147,64.47) node [anchor=north west][inner sep=0.75pt]   [align=left] {6};
% Text Node
\draw (230,64.47) node [anchor=north west][inner sep=0.75pt]   [align=left] {7};
% Text Node
\draw (314,63.48) node [anchor=north west][inner sep=0.75pt]   [align=left] {2};
% Text Node
\draw (398,62.49) node [anchor=north west][inner sep=0.75pt]   [align=left] {4};
% Text Node
\draw (62,133.81) node [anchor=north west][inner sep=0.75pt]   [align=left] {3};
% Text Node
\draw (549,134.8) node [anchor=north west][inner sep=0.75pt]   [align=left] {4};
% Text Node
\draw (549,206.13) node [anchor=north west][inner sep=0.75pt]   [align=left] {6};
% Text Node
\draw (398,271.49) node [anchor=north west][inner sep=0.75pt]   [align=left] {7};
% Text Node
\draw (314,271.49) node [anchor=north west][inner sep=0.75pt]   [align=left] {6};
% Text Node
\draw (229,272.48) node [anchor=north west][inner sep=0.75pt]   [align=left] {2};
% Text Node
\draw (145,272.48) node [anchor=north west][inner sep=0.75pt]   [align=left] {4};
% Text Node
\draw (62,206.12) node [anchor=north west][inner sep=0.75pt]   [align=left] {6};
% Text Node
\draw (186,138.76) node [anchor=north west][inner sep=0.75pt]   [align=left] {7};
% Text Node
\draw (187,200.17) node [anchor=north west][inner sep=0.75pt]   [align=left] {7};
% Text Node
\draw (272,138.76) node [anchor=north west][inner sep=0.75pt]   [align=left] {3};
% Text Node
\draw (272,201.16) node [anchor=north west][inner sep=0.75pt]   [align=left] {8};
% Text Node
\draw (355,138.76) node [anchor=north west][inner sep=0.75pt]   [align=left] {2};
% Text Node
\draw (356,200.17) node [anchor=north west][inner sep=0.75pt]   [align=left] {5};
% Text Node
\draw (187,19.9) node [anchor=north west][inner sep=0.75pt]   [align=left] {Linea de Producción 1};
% Text Node
\draw (163,338.1) node [anchor=north west][inner sep=0.75pt]   [align=left] {Linea de Producción 2};
% Text Node
\draw (474,64.49) node [anchor=north west][inner sep=0.75pt]   [align=left] {8};
% Text Node
\draw (475,272.49) node [anchor=north west][inner sep=0.75pt]   [align=left] {6};
% Text Node
\draw (439,138.76) node [anchor=north west][inner sep=0.75pt]   [align=left] {4};
% Text Node
\draw (439,200.76) node [anchor=north west][inner sep=0.75pt]   [align=left] {2};


\end{tikzpicture}

    Figura 51 - Ruta óptima del segundo ejemplo
\end{center}
El tercer ejemplo decidimos hacerlo algo parecido al mostrado en la diapositiva, pero este tomará valores aleatorios en sus listas y matrices. En la figura 52 observamos que ahora sí hubo un intercambio de línea ocurrido en la estación 3, pero vemos que practicamente la mitad del recorrido se mantuvieron en una misma línea. Aún así seguimos obteniendo las soluciones óptimas a los problemas que se han estado generando.
\begin{center}
    \includegraphics[width = 6.6cm]{LineasProduccion/95.png}\\
    Figura 52 - Tercer ejemplo del algoritmo de las líneas de producción
\end{center}
La figura 53 nos muestra esta solución óptima obtenida del tercer ejemplo.
\begin{center}


\tikzset{every picture/.style={line width=0.75pt}} %set default line width to 0.75pt

\begin{tikzpicture}[x=0.75pt,y=0.75pt,yscale=-1,xscale=1]
%uncomment if require: \path (0,432); %set diagram left start at 0, and has height of 432

%Shape: Rectangle [id:dp3685700062544466]
\draw   (82,47.77) -- (561.5,47.77) -- (561.5,107.2) -- (82,107.2) -- cycle ;
%Shape: Ellipse [id:dp9649156214061496]
\draw  [color={rgb, 255:red, 255; green, 0; blue, 0 }  ,draw opacity=1 ] (94,78.97) .. controls (94,68.31) and (102.73,59.66) .. (113.5,59.66) .. controls (124.27,59.66) and (133,68.31) .. (133,78.97) .. controls (133,89.64) and (124.27,98.29) .. (113.5,98.29) .. controls (102.73,98.29) and (94,89.64) .. (94,78.97) -- cycle ;
%Shape: Ellipse [id:dp6329021506746515]
\draw  [color={rgb, 255:red, 255; green, 0; blue, 0 }  ,draw opacity=1 ] (178,78.97) .. controls (178,68.31) and (186.73,59.66) .. (197.5,59.66) .. controls (208.27,59.66) and (217,68.31) .. (217,78.97) .. controls (217,89.64) and (208.27,98.29) .. (197.5,98.29) .. controls (186.73,98.29) and (178,89.64) .. (178,78.97) -- cycle ;
%Shape: Ellipse [id:dp46085585699592846]
\draw  [color={rgb, 255:red, 0; green, 0; blue, 0 }  ,draw opacity=1 ] (263,76.99) .. controls (263,66.32) and (271.73,57.68) .. (282.5,57.68) .. controls (293.27,57.68) and (302,66.32) .. (302,76.99) .. controls (302,87.66) and (293.27,96.31) .. (282.5,96.31) .. controls (271.73,96.31) and (263,87.66) .. (263,76.99) -- cycle ;
%Shape: Ellipse [id:dp6724104265963096]
\draw  [color={rgb, 255:red, 0; green, 0; blue, 0 }  ,draw opacity=1 ] (346,76.99) .. controls (346,66.32) and (354.73,57.68) .. (365.5,57.68) .. controls (376.27,57.68) and (385,66.32) .. (385,76.99) .. controls (385,87.66) and (376.27,96.31) .. (365.5,96.31) .. controls (354.73,96.31) and (346,87.66) .. (346,76.99) -- cycle ;
%Shape: Ellipse [id:dp51290516302854]
\draw  [color={rgb, 255:red, 255; green, 0; blue, 0 }  ,draw opacity=1 ] (10,148.31) .. controls (10,137.64) and (18.73,128.99) .. (29.5,128.99) .. controls (40.27,128.99) and (49,137.64) .. (49,148.31) .. controls (49,158.98) and (40.27,167.62) .. (29.5,167.62) .. controls (18.73,167.62) and (10,158.98) .. (10,148.31) -- cycle ;
%Shape: Ellipse [id:dp6625964077851083]
\draw   (11,220.62) .. controls (11,209.95) and (19.73,201.3) .. (30.5,201.3) .. controls (41.27,201.3) and (50,209.95) .. (50,220.62) .. controls (50,231.28) and (41.27,239.93) .. (30.5,239.93) .. controls (19.73,239.93) and (11,231.28) .. (11,220.62) -- cycle ;
%Shape: Ellipse [id:dp6236617218136584]
\draw  [color={rgb, 255:red, 0; green, 0; blue, 0 }  ,draw opacity=1 ] (592,149.3) .. controls (592,138.63) and (600.73,129.98) .. (611.5,129.98) .. controls (622.27,129.98) and (631,138.63) .. (631,149.3) .. controls (631,159.97) and (622.27,168.61) .. (611.5,168.61) .. controls (600.73,168.61) and (592,159.97) .. (592,149.3) -- cycle ;
%Shape: Ellipse [id:dp9293782542736218]
\draw  [color={rgb, 255:red, 250; green, 0; blue, 0 }  ,draw opacity=1 ] (592,220.62) .. controls (592,209.95) and (600.73,201.3) .. (611.5,201.3) .. controls (622.27,201.3) and (631,209.95) .. (631,220.62) .. controls (631,231.28) and (622.27,239.93) .. (611.5,239.93) .. controls (600.73,239.93) and (592,231.28) .. (592,220.62) -- cycle ;
%Shape: Rectangle [id:dp43119014619460905]
\draw   (82,255.78) -- (560.5,255.78) -- (560.5,315.21) -- (82,315.21) -- cycle ;
%Shape: Ellipse [id:dp5845055683378779]
\draw   (94,286.98) .. controls (94,276.31) and (102.73,267.66) .. (113.5,267.66) .. controls (124.27,267.66) and (133,276.31) .. (133,286.98) .. controls (133,297.65) and (124.27,306.29) .. (113.5,306.29) .. controls (102.73,306.29) and (94,297.65) .. (94,286.98) -- cycle ;
%Shape: Ellipse [id:dp48987990927874736]
\draw   (178,286.98) .. controls (178,276.31) and (186.73,267.66) .. (197.5,267.66) .. controls (208.27,267.66) and (217,276.31) .. (217,286.98) .. controls (217,297.65) and (208.27,306.29) .. (197.5,306.29) .. controls (186.73,306.29) and (178,297.65) .. (178,286.98) -- cycle ;
%Shape: Ellipse [id:dp15606055372527372]
\draw  [color={rgb, 255:red, 255; green, 0; blue, 0 }  ,draw opacity=1 ] (263,285.99) .. controls (263,275.32) and (271.73,266.67) .. (282.5,266.67) .. controls (293.27,266.67) and (302,275.32) .. (302,285.99) .. controls (302,296.66) and (293.27,305.3) .. (282.5,305.3) .. controls (271.73,305.3) and (263,296.66) .. (263,285.99) -- cycle ;
%Shape: Ellipse [id:dp032602110302615284]
\draw  [color={rgb, 255:red, 255; green, 0; blue, 0 }  ,draw opacity=1 ] (346,285.99) .. controls (346,275.32) and (354.73,266.67) .. (365.5,266.67) .. controls (376.27,266.67) and (385,275.32) .. (385,285.99) .. controls (385,296.66) and (376.27,305.3) .. (365.5,305.3) .. controls (354.73,305.3) and (346,296.66) .. (346,285.99) -- cycle ;
%Shape: Ellipse [id:dp7718255418535962]
\draw   (141,153.01) .. controls (141,145.77) and (146.93,139.89) .. (154.25,139.89) .. controls (161.57,139.89) and (167.5,145.77) .. (167.5,153.01) .. controls (167.5,160.26) and (161.57,166.14) .. (154.25,166.14) .. controls (146.93,166.14) and (141,160.26) .. (141,153.01) -- cycle ;
%Shape: Ellipse [id:dp4964089034365795]
\draw   (141,214.42) .. controls (141,207.18) and (146.93,201.3) .. (154.25,201.3) .. controls (161.57,201.3) and (167.5,207.18) .. (167.5,214.42) .. controls (167.5,221.67) and (161.57,227.55) .. (154.25,227.55) .. controls (146.93,227.55) and (141,221.67) .. (141,214.42) -- cycle ;
%Shape: Ellipse [id:dp17642671971575474]
\draw  [color={rgb, 255:red, 255; green, 0; blue, 0 }  ,draw opacity=1 ] (227,153.01) .. controls (227,145.77) and (232.93,139.89) .. (240.25,139.89) .. controls (247.57,139.89) and (253.5,145.77) .. (253.5,153.01) .. controls (253.5,160.26) and (247.57,166.14) .. (240.25,166.14) .. controls (232.93,166.14) and (227,160.26) .. (227,153.01) -- cycle ;
%Shape: Ellipse [id:dp9659876401787248]
\draw   (227,214.42) .. controls (227,207.18) and (232.93,201.3) .. (240.25,201.3) .. controls (247.57,201.3) and (253.5,207.18) .. (253.5,214.42) .. controls (253.5,221.67) and (247.57,227.55) .. (240.25,227.55) .. controls (232.93,227.55) and (227,221.67) .. (227,214.42) -- cycle ;
%Shape: Ellipse [id:dp440659244864692]
\draw   (310,153.01) .. controls (310,145.77) and (315.93,139.89) .. (323.25,139.89) .. controls (330.57,139.89) and (336.5,145.77) .. (336.5,153.01) .. controls (336.5,160.26) and (330.57,166.14) .. (323.25,166.14) .. controls (315.93,166.14) and (310,160.26) .. (310,153.01) -- cycle ;
%Shape: Ellipse [id:dp9286960858057189]
\draw   (311,214.42) .. controls (311,207.18) and (316.93,201.3) .. (324.25,201.3) .. controls (331.57,201.3) and (337.5,207.18) .. (337.5,214.42) .. controls (337.5,221.67) and (331.57,227.55) .. (324.25,227.55) .. controls (316.93,227.55) and (311,221.67) .. (311,214.42) -- cycle ;
%Straight Lines [id:da662792588053928]
\draw    (137.5,38.86) -- (137.5,334.03) ;
%Straight Lines [id:da5941487383274984]
\draw    (222.5,42.86) -- (222.5,338.03) ;
%Straight Lines [id:da003425025072895105]
\draw    (307.5,43.86) -- (307.5,339.03) ;
%Straight Lines [id:da9077715205961372]
\draw [color={rgb, 255:red, 255; green, 0; blue, 0 }  ,draw opacity=1 ]   (42,134) -- (92.63,80.43) ;
\draw [shift={(94,78.97)}, rotate = 493.38] [color={rgb, 255:red, 255; green, 0; blue, 0 }  ,draw opacity=1 ][line width=0.75]    (10.93,-3.29) .. controls (6.95,-1.4) and (3.31,-0.3) .. (0,0) .. controls (3.31,0.3) and (6.95,1.4) .. (10.93,3.29)   ;
%Straight Lines [id:da9581084390032171]
\draw [color={rgb, 255:red, 255; green, 0; blue, 0 }  ,draw opacity=1 ]   (133,78.97) -- (176,78.97) ;
\draw [shift={(178,78.97)}, rotate = 180] [color={rgb, 255:red, 255; green, 0; blue, 0 }  ,draw opacity=1 ][line width=0.75]    (10.93,-3.29) .. controls (6.95,-1.4) and (3.31,-0.3) .. (0,0) .. controls (3.31,0.3) and (6.95,1.4) .. (10.93,3.29)   ;
%Straight Lines [id:da7871923244680363]
\draw [color={rgb, 255:red, 255; green, 0; blue, 0 }  ,draw opacity=1 ]   (217,78.97) -- (239.54,138.02) ;
\draw [shift={(240.25,139.89)}, rotate = 249.11] [color={rgb, 255:red, 255; green, 0; blue, 0 }  ,draw opacity=1 ][line width=0.75]    (10.93,-3.29) .. controls (6.95,-1.4) and (3.31,-0.3) .. (0,0) .. controls (3.31,0.3) and (6.95,1.4) .. (10.93,3.29)   ;
%Straight Lines [id:da9126995185441131]
\draw    (389.5,42.86) -- (389.5,338.03) ;
%Shape: Ellipse [id:dp6768805467894146]
\draw  [color={rgb, 255:red, 0; green, 0; blue, 0 }  ,draw opacity=1 ] (423,77.99) .. controls (423,67.32) and (431.73,58.68) .. (442.5,58.68) .. controls (453.27,58.68) and (462,67.32) .. (462,77.99) .. controls (462,88.66) and (453.27,97.31) .. (442.5,97.31) .. controls (431.73,97.31) and (423,88.66) .. (423,77.99) -- cycle ;
%Shape: Ellipse [id:dp2809071485474479]
\draw  [color={rgb, 255:red, 255; green, 0; blue, 0 }  ,draw opacity=1 ] (423,285.99) .. controls (423,275.32) and (431.73,266.68) .. (442.5,266.68) .. controls (453.27,266.68) and (462,275.32) .. (462,285.99) .. controls (462,296.66) and (453.27,305.31) .. (442.5,305.31) .. controls (431.73,305.31) and (423,296.66) .. (423,285.99) -- cycle ;
%Shape: Ellipse [id:dp24980388801748288]
\draw   (394,153.01) .. controls (394,145.77) and (399.93,139.89) .. (407.25,139.89) .. controls (414.57,139.89) and (420.5,145.77) .. (420.5,153.01) .. controls (420.5,160.26) and (414.57,166.14) .. (407.25,166.14) .. controls (399.93,166.14) and (394,160.26) .. (394,153.01) -- cycle ;
%Shape: Ellipse [id:dp10909438427230511]
\draw   (394,215.01) .. controls (394,207.77) and (399.93,201.89) .. (407.25,201.89) .. controls (414.57,201.89) and (420.5,207.77) .. (420.5,215.01) .. controls (420.5,222.26) and (414.57,228.14) .. (407.25,228.14) .. controls (399.93,228.14) and (394,222.26) .. (394,215.01) -- cycle ;
%Straight Lines [id:da12155298850879026]
\draw    (469.5,43.86) -- (469.5,339.03) ;
%Shape: Ellipse [id:dp6473289236347626]
\draw  [color={rgb, 255:red, 0; green, 0; blue, 0 }  ,draw opacity=1 ] (507,78.99) .. controls (507,68.32) and (515.73,59.68) .. (526.5,59.68) .. controls (537.27,59.68) and (546,68.32) .. (546,78.99) .. controls (546,89.66) and (537.27,98.31) .. (526.5,98.31) .. controls (515.73,98.31) and (507,89.66) .. (507,78.99) -- cycle ;
%Shape: Ellipse [id:dp2954804230676249]
\draw  [color={rgb, 255:red, 250; green, 0; blue, 0 }  ,draw opacity=1 ] (506,285.99) .. controls (506,275.32) and (514.73,266.68) .. (525.5,266.68) .. controls (536.27,266.68) and (545,275.32) .. (545,285.99) .. controls (545,296.66) and (536.27,305.31) .. (525.5,305.31) .. controls (514.73,305.31) and (506,296.66) .. (506,285.99) -- cycle ;
%Shape: Ellipse [id:dp7637973407657246]
\draw   (473,153.01) .. controls (473,145.77) and (478.93,139.89) .. (486.25,139.89) .. controls (493.57,139.89) and (499.5,145.77) .. (499.5,153.01) .. controls (499.5,160.26) and (493.57,166.14) .. (486.25,166.14) .. controls (478.93,166.14) and (473,160.26) .. (473,153.01) -- cycle ;
%Shape: Ellipse [id:dp20173699209939167]
\draw   (472,215.01) .. controls (472,207.77) and (477.93,201.89) .. (485.25,201.89) .. controls (492.57,201.89) and (498.5,207.77) .. (498.5,215.01) .. controls (498.5,222.26) and (492.57,228.14) .. (485.25,228.14) .. controls (477.93,228.14) and (472,222.26) .. (472,215.01) -- cycle ;
%Straight Lines [id:da05574003790012938]
\draw [color={rgb, 255:red, 255; green, 6; blue, 37 }  ,draw opacity=1 ]   (545,285.99) -- (609.86,241.07) ;
\draw [shift={(611.5,239.93)}, rotate = 505.29] [color={rgb, 255:red, 255; green, 6; blue, 37 }  ,draw opacity=1 ][line width=0.75]    (10.93,-3.29) .. controls (6.95,-1.4) and (3.31,-0.3) .. (0,0) .. controls (3.31,0.3) and (6.95,1.4) .. (10.93,3.29)   ;
%Straight Lines [id:da08163021016655181]
\draw [color={rgb, 255:red, 255; green, 6; blue, 37 }  ,draw opacity=1 ]   (462,285.99) -- (504,285.99) ;
\draw [shift={(506,285.99)}, rotate = 180] [color={rgb, 255:red, 255; green, 6; blue, 37 }  ,draw opacity=1 ][line width=0.75]    (10.93,-3.29) .. controls (6.95,-1.4) and (3.31,-0.3) .. (0,0) .. controls (3.31,0.3) and (6.95,1.4) .. (10.93,3.29)   ;
%Straight Lines [id:da7786535525456149]
\draw [color={rgb, 255:red, 255; green, 6; blue, 37 }  ,draw opacity=1 ]   (385,285.99) -- (421,285.99) ;
\draw [shift={(423,285.99)}, rotate = 180] [color={rgb, 255:red, 255; green, 6; blue, 37 }  ,draw opacity=1 ][line width=0.75]    (10.93,-3.29) .. controls (6.95,-1.4) and (3.31,-0.3) .. (0,0) .. controls (3.31,0.3) and (6.95,1.4) .. (10.93,3.29)   ;
%Straight Lines [id:da9096675141760333]
\draw [color={rgb, 255:red, 255; green, 6; blue, 37 }  ,draw opacity=1 ]   (240.25,166.14) -- (281.73,264.83) ;
\draw [shift={(282.5,266.67)}, rotate = 247.20999999999998] [color={rgb, 255:red, 255; green, 6; blue, 37 }  ,draw opacity=1 ][line width=0.75]    (10.93,-3.29) .. controls (6.95,-1.4) and (3.31,-0.3) .. (0,0) .. controls (3.31,0.3) and (6.95,1.4) .. (10.93,3.29)   ;
%Straight Lines [id:da981816720767215]
\draw [color={rgb, 255:red, 255; green, 6; blue, 37 }  ,draw opacity=1 ]   (302,285.99) -- (344,285.99) ;
\draw [shift={(346,285.99)}, rotate = 180] [color={rgb, 255:red, 255; green, 6; blue, 37 }  ,draw opacity=1 ][line width=0.75]    (10.93,-3.29) .. controls (6.95,-1.4) and (3.31,-0.3) .. (0,0) .. controls (3.31,0.3) and (6.95,1.4) .. (10.93,3.29)   ;

% Text Node
\draw (110,70.47) node [anchor=north west][inner sep=0.75pt]   [align=left] {5};
% Text Node
\draw (193,70.47) node [anchor=north west][inner sep=0.75pt]   [align=left] {4};
% Text Node
\draw (277,69.48) node [anchor=north west][inner sep=0.75pt]   [align=left] {7};
% Text Node
\draw (361,67.49) node [anchor=north west][inner sep=0.75pt]   [align=left] {8};
% Text Node
\draw (25,139.81) node [anchor=north west][inner sep=0.75pt]   [align=left] {2};
% Text Node
\draw (606,140.8) node [anchor=north west][inner sep=0.75pt]   [align=left] {9};
% Text Node
\draw (606,212.13) node [anchor=north west][inner sep=0.75pt]   [align=left] {4};
% Text Node
\draw (361,277.49) node [anchor=north west][inner sep=0.75pt]   [align=left] {4};
% Text Node
\draw (277,277.49) node [anchor=north west][inner sep=0.75pt]   [align=left] {5};
% Text Node
\draw (192,278.48) node [anchor=north west][inner sep=0.75pt]   [align=left] {9};
% Text Node
\draw (108,278.48) node [anchor=north west][inner sep=0.75pt]   [align=left] {6};
% Text Node
\draw (25,212.12) node [anchor=north west][inner sep=0.75pt]   [align=left] {8};
% Text Node
\draw (149,144.76) node [anchor=north west][inner sep=0.75pt]   [align=left] {8};
% Text Node
\draw (150,206.17) node [anchor=north west][inner sep=0.75pt]   [align=left] {5};
% Text Node
\draw (235,144.76) node [anchor=north west][inner sep=0.75pt]   [align=left] {2};
% Text Node
\draw (235,207.16) node [anchor=north west][inner sep=0.75pt]   [align=left] {6};
% Text Node
\draw (318,144.76) node [anchor=north west][inner sep=0.75pt]   [align=left] {4};
% Text Node
\draw (319,206.17) node [anchor=north west][inner sep=0.75pt]   [align=left] {3};
% Text Node
\draw (150,25.9) node [anchor=north west][inner sep=0.75pt]   [align=left] {Linea de Producción 1};
% Text Node
\draw (126,344.1) node [anchor=north west][inner sep=0.75pt]   [align=left] {Linea de Producción 2};
% Text Node
\draw (437,70.49) node [anchor=north west][inner sep=0.75pt]   [align=left] {5};
% Text Node
\draw (438,278.49) node [anchor=north west][inner sep=0.75pt]   [align=left] {7};
% Text Node
\draw (402,144.76) node [anchor=north west][inner sep=0.75pt]   [align=left] {3};
% Text Node
\draw (402,206.76) node [anchor=north west][inner sep=0.75pt]   [align=left] {6};
% Text Node
\draw (481,144.76) node [anchor=north west][inner sep=0.75pt]   [align=left] {8};
% Text Node
\draw (480,206.76) node [anchor=north west][inner sep=0.75pt]   [align=left] {6};
% Text Node
\draw (521,72.49) node [anchor=north west][inner sep=0.75pt]   [align=left] {3};
% Text Node
\draw (520,278.49) node [anchor=north west][inner sep=0.75pt]   [align=left] {2};


\end{tikzpicture}

    Figura 53 - Ruta óptima del tercer ejemplo
\end{center}
En nuestro penúltimo ejemplo aumentamos todo de manera significante, el número de estaciones y el rango lo variamos más. Observamos que con un rango más variado ocurren un poco más de transiciones de línea a línea, pero sigue tendiendo un poco a mantenerse en una misma línea. La figura 54 nos muestra todo esto.
\begin{center}
    \includegraphics[width = 11cm]{LineasProduccion/10.png}\\
    Figura 54 - Cuarto ejemplo del algoritmo de las líneas de producción
\end{center}
Y por último para este ejemplo generamos su ruta óptima mostrada en la figura 55.
\begin{center}


\tikzset{every picture/.style={line width=0.75pt}} %set default line width to 0.75pt

\begin{tikzpicture}[x=0.75pt,y=0.75pt,yscale=-1,xscale=1]
%uncomment if require: \path (0,432); %set diagram left start at 0, and has height of 432

%Shape: Rectangle [id:dp3685700062544466]
\draw   (39.5,47.77) -- (621.5,47.77) -- (621.5,107.2) -- (39.5,107.2) -- cycle ;
%Shape: Ellipse [id:dp51290516302854]
\draw  [color={rgb, 255:red, 0; green, 0; blue, 0 }  ,draw opacity=1 ] (9,137.5) .. controls (9,131.14) and (13.81,125.99) .. (19.75,125.99) .. controls (25.69,125.99) and (30.5,131.14) .. (30.5,137.5) .. controls (30.5,143.85) and (25.69,149) .. (19.75,149) .. controls (13.81,149) and (9,143.85) .. (9,137.5) -- cycle ;
%Shape: Rectangle [id:dp43119014619460905]
\draw  [color={rgb, 255:red, 0; green, 0; blue, 0 }  ,draw opacity=1 ] (41.5,257.78) -- (618.5,257.78) -- (618.5,317.21) -- (41.5,317.21) -- cycle ;
%Straight Lines [id:da12155298850879026]
\draw [color={rgb, 255:red, 0; green, 0; blue, 0 }  ,draw opacity=1 ]   (77.5,37.86) -- (77.5,333.03) ;
%Straight Lines [id:da9096675141760333]
\draw [color={rgb, 255:red, 255; green, 6; blue, 37 }  ,draw opacity=1 ]   (19.75,238) -- (51.92,287.82) ;
\draw [shift={(53,289.5)}, rotate = 237.15] [color={rgb, 255:red, 255; green, 6; blue, 37 }  ,draw opacity=1 ][line width=0.75]    (10.93,-3.29) .. controls (6.95,-1.4) and (3.31,-0.3) .. (0,0) .. controls (3.31,0.3) and (6.95,1.4) .. (10.93,3.29)   ;
%Shape: Ellipse [id:dp06807039831730455]
\draw  [color={rgb, 255:red, 0; green, 0; blue, 0 }  ,draw opacity=1 ] (53,74.5) .. controls (53,68.14) and (57.81,62.99) .. (63.75,62.99) .. controls (69.69,62.99) and (74.5,68.14) .. (74.5,74.5) .. controls (74.5,80.85) and (69.69,86) .. (63.75,86) .. controls (57.81,86) and (53,80.85) .. (53,74.5) -- cycle ;
%Shape: Ellipse [id:dp42935919875661943]
\draw  [color={rgb, 255:red, 0; green, 0; blue, 0 }  ,draw opacity=1 ] (90,75.5) .. controls (90,69.14) and (94.81,63.99) .. (100.75,63.99) .. controls (106.69,63.99) and (111.5,69.14) .. (111.5,75.5) .. controls (111.5,81.85) and (106.69,87) .. (100.75,87) .. controls (94.81,87) and (90,81.85) .. (90,75.5) -- cycle ;
%Shape: Ellipse [id:dp1324697415650704]
\draw  [color={rgb, 255:red, 0; green, 0; blue, 0 }  ,draw opacity=1 ] (127,74.5) .. controls (127,68.14) and (131.81,62.99) .. (137.75,62.99) .. controls (143.69,62.99) and (148.5,68.14) .. (148.5,74.5) .. controls (148.5,80.85) and (143.69,86) .. (137.75,86) .. controls (131.81,86) and (127,80.85) .. (127,74.5) -- cycle ;
%Shape: Ellipse [id:dp1689554548890786]
\draw  [color={rgb, 255:red, 0; green, 0; blue, 0 }  ,draw opacity=1 ] (166,75.5) .. controls (166,69.14) and (170.81,63.99) .. (176.75,63.99) .. controls (182.69,63.99) and (187.5,69.14) .. (187.5,75.5) .. controls (187.5,81.85) and (182.69,87) .. (176.75,87) .. controls (170.81,87) and (166,81.85) .. (166,75.5) -- cycle ;
%Shape: Ellipse [id:dp7933707169883619]
\draw  [color={rgb, 255:red, 255; green, 0; blue, 0 }  ,draw opacity=1 ] (205,75.5) .. controls (205,69.14) and (209.81,63.99) .. (215.75,63.99) .. controls (221.69,63.99) and (226.5,69.14) .. (226.5,75.5) .. controls (226.5,81.85) and (221.69,87) .. (215.75,87) .. controls (209.81,87) and (205,81.85) .. (205,75.5) -- cycle ;
%Shape: Ellipse [id:dp5149000200742411]
\draw  [color={rgb, 255:red, 0; green, 0; blue, 0 }  ,draw opacity=1 ] (242,75.5) .. controls (242,69.14) and (246.81,63.99) .. (252.75,63.99) .. controls (258.69,63.99) and (263.5,69.14) .. (263.5,75.5) .. controls (263.5,81.85) and (258.69,87) .. (252.75,87) .. controls (246.81,87) and (242,81.85) .. (242,75.5) -- cycle ;
%Shape: Ellipse [id:dp20035823021434207]
\draw  [color={rgb, 255:red, 0; green, 0; blue, 0 }  ,draw opacity=1 ] (280,75.5) .. controls (280,69.14) and (284.81,63.99) .. (290.75,63.99) .. controls (296.69,63.99) and (301.5,69.14) .. (301.5,75.5) .. controls (301.5,81.85) and (296.69,87) .. (290.75,87) .. controls (284.81,87) and (280,81.85) .. (280,75.5) -- cycle ;
%Shape: Ellipse [id:dp0814774775452547]
\draw  [color={rgb, 255:red, 0; green, 0; blue, 0 }  ,draw opacity=1 ] (317,75.5) .. controls (317,69.14) and (321.81,63.99) .. (327.75,63.99) .. controls (333.69,63.99) and (338.5,69.14) .. (338.5,75.5) .. controls (338.5,81.85) and (333.69,87) .. (327.75,87) .. controls (321.81,87) and (317,81.85) .. (317,75.5) -- cycle ;
%Shape: Ellipse [id:dp13446919244322708]
\draw  [color={rgb, 255:red, 0; green, 0; blue, 0 }  ,draw opacity=1 ] (353,75.5) .. controls (353,69.14) and (357.81,63.99) .. (363.75,63.99) .. controls (369.69,63.99) and (374.5,69.14) .. (374.5,75.5) .. controls (374.5,81.85) and (369.69,87) .. (363.75,87) .. controls (357.81,87) and (353,81.85) .. (353,75.5) -- cycle ;
%Shape: Ellipse [id:dp9536584042655418]
\draw  [color={rgb, 255:red, 0; green, 0; blue, 0 }  ,draw opacity=1 ] (394,75.5) .. controls (394,69.14) and (398.81,63.99) .. (404.75,63.99) .. controls (410.69,63.99) and (415.5,69.14) .. (415.5,75.5) .. controls (415.5,81.85) and (410.69,87) .. (404.75,87) .. controls (398.81,87) and (394,81.85) .. (394,75.5) -- cycle ;
%Shape: Ellipse [id:dp2916139054636573]
\draw  [color={rgb, 255:red, 0; green, 0; blue, 0 }  ,draw opacity=1 ] (433,75.5) .. controls (433,69.14) and (437.81,63.99) .. (443.75,63.99) .. controls (449.69,63.99) and (454.5,69.14) .. (454.5,75.5) .. controls (454.5,81.85) and (449.69,87) .. (443.75,87) .. controls (437.81,87) and (433,81.85) .. (433,75.5) -- cycle ;
%Shape: Ellipse [id:dp9264638956869218]
\draw  [color={rgb, 255:red, 0; green, 0; blue, 0 }  ,draw opacity=1 ] (472,74.5) .. controls (472,68.14) and (476.81,62.99) .. (482.75,62.99) .. controls (488.69,62.99) and (493.5,68.14) .. (493.5,74.5) .. controls (493.5,80.85) and (488.69,86) .. (482.75,86) .. controls (476.81,86) and (472,80.85) .. (472,74.5) -- cycle ;
%Shape: Ellipse [id:dp7777370109244943]
\draw  [color={rgb, 255:red, 255; green, 0; blue, 0 }  ,draw opacity=1 ] (512,74.5) .. controls (512,68.14) and (516.81,62.99) .. (522.75,62.99) .. controls (528.69,62.99) and (533.5,68.14) .. (533.5,74.5) .. controls (533.5,80.85) and (528.69,86) .. (522.75,86) .. controls (516.81,86) and (512,80.85) .. (512,74.5) -- cycle ;
%Shape: Ellipse [id:dp04457771107220165]
\draw  [color={rgb, 255:red, 255; green, 0; blue, 0 }  ,draw opacity=1 ] (551,74.5) .. controls (551,68.14) and (555.81,62.99) .. (561.75,62.99) .. controls (567.69,62.99) and (572.5,68.14) .. (572.5,74.5) .. controls (572.5,80.85) and (567.69,86) .. (561.75,86) .. controls (555.81,86) and (551,80.85) .. (551,74.5) -- cycle ;
%Shape: Ellipse [id:dp7998590240157639]
\draw  [color={rgb, 255:red, 255; green, 0; blue, 0 }  ,draw opacity=1 ] (587,74.5) .. controls (587,68.14) and (591.81,62.99) .. (597.75,62.99) .. controls (603.69,62.99) and (608.5,68.14) .. (608.5,74.5) .. controls (608.5,80.85) and (603.69,86) .. (597.75,86) .. controls (591.81,86) and (587,80.85) .. (587,74.5) -- cycle ;
%Shape: Ellipse [id:dp07214101457810629]
\draw  [color={rgb, 255:red, 255; green, 0; blue, 0 }  ,draw opacity=1 ] (53,289.5) .. controls (53,283.14) and (57.81,277.99) .. (63.75,277.99) .. controls (69.69,277.99) and (74.5,283.14) .. (74.5,289.5) .. controls (74.5,295.85) and (69.69,301) .. (63.75,301) .. controls (57.81,301) and (53,295.85) .. (53,289.5) -- cycle ;
%Shape: Ellipse [id:dp6422913900875762]
\draw  [color={rgb, 255:red, 255; green, 0; blue, 0 }  ,draw opacity=1 ] (90,290.5) .. controls (90,284.14) and (94.81,278.99) .. (100.75,278.99) .. controls (106.69,278.99) and (111.5,284.14) .. (111.5,290.5) .. controls (111.5,296.85) and (106.69,302) .. (100.75,302) .. controls (94.81,302) and (90,296.85) .. (90,290.5) -- cycle ;
%Shape: Ellipse [id:dp8188912159391883]
\draw  [color={rgb, 255:red, 255; green, 0; blue, 0 }  ,draw opacity=1 ] (127,289.5) .. controls (127,283.14) and (131.81,277.99) .. (137.75,277.99) .. controls (143.69,277.99) and (148.5,283.14) .. (148.5,289.5) .. controls (148.5,295.85) and (143.69,301) .. (137.75,301) .. controls (131.81,301) and (127,295.85) .. (127,289.5) -- cycle ;
%Shape: Ellipse [id:dp5363146050444563]
\draw  [color={rgb, 255:red, 255; green, 0; blue, 0 }  ,draw opacity=1 ] (166,290.5) .. controls (166,284.14) and (170.81,278.99) .. (176.75,278.99) .. controls (182.69,278.99) and (187.5,284.14) .. (187.5,290.5) .. controls (187.5,296.85) and (182.69,302) .. (176.75,302) .. controls (170.81,302) and (166,296.85) .. (166,290.5) -- cycle ;
%Shape: Ellipse [id:dp8353752555307812]
\draw  [color={rgb, 255:red, 0; green, 0; blue, 0 }  ,draw opacity=1 ] (205,290.5) .. controls (205,284.14) and (209.81,278.99) .. (215.75,278.99) .. controls (221.69,278.99) and (226.5,284.14) .. (226.5,290.5) .. controls (226.5,296.85) and (221.69,302) .. (215.75,302) .. controls (209.81,302) and (205,296.85) .. (205,290.5) -- cycle ;
%Shape: Ellipse [id:dp8192352713841125]
\draw  [color={rgb, 255:red, 255; green, 0; blue, 0 }  ,draw opacity=1 ] (242,290.5) .. controls (242,284.14) and (246.81,278.99) .. (252.75,278.99) .. controls (258.69,278.99) and (263.5,284.14) .. (263.5,290.5) .. controls (263.5,296.85) and (258.69,302) .. (252.75,302) .. controls (246.81,302) and (242,296.85) .. (242,290.5) -- cycle ;
%Shape: Ellipse [id:dp15240194060892698]
\draw  [color={rgb, 255:red, 255; green, 0; blue, 0 }  ,draw opacity=1 ] (280,290.5) .. controls (280,284.14) and (284.81,278.99) .. (290.75,278.99) .. controls (296.69,278.99) and (301.5,284.14) .. (301.5,290.5) .. controls (301.5,296.85) and (296.69,302) .. (290.75,302) .. controls (284.81,302) and (280,296.85) .. (280,290.5) -- cycle ;
%Shape: Ellipse [id:dp07750903322708602]
\draw  [color={rgb, 255:red, 255; green, 0; blue, 0 }  ,draw opacity=1 ] (317,290.5) .. controls (317,284.14) and (321.81,278.99) .. (327.75,278.99) .. controls (333.69,278.99) and (338.5,284.14) .. (338.5,290.5) .. controls (338.5,296.85) and (333.69,302) .. (327.75,302) .. controls (321.81,302) and (317,296.85) .. (317,290.5) -- cycle ;
%Shape: Ellipse [id:dp2598014428916928]
\draw  [color={rgb, 255:red, 255; green, 0; blue, 0 }  ,draw opacity=1 ] (353,290.5) .. controls (353,284.14) and (357.81,278.99) .. (363.75,278.99) .. controls (369.69,278.99) and (374.5,284.14) .. (374.5,290.5) .. controls (374.5,296.85) and (369.69,302) .. (363.75,302) .. controls (357.81,302) and (353,296.85) .. (353,290.5) -- cycle ;
%Shape: Ellipse [id:dp9377462422816254]
\draw  [color={rgb, 255:red, 255; green, 0; blue, 0 }  ,draw opacity=1 ] (394,290.5) .. controls (394,284.14) and (398.81,278.99) .. (404.75,278.99) .. controls (410.69,278.99) and (415.5,284.14) .. (415.5,290.5) .. controls (415.5,296.85) and (410.69,302) .. (404.75,302) .. controls (398.81,302) and (394,296.85) .. (394,290.5) -- cycle ;
%Shape: Ellipse [id:dp39500914120699404]
\draw  [color={rgb, 255:red, 255; green, 0; blue, 0 }  ,draw opacity=1 ] (433,290.5) .. controls (433,284.14) and (437.81,278.99) .. (443.75,278.99) .. controls (449.69,278.99) and (454.5,284.14) .. (454.5,290.5) .. controls (454.5,296.85) and (449.69,302) .. (443.75,302) .. controls (437.81,302) and (433,296.85) .. (433,290.5) -- cycle ;
%Shape: Ellipse [id:dp8414742477181929]
\draw  [color={rgb, 255:red, 255; green, 0; blue, 0 }  ,draw opacity=1 ] (472,289.5) .. controls (472,283.14) and (476.81,277.99) .. (482.75,277.99) .. controls (488.69,277.99) and (493.5,283.14) .. (493.5,289.5) .. controls (493.5,295.85) and (488.69,301) .. (482.75,301) .. controls (476.81,301) and (472,295.85) .. (472,289.5) -- cycle ;
%Shape: Ellipse [id:dp3082009921160864]
\draw  [color={rgb, 255:red, 0; green, 0; blue, 0 }  ,draw opacity=1 ] (512,289.5) .. controls (512,283.14) and (516.81,277.99) .. (522.75,277.99) .. controls (528.69,277.99) and (533.5,283.14) .. (533.5,289.5) .. controls (533.5,295.85) and (528.69,301) .. (522.75,301) .. controls (516.81,301) and (512,295.85) .. (512,289.5) -- cycle ;
%Shape: Ellipse [id:dp1467976253922103]
\draw  [color={rgb, 255:red, 0; green, 0; blue, 0 }  ,draw opacity=1 ] (551,289.5) .. controls (551,283.14) and (555.81,277.99) .. (561.75,277.99) .. controls (567.69,277.99) and (572.5,283.14) .. (572.5,289.5) .. controls (572.5,295.85) and (567.69,301) .. (561.75,301) .. controls (555.81,301) and (551,295.85) .. (551,289.5) -- cycle ;
%Shape: Ellipse [id:dp6433712019371127]
\draw  [color={rgb, 255:red, 0; green, 0; blue, 0 }  ,draw opacity=1 ] (587,289.5) .. controls (587,283.14) and (591.81,277.99) .. (597.75,277.99) .. controls (603.69,277.99) and (608.5,283.14) .. (608.5,289.5) .. controls (608.5,295.85) and (603.69,301) .. (597.75,301) .. controls (591.81,301) and (587,295.85) .. (587,289.5) -- cycle ;
%Straight Lines [id:da06605467638753848]
\draw [color={rgb, 255:red, 0; green, 0; blue, 0 }  ,draw opacity=1 ]   (114.5,37.86) -- (114.5,333.03) ;
%Straight Lines [id:da24813536029182304]
\draw [color={rgb, 255:red, 0; green, 0; blue, 0 }  ,draw opacity=1 ]   (151.5,37.86) -- (151.5,333.03) ;
%Straight Lines [id:da2234585415413517]
\draw    (191.5,37.86) -- (191.5,333.03) ;
%Straight Lines [id:da0900783511665173]
\draw    (229.5,38.86) -- (229.5,334.03) ;
%Straight Lines [id:da9562023238190085]
\draw [color={rgb, 255:red, 0; green, 0; blue, 0 }  ,draw opacity=1 ]   (267.5,38.86) -- (267.5,334.03) ;
%Straight Lines [id:da41266206814469863]
\draw [color={rgb, 255:red, 0; green, 0; blue, 0 }  ,draw opacity=1 ]   (305.5,39.86) -- (305.5,335.03) ;
%Straight Lines [id:da19467659136269466]
\draw [color={rgb, 255:red, 0; green, 0; blue, 0 }  ,draw opacity=1 ]   (341.5,39.86) -- (341.5,335.03) ;
%Straight Lines [id:da07374701034023179]
\draw [color={rgb, 255:red, 0; green, 0; blue, 0 }  ,draw opacity=1 ]   (378.5,39.86) -- (378.5,335.03) ;
%Straight Lines [id:da7866926227139093]
\draw [color={rgb, 255:red, 0; green, 0; blue, 0 }  ,draw opacity=1 ]   (418.5,39.86) -- (418.5,335.03) ;
%Straight Lines [id:da34342920694004486]
\draw [color={rgb, 255:red, 0; green, 0; blue, 0 }  ,draw opacity=1 ]   (457.5,40.86) -- (457.5,336.03) ;
%Straight Lines [id:da7195305797795226]
\draw    (496.5,39.86) -- (496.5,335.03) ;
%Straight Lines [id:da337394814000757]
\draw [color={rgb, 255:red, 0; green, 0; blue, 0 }  ,draw opacity=1 ]   (536.5,40.86) -- (536.5,336.03) ;
%Straight Lines [id:da2665455381345483]
\draw [color={rgb, 255:red, 0; green, 0; blue, 0 }  ,draw opacity=1 ]   (575.5,41.86) -- (575.5,337.03) ;
%Shape: Ellipse [id:dp24253779349833549]
\draw  [color={rgb, 255:red, 255; green, 0; blue, 0 }  ,draw opacity=1 ] (9,226.5) .. controls (9,220.14) and (13.81,214.99) .. (19.75,214.99) .. controls (25.69,214.99) and (30.5,220.14) .. (30.5,226.5) .. controls (30.5,232.85) and (25.69,238) .. (19.75,238) .. controls (13.81,238) and (9,232.85) .. (9,226.5) -- cycle ;
%Shape: Ellipse [id:dp6217032517818393]
\draw  [color={rgb, 255:red, 255; green, 0; blue, 0 }  ,draw opacity=1 ] (625,137.5) .. controls (625,131.14) and (629.81,125.99) .. (635.75,125.99) .. controls (641.69,125.99) and (646.5,131.14) .. (646.5,137.5) .. controls (646.5,143.85) and (641.69,149) .. (635.75,149) .. controls (629.81,149) and (625,143.85) .. (625,137.5) -- cycle ;
%Shape: Ellipse [id:dp9388393357696654]
\draw  [color={rgb, 255:red, 0; green, 0; blue, 0 }  ,draw opacity=1 ] (625,226.5) .. controls (625,220.14) and (629.81,214.99) .. (635.75,214.99) .. controls (641.69,214.99) and (646.5,220.14) .. (646.5,226.5) .. controls (646.5,232.85) and (641.69,238) .. (635.75,238) .. controls (629.81,238) and (625,232.85) .. (625,226.5) -- cycle ;
%Shape: Ellipse [id:dp4286706815908876]
\draw  [color={rgb, 255:red, 0; green, 0; blue, 0 }  ,draw opacity=1 ] (80,157.5) .. controls (80,152.25) and (83.92,147.99) .. (88.75,147.99) .. controls (93.58,147.99) and (97.5,152.25) .. (97.5,157.5) .. controls (97.5,162.75) and (93.58,167) .. (88.75,167) .. controls (83.92,167) and (80,162.75) .. (80,157.5) -- cycle ;
%Shape: Ellipse [id:dp6472724834712311]
\draw  [color={rgb, 255:red, 0; green, 0; blue, 0 }  ,draw opacity=1 ] (80,213.5) .. controls (80,208.25) and (83.92,203.99) .. (88.75,203.99) .. controls (93.58,203.99) and (97.5,208.25) .. (97.5,213.5) .. controls (97.5,218.75) and (93.58,223) .. (88.75,223) .. controls (83.92,223) and (80,218.75) .. (80,213.5) -- cycle ;
%Shape: Ellipse [id:dp9562542947961445]
\draw  [color={rgb, 255:red, 0; green, 0; blue, 0 }  ,draw opacity=1 ] (117,158.5) .. controls (117,153.25) and (120.92,148.99) .. (125.75,148.99) .. controls (130.58,148.99) and (134.5,153.25) .. (134.5,158.5) .. controls (134.5,163.75) and (130.58,168) .. (125.75,168) .. controls (120.92,168) and (117,163.75) .. (117,158.5) -- cycle ;
%Shape: Ellipse [id:dp3606766136686099]
\draw  [color={rgb, 255:red, 0; green, 0; blue, 0 }  ,draw opacity=1 ] (117,214.5) .. controls (117,209.25) and (120.92,204.99) .. (125.75,204.99) .. controls (130.58,204.99) and (134.5,209.25) .. (134.5,214.5) .. controls (134.5,219.75) and (130.58,224) .. (125.75,224) .. controls (120.92,224) and (117,219.75) .. (117,214.5) -- cycle ;
%Shape: Ellipse [id:dp8755768196635936]
\draw  [color={rgb, 255:red, 0; green, 0; blue, 0 }  ,draw opacity=1 ] (155,159.5) .. controls (155,154.25) and (158.92,149.99) .. (163.75,149.99) .. controls (168.58,149.99) and (172.5,154.25) .. (172.5,159.5) .. controls (172.5,164.75) and (168.58,169) .. (163.75,169) .. controls (158.92,169) and (155,164.75) .. (155,159.5) -- cycle ;
%Shape: Ellipse [id:dp7198580175278853]
\draw  [color={rgb, 255:red, 0; green, 0; blue, 0 }  ,draw opacity=1 ] (155,215.5) .. controls (155,210.25) and (158.92,205.99) .. (163.75,205.99) .. controls (168.58,205.99) and (172.5,210.25) .. (172.5,215.5) .. controls (172.5,220.75) and (168.58,225) .. (163.75,225) .. controls (158.92,225) and (155,220.75) .. (155,215.5) -- cycle ;
%Shape: Ellipse [id:dp7856444475808753]
\draw  [color={rgb, 255:red, 255; green, 0; blue, 0 }  ,draw opacity=1 ] (194,158.5) .. controls (194,153.25) and (197.92,148.99) .. (202.75,148.99) .. controls (207.58,148.99) and (211.5,153.25) .. (211.5,158.5) .. controls (211.5,163.75) and (207.58,168) .. (202.75,168) .. controls (197.92,168) and (194,163.75) .. (194,158.5) -- cycle ;
%Shape: Ellipse [id:dp7323175766073924]
\draw  [color={rgb, 255:red, 255; green, 0; blue, 0 }  ,draw opacity=1 ] (194,214.5) .. controls (194,209.25) and (197.92,204.99) .. (202.75,204.99) .. controls (207.58,204.99) and (211.5,209.25) .. (211.5,214.5) .. controls (211.5,219.75) and (207.58,224) .. (202.75,224) .. controls (197.92,224) and (194,219.75) .. (194,214.5) -- cycle ;
%Shape: Ellipse [id:dp022516483358351325]
\draw  [color={rgb, 255:red, 255; green, 0; blue, 0 }  ,draw opacity=1 ] (232,158.5) .. controls (232,153.25) and (235.92,148.99) .. (240.75,148.99) .. controls (245.58,148.99) and (249.5,153.25) .. (249.5,158.5) .. controls (249.5,163.75) and (245.58,168) .. (240.75,168) .. controls (235.92,168) and (232,163.75) .. (232,158.5) -- cycle ;
%Shape: Ellipse [id:dp9982357563290403]
\draw  [color={rgb, 255:red, 255; green, 0; blue, 0 }  ,draw opacity=1 ] (232,214.5) .. controls (232,209.25) and (235.92,204.99) .. (240.75,204.99) .. controls (245.58,204.99) and (249.5,209.25) .. (249.5,214.5) .. controls (249.5,219.75) and (245.58,224) .. (240.75,224) .. controls (235.92,224) and (232,219.75) .. (232,214.5) -- cycle ;
%Shape: Ellipse [id:dp6619851663686465]
\draw  [color={rgb, 255:red, 0; green, 0; blue, 0 }  ,draw opacity=1 ] (270,158.5) .. controls (270,153.25) and (273.92,148.99) .. (278.75,148.99) .. controls (283.58,148.99) and (287.5,153.25) .. (287.5,158.5) .. controls (287.5,163.75) and (283.58,168) .. (278.75,168) .. controls (273.92,168) and (270,163.75) .. (270,158.5) -- cycle ;
%Shape: Ellipse [id:dp5866878999050675]
\draw  [color={rgb, 255:red, 0; green, 0; blue, 0 }  ,draw opacity=1 ] (270,214.5) .. controls (270,209.25) and (273.92,204.99) .. (278.75,204.99) .. controls (283.58,204.99) and (287.5,209.25) .. (287.5,214.5) .. controls (287.5,219.75) and (283.58,224) .. (278.75,224) .. controls (273.92,224) and (270,219.75) .. (270,214.5) -- cycle ;
%Shape: Ellipse [id:dp8687190310724369]
\draw  [color={rgb, 255:red, 0; green, 0; blue, 0 }  ,draw opacity=1 ] (308,158.5) .. controls (308,153.25) and (311.92,148.99) .. (316.75,148.99) .. controls (321.58,148.99) and (325.5,153.25) .. (325.5,158.5) .. controls (325.5,163.75) and (321.58,168) .. (316.75,168) .. controls (311.92,168) and (308,163.75) .. (308,158.5) -- cycle ;
%Shape: Ellipse [id:dp6835782848114966]
\draw  [color={rgb, 255:red, 0; green, 0; blue, 0 }  ,draw opacity=1 ] (308,214.5) .. controls (308,209.25) and (311.92,204.99) .. (316.75,204.99) .. controls (321.58,204.99) and (325.5,209.25) .. (325.5,214.5) .. controls (325.5,219.75) and (321.58,224) .. (316.75,224) .. controls (311.92,224) and (308,219.75) .. (308,214.5) -- cycle ;
%Shape: Ellipse [id:dp09147115152995555]
\draw  [color={rgb, 255:red, 0; green, 0; blue, 0 }  ,draw opacity=1 ] (344,158.5) .. controls (344,153.25) and (347.92,148.99) .. (352.75,148.99) .. controls (357.58,148.99) and (361.5,153.25) .. (361.5,158.5) .. controls (361.5,163.75) and (357.58,168) .. (352.75,168) .. controls (347.92,168) and (344,163.75) .. (344,158.5) -- cycle ;
%Shape: Ellipse [id:dp12774696676046737]
\draw  [color={rgb, 255:red, 0; green, 0; blue, 0 }  ,draw opacity=1 ] (344,214.5) .. controls (344,209.25) and (347.92,204.99) .. (352.75,204.99) .. controls (357.58,204.99) and (361.5,209.25) .. (361.5,214.5) .. controls (361.5,219.75) and (357.58,224) .. (352.75,224) .. controls (347.92,224) and (344,219.75) .. (344,214.5) -- cycle ;
%Shape: Ellipse [id:dp7822209229895358]
\draw  [color={rgb, 255:red, 0; green, 0; blue, 0 }  ,draw opacity=1 ] (381,158.5) .. controls (381,153.25) and (384.92,148.99) .. (389.75,148.99) .. controls (394.58,148.99) and (398.5,153.25) .. (398.5,158.5) .. controls (398.5,163.75) and (394.58,168) .. (389.75,168) .. controls (384.92,168) and (381,163.75) .. (381,158.5) -- cycle ;
%Shape: Ellipse [id:dp08115095790577942]
\draw  [color={rgb, 255:red, 0; green, 0; blue, 0 }  ,draw opacity=1 ] (381,214.5) .. controls (381,209.25) and (384.92,204.99) .. (389.75,204.99) .. controls (394.58,204.99) and (398.5,209.25) .. (398.5,214.5) .. controls (398.5,219.75) and (394.58,224) .. (389.75,224) .. controls (384.92,224) and (381,219.75) .. (381,214.5) -- cycle ;
%Shape: Ellipse [id:dp5652975406491305]
\draw  [color={rgb, 255:red, 0; green, 0; blue, 0 }  ,draw opacity=1 ] (421,159.5) .. controls (421,154.25) and (424.92,149.99) .. (429.75,149.99) .. controls (434.58,149.99) and (438.5,154.25) .. (438.5,159.5) .. controls (438.5,164.75) and (434.58,169) .. (429.75,169) .. controls (424.92,169) and (421,164.75) .. (421,159.5) -- cycle ;
%Shape: Ellipse [id:dp26194252392081063]
\draw  [color={rgb, 255:red, 0; green, 0; blue, 0 }  ,draw opacity=1 ] (421,215.5) .. controls (421,210.25) and (424.92,205.99) .. (429.75,205.99) .. controls (434.58,205.99) and (438.5,210.25) .. (438.5,215.5) .. controls (438.5,220.75) and (434.58,225) .. (429.75,225) .. controls (424.92,225) and (421,220.75) .. (421,215.5) -- cycle ;
%Shape: Ellipse [id:dp7917683818654446]
\draw  [color={rgb, 255:red, 0; green, 0; blue, 0 }  ,draw opacity=1 ] (460,158.5) .. controls (460,153.25) and (463.92,148.99) .. (468.75,148.99) .. controls (473.58,148.99) and (477.5,153.25) .. (477.5,158.5) .. controls (477.5,163.75) and (473.58,168) .. (468.75,168) .. controls (463.92,168) and (460,163.75) .. (460,158.5) -- cycle ;
%Shape: Ellipse [id:dp330183781472168]
\draw  [color={rgb, 255:red, 0; green, 0; blue, 0 }  ,draw opacity=1 ] (460,214.5) .. controls (460,209.25) and (463.92,204.99) .. (468.75,204.99) .. controls (473.58,204.99) and (477.5,209.25) .. (477.5,214.5) .. controls (477.5,219.75) and (473.58,224) .. (468.75,224) .. controls (463.92,224) and (460,219.75) .. (460,214.5) -- cycle ;
%Shape: Ellipse [id:dp18296148469504403]
\draw  [color={rgb, 255:red, 255; green, 0; blue, 0 }  ,draw opacity=1 ] (499,158.5) .. controls (499,153.25) and (502.92,148.99) .. (507.75,148.99) .. controls (512.58,148.99) and (516.5,153.25) .. (516.5,158.5) .. controls (516.5,163.75) and (512.58,168) .. (507.75,168) .. controls (502.92,168) and (499,163.75) .. (499,158.5) -- cycle ;
%Shape: Ellipse [id:dp45774086752288934]
\draw  [color={rgb, 255:red, 255; green, 0; blue, 0 }  ,draw opacity=1 ] (499,214.5) .. controls (499,209.25) and (502.92,204.99) .. (507.75,204.99) .. controls (512.58,204.99) and (516.5,209.25) .. (516.5,214.5) .. controls (516.5,219.75) and (512.58,224) .. (507.75,224) .. controls (502.92,224) and (499,219.75) .. (499,214.5) -- cycle ;
%Shape: Ellipse [id:dp11980696846380456]
\draw  [color={rgb, 255:red, 0; green, 0; blue, 0 }  ,draw opacity=1 ] (540,158.5) .. controls (540,153.25) and (543.92,148.99) .. (548.75,148.99) .. controls (553.58,148.99) and (557.5,153.25) .. (557.5,158.5) .. controls (557.5,163.75) and (553.58,168) .. (548.75,168) .. controls (543.92,168) and (540,163.75) .. (540,158.5) -- cycle ;
%Shape: Ellipse [id:dp9352758847752061]
\draw  [color={rgb, 255:red, 0; green, 0; blue, 0 }  ,draw opacity=1 ] (540,214.5) .. controls (540,209.25) and (543.92,204.99) .. (548.75,204.99) .. controls (553.58,204.99) and (557.5,209.25) .. (557.5,214.5) .. controls (557.5,219.75) and (553.58,224) .. (548.75,224) .. controls (543.92,224) and (540,219.75) .. (540,214.5) -- cycle ;
%Shape: Ellipse [id:dp6535710414334979]
\draw  [color={rgb, 255:red, 0; green, 0; blue, 0 }  ,draw opacity=1 ] (578,158.5) .. controls (578,153.25) and (581.92,148.99) .. (586.75,148.99) .. controls (591.58,148.99) and (595.5,153.25) .. (595.5,158.5) .. controls (595.5,163.75) and (591.58,168) .. (586.75,168) .. controls (581.92,168) and (578,163.75) .. (578,158.5) -- cycle ;
%Shape: Ellipse [id:dp7038181910417307]
\draw  [color={rgb, 255:red, 0; green, 0; blue, 0 }  ,draw opacity=1 ] (578,214.5) .. controls (578,209.25) and (581.92,204.99) .. (586.75,204.99) .. controls (591.58,204.99) and (595.5,209.25) .. (595.5,214.5) .. controls (595.5,219.75) and (591.58,224) .. (586.75,224) .. controls (581.92,224) and (578,219.75) .. (578,214.5) -- cycle ;
%Straight Lines [id:da5553900534796801]
\draw [color={rgb, 255:red, 255; green, 6; blue, 37 }  ,draw opacity=1 ]   (74.5,289.5) -- (88,290.37) ;
\draw [shift={(90,290.5)}, rotate = 183.69] [color={rgb, 255:red, 255; green, 6; blue, 37 }  ,draw opacity=1 ][line width=0.75]    (10.93,-3.29) .. controls (6.95,-1.4) and (3.31,-0.3) .. (0,0) .. controls (3.31,0.3) and (6.95,1.4) .. (10.93,3.29)   ;
%Straight Lines [id:da013078875545969204]
\draw [color={rgb, 255:red, 255; green, 6; blue, 37 }  ,draw opacity=1 ]   (111.5,290.5) -- (125,289.63) ;
\draw [shift={(127,289.5)}, rotate = 536.31] [color={rgb, 255:red, 255; green, 6; blue, 37 }  ,draw opacity=1 ][line width=0.75]    (10.93,-3.29) .. controls (6.95,-1.4) and (3.31,-0.3) .. (0,0) .. controls (3.31,0.3) and (6.95,1.4) .. (10.93,3.29)   ;
%Straight Lines [id:da35433459347190066]
\draw [color={rgb, 255:red, 255; green, 6; blue, 37 }  ,draw opacity=1 ]   (148.5,289.5) -- (164,290.38) ;
\draw [shift={(166,290.5)}, rotate = 183.27] [color={rgb, 255:red, 255; green, 6; blue, 37 }  ,draw opacity=1 ][line width=0.75]    (10.93,-3.29) .. controls (6.95,-1.4) and (3.31,-0.3) .. (0,0) .. controls (3.31,0.3) and (6.95,1.4) .. (10.93,3.29)   ;
%Straight Lines [id:da09037610851492439]
\draw [color={rgb, 255:red, 255; green, 6; blue, 37 }  ,draw opacity=1 ]   (176.75,278.99) -- (201.9,225.81) ;
\draw [shift={(202.75,224)}, rotate = 475.3] [color={rgb, 255:red, 255; green, 6; blue, 37 }  ,draw opacity=1 ][line width=0.75]    (10.93,-3.29) .. controls (6.95,-1.4) and (3.31,-0.3) .. (0,0) .. controls (3.31,0.3) and (6.95,1.4) .. (10.93,3.29)   ;
%Straight Lines [id:da5606560797647313]
\draw [color={rgb, 255:red, 255; green, 6; blue, 37 }  ,draw opacity=1 ]   (211.5,214.5) -- (215.68,89) ;
\draw [shift={(215.75,87)}, rotate = 451.91] [color={rgb, 255:red, 255; green, 6; blue, 37 }  ,draw opacity=1 ][line width=0.75]    (10.93,-3.29) .. controls (6.95,-1.4) and (3.31,-0.3) .. (0,0) .. controls (3.31,0.3) and (6.95,1.4) .. (10.93,3.29)   ;
%Straight Lines [id:da49658813346880804]
\draw [color={rgb, 255:red, 255; green, 6; blue, 37 }  ,draw opacity=1 ]   (226.5,75.5) -- (240.37,147.03) ;
\draw [shift={(240.75,148.99)}, rotate = 259.03] [color={rgb, 255:red, 255; green, 6; blue, 37 }  ,draw opacity=1 ][line width=0.75]    (10.93,-3.29) .. controls (6.95,-1.4) and (3.31,-0.3) .. (0,0) .. controls (3.31,0.3) and (6.95,1.4) .. (10.93,3.29)   ;
%Straight Lines [id:da9029770043546983]
\draw [color={rgb, 255:red, 255; green, 6; blue, 37 }  ,draw opacity=1 ]   (249.5,158.5) -- (252.7,276.99) ;
\draw [shift={(252.75,278.99)}, rotate = 268.46] [color={rgb, 255:red, 255; green, 6; blue, 37 }  ,draw opacity=1 ][line width=0.75]    (10.93,-3.29) .. controls (6.95,-1.4) and (3.31,-0.3) .. (0,0) .. controls (3.31,0.3) and (6.95,1.4) .. (10.93,3.29)   ;
%Straight Lines [id:da5633211312675341]
\draw [color={rgb, 255:red, 255; green, 6; blue, 37 }  ,draw opacity=1 ]   (263.5,290.5) -- (278,290.5) ;
\draw [shift={(280,290.5)}, rotate = 180] [color={rgb, 255:red, 255; green, 6; blue, 37 }  ,draw opacity=1 ][line width=0.75]    (10.93,-3.29) .. controls (6.95,-1.4) and (3.31,-0.3) .. (0,0) .. controls (3.31,0.3) and (6.95,1.4) .. (10.93,3.29)   ;
%Straight Lines [id:da3541729849335855]
\draw [color={rgb, 255:red, 255; green, 6; blue, 37 }  ,draw opacity=1 ]   (301.5,290.5) -- (315,290.5) ;
\draw [shift={(317,290.5)}, rotate = 180] [color={rgb, 255:red, 255; green, 6; blue, 37 }  ,draw opacity=1 ][line width=0.75]    (10.93,-3.29) .. controls (6.95,-1.4) and (3.31,-0.3) .. (0,0) .. controls (3.31,0.3) and (6.95,1.4) .. (10.93,3.29)   ;
%Straight Lines [id:da12036880586157217]
\draw [color={rgb, 255:red, 255; green, 6; blue, 37 }  ,draw opacity=1 ]   (338.5,290.5) -- (351,290.5) ;
\draw [shift={(353,290.5)}, rotate = 180] [color={rgb, 255:red, 255; green, 6; blue, 37 }  ,draw opacity=1 ][line width=0.75]    (10.93,-3.29) .. controls (6.95,-1.4) and (3.31,-0.3) .. (0,0) .. controls (3.31,0.3) and (6.95,1.4) .. (10.93,3.29)   ;
%Straight Lines [id:da1824186246956936]
\draw [color={rgb, 255:red, 255; green, 6; blue, 37 }  ,draw opacity=1 ]   (374.5,290.5) -- (392,290.5) ;
\draw [shift={(394,290.5)}, rotate = 180] [color={rgb, 255:red, 255; green, 6; blue, 37 }  ,draw opacity=1 ][line width=0.75]    (10.93,-3.29) .. controls (6.95,-1.4) and (3.31,-0.3) .. (0,0) .. controls (3.31,0.3) and (6.95,1.4) .. (10.93,3.29)   ;
%Straight Lines [id:da8221454871134097]
\draw [color={rgb, 255:red, 255; green, 6; blue, 37 }  ,draw opacity=1 ]   (415.5,290.5) -- (431,290.5) ;
\draw [shift={(433,290.5)}, rotate = 180] [color={rgb, 255:red, 255; green, 6; blue, 37 }  ,draw opacity=1 ][line width=0.75]    (10.93,-3.29) .. controls (6.95,-1.4) and (3.31,-0.3) .. (0,0) .. controls (3.31,0.3) and (6.95,1.4) .. (10.93,3.29)   ;
%Straight Lines [id:da9488029220729186]
\draw [color={rgb, 255:red, 255; green, 6; blue, 37 }  ,draw opacity=1 ]   (454.5,290.5) -- (470,289.61) ;
\draw [shift={(472,289.5)}, rotate = 536.73] [color={rgb, 255:red, 255; green, 6; blue, 37 }  ,draw opacity=1 ][line width=0.75]    (10.93,-3.29) .. controls (6.95,-1.4) and (3.31,-0.3) .. (0,0) .. controls (3.31,0.3) and (6.95,1.4) .. (10.93,3.29)   ;
%Straight Lines [id:da34016762564220904]
\draw [color={rgb, 255:red, 255; green, 6; blue, 37 }  ,draw opacity=1 ]   (493.5,289.5) -- (507.32,225.95) ;
\draw [shift={(507.75,224)}, rotate = 462.27] [color={rgb, 255:red, 255; green, 6; blue, 37 }  ,draw opacity=1 ][line width=0.75]    (10.93,-3.29) .. controls (6.95,-1.4) and (3.31,-0.3) .. (0,0) .. controls (3.31,0.3) and (6.95,1.4) .. (10.93,3.29)   ;
%Straight Lines [id:da7160494451689643]
\draw [color={rgb, 255:red, 255; green, 6; blue, 37 }  ,draw opacity=1 ]   (516.5,214.5) -- (522.65,88) ;
\draw [shift={(522.75,86)}, rotate = 452.78] [color={rgb, 255:red, 255; green, 6; blue, 37 }  ,draw opacity=1 ][line width=0.75]    (10.93,-3.29) .. controls (6.95,-1.4) and (3.31,-0.3) .. (0,0) .. controls (3.31,0.3) and (6.95,1.4) .. (10.93,3.29)   ;
%Straight Lines [id:da8287482614528492]
\draw [color={rgb, 255:red, 255; green, 6; blue, 37 }  ,draw opacity=1 ]   (533.5,74.5) -- (549,74.5) ;
\draw [shift={(551,74.5)}, rotate = 180] [color={rgb, 255:red, 255; green, 6; blue, 37 }  ,draw opacity=1 ][line width=0.75]    (10.93,-3.29) .. controls (6.95,-1.4) and (3.31,-0.3) .. (0,0) .. controls (3.31,0.3) and (6.95,1.4) .. (10.93,3.29)   ;
%Straight Lines [id:da9474380471440238]
\draw [color={rgb, 255:red, 255; green, 6; blue, 37 }  ,draw opacity=1 ]   (572.5,74.5) -- (585,74.5) ;
\draw [shift={(587,74.5)}, rotate = 180] [color={rgb, 255:red, 255; green, 6; blue, 37 }  ,draw opacity=1 ][line width=0.75]    (10.93,-3.29) .. controls (6.95,-1.4) and (3.31,-0.3) .. (0,0) .. controls (3.31,0.3) and (6.95,1.4) .. (10.93,3.29)   ;
%Straight Lines [id:da024671690265692714]
\draw [color={rgb, 255:red, 255; green, 6; blue, 37 }  ,draw opacity=1 ]   (608.5,74.5) -- (634.81,124.23) ;
\draw [shift={(635.75,125.99)}, rotate = 242.11] [color={rgb, 255:red, 255; green, 6; blue, 37 }  ,draw opacity=1 ][line width=0.75]    (10.93,-3.29) .. controls (6.95,-1.4) and (3.31,-0.3) .. (0,0) .. controls (3.31,0.3) and (6.95,1.4) .. (10.93,3.29)   ;

% Text Node
\draw (11,128.81) node [anchor=north west][inner sep=0.75pt]   [align=left] {10};
% Text Node
\draw (217,16.9) node [anchor=north west][inner sep=0.75pt]   [align=left] {Linea de Producción 1};
% Text Node
\draw (221,353.1) node [anchor=north west][inner sep=0.75pt]   [align=left] {Linea de Producción 2};
% Text Node
\draw (55,65.81) node [anchor=north west][inner sep=0.75pt]   [align=left] {71};
% Text Node
\draw (92,66.81) node [anchor=north west][inner sep=0.75pt]   [align=left] {63};
% Text Node
\draw (129,65.81) node [anchor=north west][inner sep=0.75pt]   [align=left] {72};
% Text Node
\draw (168,66.81) node [anchor=north west][inner sep=0.75pt]   [align=left] {69};
% Text Node
\draw (207,67.81) node [anchor=north west][inner sep=0.75pt]   [align=left] {16};
% Text Node
\draw (244,67.81) node [anchor=north west][inner sep=0.75pt]   [align=left] {65};
% Text Node
\draw (282,66.81) node [anchor=north west][inner sep=0.75pt]   [align=left] {24};
% Text Node
\draw (319,65.81) node [anchor=north west][inner sep=0.75pt]   [align=left] {57};
% Text Node
\draw (354,66.81) node [anchor=north west][inner sep=0.75pt]   [align=left] {26};
% Text Node
\draw (396,67.81) node [anchor=north west][inner sep=0.75pt]   [align=left] {54};
% Text Node
\draw (434,65.81) node [anchor=north west][inner sep=0.75pt]   [align=left] {49};
% Text Node
\draw (474,66.81) node [anchor=north west][inner sep=0.75pt]   [align=left] {56};
% Text Node
\draw (514,65.81) node [anchor=north west][inner sep=0.75pt]   [align=left] {42};
% Text Node
\draw (553,65.81) node [anchor=north west][inner sep=0.75pt]   [align=left] {15};
% Text Node
\draw (593,65.81) node [anchor=north west][inner sep=0.75pt]   [align=left] {9};
% Text Node
\draw (59,280.81) node [anchor=north west][inner sep=0.75pt]   [align=left] {5};
% Text Node
\draw (95,281.81) node [anchor=north west][inner sep=0.75pt]   [align=left] {7};
% Text Node
\draw (129,280.81) node [anchor=north west][inner sep=0.75pt]   [align=left] {37};
% Text Node
\draw (168,281.81) node [anchor=north west][inner sep=0.75pt]   [align=left] {48};
% Text Node
\draw (207,282.81) node [anchor=north west][inner sep=0.75pt]   [align=left] {44};
% Text Node
\draw (244,282.81) node [anchor=north west][inner sep=0.75pt]   [align=left] {33};
% Text Node
\draw (282,281.81) node [anchor=north west][inner sep=0.75pt]   [align=left] {7};
% Text Node
\draw (319,280.81) node [anchor=north west][inner sep=0.75pt]   [align=left] {19};
% Text Node
\draw (354,281.81) node [anchor=north west][inner sep=0.75pt]   [align=left] {27};
% Text Node
\draw (396,282.81) node [anchor=north west][inner sep=0.75pt]   [align=left] {18};
% Text Node
\draw (434,280.81) node [anchor=north west][inner sep=0.75pt]   [align=left] {39};
% Text Node
\draw (474,281.81) node [anchor=north west][inner sep=0.75pt]   [align=left] {25};
% Text Node
\draw (514,280.81) node [anchor=north west][inner sep=0.75pt]   [align=left] {72};
% Text Node
\draw (553,280.81) node [anchor=north west][inner sep=0.75pt]   [align=left] {43};
% Text Node
\draw (589,280.81) node [anchor=north west][inner sep=0.75pt]   [align=left] {59};
% Text Node
\draw (11,217.81) node [anchor=north west][inner sep=0.75pt]   [align=left] {49};
% Text Node
\draw (627,128.81) node [anchor=north west][inner sep=0.75pt]   [align=left] {30};
% Text Node
\draw (631,215.81) node [anchor=north west][inner sep=0.75pt]   [align=left] {5};
% Text Node
\draw (82,146.81) node [anchor=north west][inner sep=0.75pt]   [align=left] {{\scriptsize 37}};
% Text Node
\draw (82,202.81) node [anchor=north west][inner sep=0.75pt]   [align=left] {{\scriptsize 20}};
% Text Node
\draw (119,147.81) node [anchor=north west][inner sep=0.75pt]   [align=left] {{\scriptsize 18}};
% Text Node
\draw (119,203.81) node [anchor=north west][inner sep=0.75pt]   [align=left] {{\scriptsize 39}};
% Text Node
\draw (157,148.81) node [anchor=north west][inner sep=0.75pt]   [align=left] {{\scriptsize 11}};
% Text Node
\draw (157,204.81) node [anchor=north west][inner sep=0.75pt]   [align=left] {{\scriptsize 13}};
% Text Node
\draw (196,147.81) node [anchor=north west][inner sep=0.75pt]   [align=left] {{\scriptsize 34}};
% Text Node
\draw (196,203.81) node [anchor=north west][inner sep=0.75pt]   [align=left] {{\scriptsize 10}};
% Text Node
\draw (234,148.81) node [anchor=north west][inner sep=0.75pt]   [align=left] {{\scriptsize 5}};
% Text Node
\draw (234,203.81) node [anchor=north west][inner sep=0.75pt]   [align=left] {{\scriptsize 74}};
% Text Node
\draw (272,147.81) node [anchor=north west][inner sep=0.75pt]   [align=left] {{\scriptsize 72}};
% Text Node
\draw (272,203.81) node [anchor=north west][inner sep=0.75pt]   [align=left] {{\scriptsize 11}};
% Text Node
\draw (310,147.81) node [anchor=north west][inner sep=0.75pt]   [align=left] {{\scriptsize 74}};
% Text Node
\draw (310,203.81) node [anchor=north west][inner sep=0.75pt]   [align=left] {{\scriptsize 7}};
% Text Node
\draw (346,147.81) node [anchor=north west][inner sep=0.75pt]   [align=left] {{\scriptsize 71}};
% Text Node
\draw (346,203.81) node [anchor=north west][inner sep=0.75pt]   [align=left] {{\scriptsize 22}};
% Text Node
\draw (383,147.81) node [anchor=north west][inner sep=0.75pt]   [align=left] {{\scriptsize 44}};
% Text Node
\draw (383,203.81) node [anchor=north west][inner sep=0.75pt]   [align=left] {{\scriptsize 36}};
% Text Node
\draw (423,148.81) node [anchor=north west][inner sep=0.75pt]   [align=left] {{\scriptsize 52}};
% Text Node
\draw (423,204.81) node [anchor=north west][inner sep=0.75pt]   [align=left] {{\scriptsize 67}};
% Text Node
\draw (462,147.81) node [anchor=north west][inner sep=0.75pt]   [align=left] {{\scriptsize 64}};
% Text Node
\draw (462,203.81) node [anchor=north west][inner sep=0.75pt]   [align=left] {{\scriptsize 52}};
% Text Node
\draw (501,147.81) node [anchor=north west][inner sep=0.75pt]   [align=left] {{\scriptsize 21}};
% Text Node
\draw (501,203.81) node [anchor=north west][inner sep=0.75pt]   [align=left] {{\scriptsize 52}};
% Text Node
\draw (542,147.81) node [anchor=north west][inner sep=0.75pt]   [align=left] {{\scriptsize 32}};
% Text Node
\draw (542,203.81) node [anchor=north west][inner sep=0.75pt]   [align=left] {{\scriptsize 69}};
% Text Node
\draw (580,147.81) node [anchor=north west][inner sep=0.75pt]   [align=left] {{\scriptsize 41}};
% Text Node
\draw (580,203.81) node [anchor=north west][inner sep=0.75pt]   [align=left] {{\scriptsize 70}};


\end{tikzpicture}

    Figura 55 - Ruta óptima del cuarto ejemplo
\end{center}
\subsubsection{Ejercicio de la diapositiva}
El ejercicio presentado en al final de las diapositivas es el que se muestra en la figura 56. La resolución de este problema a detalle se muestra en el anexo, donde se tendrá que llegar a la misma solución que el problema nos arroje.
\tikzset{every picture/.style={line width=0.75pt}} %set default line width to 0.75pt

\begin{tikzpicture}[x=0.75pt,y=0.75pt,yscale=-1,xscale=1]
%uncomment if require: \path (0,432); %set diagram left start at 0, and has height of 432

%Shape: Rectangle [id:dp3882412922041636]
\draw   (168,28.77) -- (483.5,28.77) -- (483.5,88.2) -- (168,88.2) -- cycle ;
%Shape: Ellipse [id:dp8409179224209302]
\draw   (180,59.97) .. controls (180,49.31) and (188.73,40.66) .. (199.5,40.66) .. controls (210.27,40.66) and (219,49.31) .. (219,59.97) .. controls (219,70.64) and (210.27,79.29) .. (199.5,79.29) .. controls (188.73,79.29) and (180,70.64) .. (180,59.97) -- cycle ;
%Shape: Ellipse [id:dp6618697065833445]
\draw   (264,59.97) .. controls (264,49.31) and (272.73,40.66) .. (283.5,40.66) .. controls (294.27,40.66) and (303,49.31) .. (303,59.97) .. controls (303,70.64) and (294.27,79.29) .. (283.5,79.29) .. controls (272.73,79.29) and (264,70.64) .. (264,59.97) -- cycle ;
%Shape: Ellipse [id:dp5721952092295051]
\draw   (349,57.99) .. controls (349,47.32) and (357.73,38.68) .. (368.5,38.68) .. controls (379.27,38.68) and (388,47.32) .. (388,57.99) .. controls (388,68.66) and (379.27,77.31) .. (368.5,77.31) .. controls (357.73,77.31) and (349,68.66) .. (349,57.99) -- cycle ;
%Shape: Ellipse [id:dp6833723282837854]
\draw   (432,57.99) .. controls (432,47.32) and (440.73,38.68) .. (451.5,38.68) .. controls (462.27,38.68) and (471,47.32) .. (471,57.99) .. controls (471,68.66) and (462.27,77.31) .. (451.5,77.31) .. controls (440.73,77.31) and (432,68.66) .. (432,57.99) -- cycle ;
%Shape: Ellipse [id:dp8985181219604552]
\draw   (96,129.31) .. controls (96,118.64) and (104.73,109.99) .. (115.5,109.99) .. controls (126.27,109.99) and (135,118.64) .. (135,129.31) .. controls (135,139.98) and (126.27,148.62) .. (115.5,148.62) .. controls (104.73,148.62) and (96,139.98) .. (96,129.31) -- cycle ;
%Shape: Ellipse [id:dp00930473167103174]
\draw   (97,201.62) .. controls (97,190.95) and (105.73,182.3) .. (116.5,182.3) .. controls (127.27,182.3) and (136,190.95) .. (136,201.62) .. controls (136,212.28) and (127.27,220.93) .. (116.5,220.93) .. controls (105.73,220.93) and (97,212.28) .. (97,201.62) -- cycle ;
%Shape: Ellipse [id:dp5615897075603871]
\draw   (523,130.3) .. controls (523,119.63) and (531.73,110.98) .. (542.5,110.98) .. controls (553.27,110.98) and (562,119.63) .. (562,130.3) .. controls (562,140.97) and (553.27,149.61) .. (542.5,149.61) .. controls (531.73,149.61) and (523,140.97) .. (523,130.3) -- cycle ;
%Shape: Ellipse [id:dp9065681617862773]
\draw   (525,201.62) .. controls (525,190.95) and (533.73,182.3) .. (544.5,182.3) .. controls (555.27,182.3) and (564,190.95) .. (564,201.62) .. controls (564,212.28) and (555.27,220.93) .. (544.5,220.93) .. controls (533.73,220.93) and (525,212.28) .. (525,201.62) -- cycle ;
%Shape: Rectangle [id:dp909666943337301]
\draw   (168,236.78) -- (484.5,236.78) -- (484.5,296.21) -- (168,296.21) -- cycle ;
%Shape: Ellipse [id:dp9711948519929523]
\draw   (180,267.98) .. controls (180,257.31) and (188.73,248.66) .. (199.5,248.66) .. controls (210.27,248.66) and (219,257.31) .. (219,267.98) .. controls (219,278.65) and (210.27,287.29) .. (199.5,287.29) .. controls (188.73,287.29) and (180,278.65) .. (180,267.98) -- cycle ;
%Shape: Ellipse [id:dp3854952507607323]
\draw   (264,267.98) .. controls (264,257.31) and (272.73,248.66) .. (283.5,248.66) .. controls (294.27,248.66) and (303,257.31) .. (303,267.98) .. controls (303,278.65) and (294.27,287.29) .. (283.5,287.29) .. controls (272.73,287.29) and (264,278.65) .. (264,267.98) -- cycle ;
%Shape: Ellipse [id:dp16404939238349958]
\draw   (349,266.99) .. controls (349,256.32) and (357.73,247.67) .. (368.5,247.67) .. controls (379.27,247.67) and (388,256.32) .. (388,266.99) .. controls (388,277.66) and (379.27,286.3) .. (368.5,286.3) .. controls (357.73,286.3) and (349,277.66) .. (349,266.99) -- cycle ;
%Shape: Ellipse [id:dp7414805922975216]
\draw   (432,266.99) .. controls (432,256.32) and (440.73,247.67) .. (451.5,247.67) .. controls (462.27,247.67) and (471,256.32) .. (471,266.99) .. controls (471,277.66) and (462.27,286.3) .. (451.5,286.3) .. controls (440.73,286.3) and (432,277.66) .. (432,266.99) -- cycle ;
%Shape: Ellipse [id:dp25214142312855037]
\draw   (227,134.01) .. controls (227,126.77) and (232.93,120.89) .. (240.25,120.89) .. controls (247.57,120.89) and (253.5,126.77) .. (253.5,134.01) .. controls (253.5,141.26) and (247.57,147.14) .. (240.25,147.14) .. controls (232.93,147.14) and (227,141.26) .. (227,134.01) -- cycle ;
%Shape: Ellipse [id:dp7893844677118396]
\draw   (227,195.42) .. controls (227,188.18) and (232.93,182.3) .. (240.25,182.3) .. controls (247.57,182.3) and (253.5,188.18) .. (253.5,195.42) .. controls (253.5,202.67) and (247.57,208.55) .. (240.25,208.55) .. controls (232.93,208.55) and (227,202.67) .. (227,195.42) -- cycle ;
%Shape: Ellipse [id:dp926816506125107]
\draw   (313,134.01) .. controls (313,126.77) and (318.93,120.89) .. (326.25,120.89) .. controls (333.57,120.89) and (339.5,126.77) .. (339.5,134.01) .. controls (339.5,141.26) and (333.57,147.14) .. (326.25,147.14) .. controls (318.93,147.14) and (313,141.26) .. (313,134.01) -- cycle ;
%Shape: Ellipse [id:dp8935781498707076]
\draw   (313,195.42) .. controls (313,188.18) and (318.93,182.3) .. (326.25,182.3) .. controls (333.57,182.3) and (339.5,188.18) .. (339.5,195.42) .. controls (339.5,202.67) and (333.57,208.55) .. (326.25,208.55) .. controls (318.93,208.55) and (313,202.67) .. (313,195.42) -- cycle ;
%Shape: Ellipse [id:dp8314486092573847]
\draw   (396,134.01) .. controls (396,126.77) and (401.93,120.89) .. (409.25,120.89) .. controls (416.57,120.89) and (422.5,126.77) .. (422.5,134.01) .. controls (422.5,141.26) and (416.57,147.14) .. (409.25,147.14) .. controls (401.93,147.14) and (396,141.26) .. (396,134.01) -- cycle ;
%Shape: Ellipse [id:dp8639588436591459]
\draw   (397,195.42) .. controls (397,188.18) and (402.93,182.3) .. (410.25,182.3) .. controls (417.57,182.3) and (423.5,188.18) .. (423.5,195.42) .. controls (423.5,202.67) and (417.57,208.55) .. (410.25,208.55) .. controls (402.93,208.55) and (397,202.67) .. (397,195.42) -- cycle ;
%Straight Lines [id:da9324098020632179]
\draw    (223.5,19.86) -- (223.5,315.03) ;
%Straight Lines [id:da507557588103063]
\draw    (308.5,23.86) -- (308.5,319.03) ;
%Straight Lines [id:da03159447721048858]
\draw    (393.5,24.86) -- (393.5,320.03) ;

% Text Node
\draw (196,51.47) node [anchor=north west][inner sep=0.75pt]   [align=left] {7};
% Text Node
\draw (279,51.47) node [anchor=north west][inner sep=0.75pt]   [align=left] {5};
% Text Node
\draw (363,50.48) node [anchor=north west][inner sep=0.75pt]   [align=left] {4};
% Text Node
\draw (447,49.49) node [anchor=north west][inner sep=0.75pt]   [align=left] {6};
% Text Node
\draw (111,120.81) node [anchor=north west][inner sep=0.75pt]   [align=left] {2};
% Text Node
\draw (538,121.8) node [anchor=north west][inner sep=0.75pt]   [align=left] {3};
% Text Node
\draw (540,192.13) node [anchor=north west][inner sep=0.75pt]   [align=left] {4};
% Text Node
\draw (447,258.49) node [anchor=north west][inner sep=0.75pt]   [align=left] {9};
% Text Node
\draw (363,258.49) node [anchor=north west][inner sep=0.75pt]   [align=left] {3};
% Text Node
\draw (278,259.48) node [anchor=north west][inner sep=0.75pt]   [align=left] {8};
% Text Node
\draw (194,259.48) node [anchor=north west][inner sep=0.75pt]   [align=left] {5};
% Text Node
\draw (111,193.12) node [anchor=north west][inner sep=0.75pt]   [align=left] {3};
% Text Node
\draw (235,125.76) node [anchor=north west][inner sep=0.75pt]   [align=left] {2};
% Text Node
\draw (236,187.17) node [anchor=north west][inner sep=0.75pt]   [align=left] {3};
% Text Node
\draw (321,125.76) node [anchor=north west][inner sep=0.75pt]   [align=left] {1};
% Text Node
\draw (321,188.16) node [anchor=north west][inner sep=0.75pt]   [align=left] {2};
% Text Node
\draw (404,125.76) node [anchor=north west][inner sep=0.75pt]   [align=left] {2};
% Text Node
\draw (405,187.17) node [anchor=north west][inner sep=0.75pt]   [align=left] {1};
% Text Node
\draw (236,6.9) node [anchor=north west][inner sep=0.75pt]   [align=left] {Linea de Producción 1};
% Text Node
\draw (212,325.1) node [anchor=north west][inner sep=0.75pt]   [align=left] {Linea de Producción 2};

\end{tikzpicture}
\begin{center}
    Figura 56 - Diagrama ejercicio Lineas de producción
\end{center}
Por lo que empezamos declarando las variables a utilizar en nuestro código, tal como se muestra en la figura 57.
\begin{center}
    \includegraphics[width = 10cm]{LineasProduccion/11.png}\\
    Figura 57 - Declaracion de variables del ejercicio
\end{center}
Posterior a esto, simplemente usamos nuestra función \textit{lineaprod2} y le mandamos las variables que recien declaramos. La solución óptima a este problema se muestra en la figura 58.
\begin{center}
    \includegraphics[width = 6.5cm]{LineasProduccion/12.png}\\
    Figura 58 - Declaracion de variables del ejercicio
\end{center}
Y finalmente de la figura 59 nos podemos apoyar para ver que realmente es la solución óptima a este problema, que ciertamente coincide con la solución también obtenida en el anexo.

\begin{center}


\tikzset{every picture/.style={line width=0.75pt}} %set default line width to 0.75pt

\begin{tikzpicture}[x=0.75pt,y=0.75pt,yscale=-1,xscale=1]
%uncomment if require: \path (0,432); %set diagram left start at 0, and has height of 432

%Shape: Rectangle [id:dp3882412922041636]
\draw   (168,28.77) -- (483.5,28.77) -- (483.5,88.2) -- (168,88.2) -- cycle ;
%Shape: Ellipse [id:dp8409179224209302]
\draw  [color={rgb, 255:red, 255; green, 0; blue, 0 }  ,draw opacity=1 ] (180,59.97) .. controls (180,49.31) and (188.73,40.66) .. (199.5,40.66) .. controls (210.27,40.66) and (219,49.31) .. (219,59.97) .. controls (219,70.64) and (210.27,79.29) .. (199.5,79.29) .. controls (188.73,79.29) and (180,70.64) .. (180,59.97) -- cycle ;
%Shape: Ellipse [id:dp6618697065833445]
\draw  [color={rgb, 255:red, 255; green, 0; blue, 0 }  ,draw opacity=1 ] (264,59.97) .. controls (264,49.31) and (272.73,40.66) .. (283.5,40.66) .. controls (294.27,40.66) and (303,49.31) .. (303,59.97) .. controls (303,70.64) and (294.27,79.29) .. (283.5,79.29) .. controls (272.73,79.29) and (264,70.64) .. (264,59.97) -- cycle ;
%Shape: Ellipse [id:dp5721952092295051]
\draw  [color={rgb, 255:red, 255; green, 0; blue, 0 }  ,draw opacity=1 ] (349,57.99) .. controls (349,47.32) and (357.73,38.68) .. (368.5,38.68) .. controls (379.27,38.68) and (388,47.32) .. (388,57.99) .. controls (388,68.66) and (379.27,77.31) .. (368.5,77.31) .. controls (357.73,77.31) and (349,68.66) .. (349,57.99) -- cycle ;
%Shape: Ellipse [id:dp6833723282837854]
\draw  [color={rgb, 255:red, 250; green, 0; blue, 0 }  ,draw opacity=1 ] (432,57.99) .. controls (432,47.32) and (440.73,38.68) .. (451.5,38.68) .. controls (462.27,38.68) and (471,47.32) .. (471,57.99) .. controls (471,68.66) and (462.27,77.31) .. (451.5,77.31) .. controls (440.73,77.31) and (432,68.66) .. (432,57.99) -- cycle ;
%Shape: Ellipse [id:dp8985181219604552]
\draw  [color={rgb, 255:red, 255; green, 0; blue, 0 }  ,draw opacity=1 ] (96,129.31) .. controls (96,118.64) and (104.73,109.99) .. (115.5,109.99) .. controls (126.27,109.99) and (135,118.64) .. (135,129.31) .. controls (135,139.98) and (126.27,148.62) .. (115.5,148.62) .. controls (104.73,148.62) and (96,139.98) .. (96,129.31) -- cycle ;
%Shape: Ellipse [id:dp00930473167103174]
\draw   (97,201.62) .. controls (97,190.95) and (105.73,182.3) .. (116.5,182.3) .. controls (127.27,182.3) and (136,190.95) .. (136,201.62) .. controls (136,212.28) and (127.27,220.93) .. (116.5,220.93) .. controls (105.73,220.93) and (97,212.28) .. (97,201.62) -- cycle ;
%Shape: Ellipse [id:dp5615897075603871]
\draw  [color={rgb, 255:red, 255; green, 0; blue, 0 }  ,draw opacity=1 ] (523,130.3) .. controls (523,119.63) and (531.73,110.98) .. (542.5,110.98) .. controls (553.27,110.98) and (562,119.63) .. (562,130.3) .. controls (562,140.97) and (553.27,149.61) .. (542.5,149.61) .. controls (531.73,149.61) and (523,140.97) .. (523,130.3) -- cycle ;
%Shape: Ellipse [id:dp9065681617862773]
\draw   (525,201.62) .. controls (525,190.95) and (533.73,182.3) .. (544.5,182.3) .. controls (555.27,182.3) and (564,190.95) .. (564,201.62) .. controls (564,212.28) and (555.27,220.93) .. (544.5,220.93) .. controls (533.73,220.93) and (525,212.28) .. (525,201.62) -- cycle ;
%Shape: Rectangle [id:dp909666943337301]
\draw   (168,236.78) -- (484.5,236.78) -- (484.5,296.21) -- (168,296.21) -- cycle ;
%Shape: Ellipse [id:dp9711948519929523]
\draw   (180,267.98) .. controls (180,257.31) and (188.73,248.66) .. (199.5,248.66) .. controls (210.27,248.66) and (219,257.31) .. (219,267.98) .. controls (219,278.65) and (210.27,287.29) .. (199.5,287.29) .. controls (188.73,287.29) and (180,278.65) .. (180,267.98) -- cycle ;
%Shape: Ellipse [id:dp3854952507607323]
\draw   (264,267.98) .. controls (264,257.31) and (272.73,248.66) .. (283.5,248.66) .. controls (294.27,248.66) and (303,257.31) .. (303,267.98) .. controls (303,278.65) and (294.27,287.29) .. (283.5,287.29) .. controls (272.73,287.29) and (264,278.65) .. (264,267.98) -- cycle ;
%Shape: Ellipse [id:dp16404939238349958]
\draw   (349,266.99) .. controls (349,256.32) and (357.73,247.67) .. (368.5,247.67) .. controls (379.27,247.67) and (388,256.32) .. (388,266.99) .. controls (388,277.66) and (379.27,286.3) .. (368.5,286.3) .. controls (357.73,286.3) and (349,277.66) .. (349,266.99) -- cycle ;
%Shape: Ellipse [id:dp7414805922975216]
\draw   (432,266.99) .. controls (432,256.32) and (440.73,247.67) .. (451.5,247.67) .. controls (462.27,247.67) and (471,256.32) .. (471,266.99) .. controls (471,277.66) and (462.27,286.3) .. (451.5,286.3) .. controls (440.73,286.3) and (432,277.66) .. (432,266.99) -- cycle ;
%Shape: Ellipse [id:dp25214142312855037]
\draw   (227,134.01) .. controls (227,126.77) and (232.93,120.89) .. (240.25,120.89) .. controls (247.57,120.89) and (253.5,126.77) .. (253.5,134.01) .. controls (253.5,141.26) and (247.57,147.14) .. (240.25,147.14) .. controls (232.93,147.14) and (227,141.26) .. (227,134.01) -- cycle ;
%Shape: Ellipse [id:dp7893844677118396]
\draw   (227,195.42) .. controls (227,188.18) and (232.93,182.3) .. (240.25,182.3) .. controls (247.57,182.3) and (253.5,188.18) .. (253.5,195.42) .. controls (253.5,202.67) and (247.57,208.55) .. (240.25,208.55) .. controls (232.93,208.55) and (227,202.67) .. (227,195.42) -- cycle ;
%Shape: Ellipse [id:dp926816506125107]
\draw   (313,134.01) .. controls (313,126.77) and (318.93,120.89) .. (326.25,120.89) .. controls (333.57,120.89) and (339.5,126.77) .. (339.5,134.01) .. controls (339.5,141.26) and (333.57,147.14) .. (326.25,147.14) .. controls (318.93,147.14) and (313,141.26) .. (313,134.01) -- cycle ;
%Shape: Ellipse [id:dp8935781498707076]
\draw   (313,195.42) .. controls (313,188.18) and (318.93,182.3) .. (326.25,182.3) .. controls (333.57,182.3) and (339.5,188.18) .. (339.5,195.42) .. controls (339.5,202.67) and (333.57,208.55) .. (326.25,208.55) .. controls (318.93,208.55) and (313,202.67) .. (313,195.42) -- cycle ;
%Shape: Ellipse [id:dp8314486092573847]
\draw   (396,134.01) .. controls (396,126.77) and (401.93,120.89) .. (409.25,120.89) .. controls (416.57,120.89) and (422.5,126.77) .. (422.5,134.01) .. controls (422.5,141.26) and (416.57,147.14) .. (409.25,147.14) .. controls (401.93,147.14) and (396,141.26) .. (396,134.01) -- cycle ;
%Shape: Ellipse [id:dp8639588436591459]
\draw   (397,195.42) .. controls (397,188.18) and (402.93,182.3) .. (410.25,182.3) .. controls (417.57,182.3) and (423.5,188.18) .. (423.5,195.42) .. controls (423.5,202.67) and (417.57,208.55) .. (410.25,208.55) .. controls (402.93,208.55) and (397,202.67) .. (397,195.42) -- cycle ;
%Straight Lines [id:da9324098020632179]
\draw    (223.5,19.86) -- (223.5,315.03) ;
%Straight Lines [id:da507557588103063]
\draw    (308.5,23.86) -- (308.5,319.03) ;
%Straight Lines [id:da03159447721048858]
\draw    (393.5,24.86) -- (393.5,320.03) ;
%Straight Lines [id:da8867835795288663]
\draw [color={rgb, 255:red, 255; green, 0; blue, 0 }  ,draw opacity=1 ]   (128,115) -- (178.63,61.43) ;
\draw [shift={(180,59.97)}, rotate = 493.38] [color={rgb, 255:red, 255; green, 0; blue, 0 }  ,draw opacity=1 ][line width=0.75]    (10.93,-3.29) .. controls (6.95,-1.4) and (3.31,-0.3) .. (0,0) .. controls (3.31,0.3) and (6.95,1.4) .. (10.93,3.29)   ;
%Straight Lines [id:da15783634325075702]
\draw [color={rgb, 255:red, 255; green, 0; blue, 0 }  ,draw opacity=1 ]   (219,59.97) -- (262,59.97) ;
\draw [shift={(264,59.97)}, rotate = 180] [color={rgb, 255:red, 255; green, 0; blue, 0 }  ,draw opacity=1 ][line width=0.75]    (10.93,-3.29) .. controls (6.95,-1.4) and (3.31,-0.3) .. (0,0) .. controls (3.31,0.3) and (6.95,1.4) .. (10.93,3.29)   ;
%Straight Lines [id:da6523795183891536]
\draw [color={rgb, 255:red, 255; green, 0; blue, 0 }  ,draw opacity=1 ]   (303,59.97) -- (347,58.08) ;
\draw [shift={(349,57.99)}, rotate = 537.53] [color={rgb, 255:red, 255; green, 0; blue, 0 }  ,draw opacity=1 ][line width=0.75]    (10.93,-3.29) .. controls (6.95,-1.4) and (3.31,-0.3) .. (0,0) .. controls (3.31,0.3) and (6.95,1.4) .. (10.93,3.29)   ;
%Straight Lines [id:da9758151143753224]
\draw [color={rgb, 255:red, 255; green, 6; blue, 37 }  ,draw opacity=1 ]   (388,57.99) -- (430,57.99) ;
\draw [shift={(432,57.99)}, rotate = 540] [color={rgb, 255:red, 255; green, 6; blue, 37 }  ,draw opacity=1 ][line width=0.75]    (10.93,-3.29) .. controls (6.95,-1.4) and (3.31,-0.3) .. (0,0) .. controls (3.31,0.3) and (6.95,1.4) .. (10.93,3.29)   ;
%Straight Lines [id:da13453787481760893]
\draw [color={rgb, 255:red, 255; green, 0; blue, 0 }  ,draw opacity=1 ]   (471,57.99) -- (529.07,114.6) ;
\draw [shift={(530.5,116)}, rotate = 224.27] [color={rgb, 255:red, 255; green, 0; blue, 0 }  ,draw opacity=1 ][line width=0.75]    (10.93,-3.29) .. controls (6.95,-1.4) and (3.31,-0.3) .. (0,0) .. controls (3.31,0.3) and (6.95,1.4) .. (10.93,3.29)   ;

% Text Node
\draw (196,51.47) node [anchor=north west][inner sep=0.75pt]   [align=left] {7};
% Text Node
\draw (279,51.47) node [anchor=north west][inner sep=0.75pt]   [align=left] {5};
% Text Node
\draw (363,50.48) node [anchor=north west][inner sep=0.75pt]   [align=left] {4};
% Text Node
\draw (447,49.49) node [anchor=north west][inner sep=0.75pt]   [align=left] {6};
% Text Node
\draw (111,120.81) node [anchor=north west][inner sep=0.75pt]   [align=left] {2};
% Text Node
\draw (538,121.8) node [anchor=north west][inner sep=0.75pt]   [align=left] {3};
% Text Node
\draw (540,192.13) node [anchor=north west][inner sep=0.75pt]   [align=left] {4};
% Text Node
\draw (447,258.49) node [anchor=north west][inner sep=0.75pt]   [align=left] {9};
% Text Node
\draw (363,258.49) node [anchor=north west][inner sep=0.75pt]   [align=left] {3};
% Text Node
\draw (278,259.48) node [anchor=north west][inner sep=0.75pt]   [align=left] {8};
% Text Node
\draw (194,259.48) node [anchor=north west][inner sep=0.75pt]   [align=left] {5};
% Text Node
\draw (111,193.12) node [anchor=north west][inner sep=0.75pt]   [align=left] {3};
% Text Node
\draw (235,125.76) node [anchor=north west][inner sep=0.75pt]   [align=left] {2};
% Text Node
\draw (236,187.17) node [anchor=north west][inner sep=0.75pt]   [align=left] {3};
% Text Node
\draw (321,125.76) node [anchor=north west][inner sep=0.75pt]   [align=left] {1};
% Text Node
\draw (321,188.16) node [anchor=north west][inner sep=0.75pt]   [align=left] {2};
% Text Node
\draw (404,125.76) node [anchor=north west][inner sep=0.75pt]   [align=left] {2};
% Text Node
\draw (405,187.17) node [anchor=north west][inner sep=0.75pt]   [align=left] {1};
% Text Node
\draw (236,6.9) node [anchor=north west][inner sep=0.75pt]   [align=left] {Linea de Producción 1};
% Text Node
\draw (212,325.1) node [anchor=north west][inner sep=0.75pt]   [align=left] {Linea de Producción 2};
\end{tikzpicture}
    \\ Figura 59 - Ruta optima
\end{center}
\newpage

\section{Conclusiones}
Eduardo Mendoza Martínez\newline
\begin{wrapfigure}{r}{0.5\textwidth}
  \begin{center}
    \includegraphics[width=6cm\textwidth]{ed.png}
  \end{center}

\end{wrapfigure}
Al término de la presente práctica fui capaz de comprender el funcionamiento de la programación dinámica y como esta puede mejorar algunos algoritmos que ya han resuelto diversos problemas, pero con programación dinámica se resuelven de una manera más rápida y eficiente, de igual modo, con este concepto se pueden resolver nuevas problemáticas tales como las que vimos en esta práctica con las \textit{Líneas de producción}. Le encontré cierto parecido a lo que se maneja en el paradigma orientado a objetos, ya que en él, se maneja una idea de que "si ya existe, úsalo" y es lo mismo que podemos observar en el algoritmo de \textit{Fibonacci}, en el cual observamos que gran parte de los números de esta sucesión son almacenados y con ellos posteriormente podemos encontrar aún más, o de manera más rápida encontrar alguno que ya se había encontrado directa o indirectamente. En los demás algoritmos hicimos mucho uso de tablas (o matrices), no fueron complicadas de implementar, sin embargo personalmente no había trabajado tanto con arreglos bidimensionales y fue un poco complicado a la hora de obtener índices, desplegar la tabla, pero al final pudimos presentar nuestros datos y las resoluciones a los problemas sin ningún percance. Conocer nuevas formas de resolver problemas es esencial para nosotros como próximos ingenieros, ya que con esto podremos encontrar solución a nuevos problemas o mejorar lo que ya existe, tal como nos lo muestra la programación dinámica.
\newline \newline \newline \newpage
Daniel Aguilar Gonzalez\newline

El desarrollo de esta práctica me ayudo a entender un poco más a fondo como es que funciona la programación dinámica y como gracias a esta podemos dar solución a algoritmos que ya habiamos desarrollado en prácticas anteriores pero con otro método como fue el caso de fibonacci.

\begin{wrapfigure}{r}{0.5\textwidth}
  \begin{center}
    \includegraphics[width=6cm\textwidth]{dan.png}
\end{center}
\end{wrapfigure}
En esta práctica vimos principalmente programación dinámica y como hacer más eficiente la implementación de algunos algoritmos como lo fue fibonacci el cual se implemento con los dos enfoques que maneja la programación dinámica top-down arriba hacia abajo y bottom-up abajo hacia arriba, me ayudo a identificar cual de los dos esmás eficiente gracias a un análisis
realizado a través de pruebas plasmadas en graficas llegando a la conclusión de que el más eficiente es top-down.
La programación dinámica es un método el cual puede tambien utilizar algún otro es decir recursividad, como se vio en los algoritmos anteriores y esto lo hace más eficiente ya que aplica la forma de que si ya calculamos algo una vez no es necesario hacerlo dos veces "Si ya lo tenemos, usemoslo" y eso nos ayuda a no hacer calculos innecesarios, repetir un proceso que ya nos dio un resultado ahorrando asi tiempo y recursos.
Implementamos el algoritmo de lineas de producción el cual en lo personal me parecio de mucha importancia ya que se puede aplicar en la vida real porque nos ayuda a encontrar un camino óptimo por el cual debe pasar un producto para producirse de manera más rápida. Se presentó un ejercicio para el cual hicimos su proceso a mano primeramente y despues en el programa implementado coincidiendo en resultados mostrando la misma ruta óptima de producción.
Es de bital importancia saber diferentes maneras de atacar un algoritmo para poder determinar cual de todas ellas nos conviene mas implementar para ahorrar tiempo y recursos y hacer más eficiente.
\newpage
\section{Anexo}
En  esta  seccíon  veremos  algunos  problemas  anexos complementarios  de  los algoritmos  vistos  en la  pŕactica.
\subsection{Lineas de Produccion}
Considerando 2 lineas de producción donde cada nodo representa una estación de trabajo y cada numero dentro el nodo es el tiempo que tarda en operar esa estación como se muestra en la siguiente figura: \\

\tikzset{every picture/.style={line width=0.75pt}} %set default line width to 0.75pt

\begin{tikzpicture}[x=0.75pt,y=0.75pt,yscale=-1,xscale=1]
%uncomment if require: \path (0,432); %set diagram left start at 0, and has height of 432

%Shape: Rectangle [id:dp3882412922041636]
\draw   (168,28.77) -- (483.5,28.77) -- (483.5,88.2) -- (168,88.2) -- cycle ;
%Shape: Ellipse [id:dp8409179224209302]
\draw   (180,59.97) .. controls (180,49.31) and (188.73,40.66) .. (199.5,40.66) .. controls (210.27,40.66) and (219,49.31) .. (219,59.97) .. controls (219,70.64) and (210.27,79.29) .. (199.5,79.29) .. controls (188.73,79.29) and (180,70.64) .. (180,59.97) -- cycle ;
%Shape: Ellipse [id:dp6618697065833445]
\draw   (264,59.97) .. controls (264,49.31) and (272.73,40.66) .. (283.5,40.66) .. controls (294.27,40.66) and (303,49.31) .. (303,59.97) .. controls (303,70.64) and (294.27,79.29) .. (283.5,79.29) .. controls (272.73,79.29) and (264,70.64) .. (264,59.97) -- cycle ;
%Shape: Ellipse [id:dp5721952092295051]
\draw   (349,57.99) .. controls (349,47.32) and (357.73,38.68) .. (368.5,38.68) .. controls (379.27,38.68) and (388,47.32) .. (388,57.99) .. controls (388,68.66) and (379.27,77.31) .. (368.5,77.31) .. controls (357.73,77.31) and (349,68.66) .. (349,57.99) -- cycle ;
%Shape: Ellipse [id:dp6833723282837854]
\draw   (432,57.99) .. controls (432,47.32) and (440.73,38.68) .. (451.5,38.68) .. controls (462.27,38.68) and (471,47.32) .. (471,57.99) .. controls (471,68.66) and (462.27,77.31) .. (451.5,77.31) .. controls (440.73,77.31) and (432,68.66) .. (432,57.99) -- cycle ;
%Shape: Ellipse [id:dp8985181219604552]
\draw   (96,129.31) .. controls (96,118.64) and (104.73,109.99) .. (115.5,109.99) .. controls (126.27,109.99) and (135,118.64) .. (135,129.31) .. controls (135,139.98) and (126.27,148.62) .. (115.5,148.62) .. controls (104.73,148.62) and (96,139.98) .. (96,129.31) -- cycle ;
%Shape: Ellipse [id:dp00930473167103174]
\draw   (97,201.62) .. controls (97,190.95) and (105.73,182.3) .. (116.5,182.3) .. controls (127.27,182.3) and (136,190.95) .. (136,201.62) .. controls (136,212.28) and (127.27,220.93) .. (116.5,220.93) .. controls (105.73,220.93) and (97,212.28) .. (97,201.62) -- cycle ;
%Shape: Ellipse [id:dp5615897075603871]
\draw   (523,130.3) .. controls (523,119.63) and (531.73,110.98) .. (542.5,110.98) .. controls (553.27,110.98) and (562,119.63) .. (562,130.3) .. controls (562,140.97) and (553.27,149.61) .. (542.5,149.61) .. controls (531.73,149.61) and (523,140.97) .. (523,130.3) -- cycle ;
%Shape: Ellipse [id:dp9065681617862773]
\draw   (525,201.62) .. controls (525,190.95) and (533.73,182.3) .. (544.5,182.3) .. controls (555.27,182.3) and (564,190.95) .. (564,201.62) .. controls (564,212.28) and (555.27,220.93) .. (544.5,220.93) .. controls (533.73,220.93) and (525,212.28) .. (525,201.62) -- cycle ;
%Shape: Rectangle [id:dp909666943337301]
\draw   (168,236.78) -- (484.5,236.78) -- (484.5,296.21) -- (168,296.21) -- cycle ;
%Shape: Ellipse [id:dp9711948519929523]
\draw   (180,267.98) .. controls (180,257.31) and (188.73,248.66) .. (199.5,248.66) .. controls (210.27,248.66) and (219,257.31) .. (219,267.98) .. controls (219,278.65) and (210.27,287.29) .. (199.5,287.29) .. controls (188.73,287.29) and (180,278.65) .. (180,267.98) -- cycle ;
%Shape: Ellipse [id:dp3854952507607323]
\draw   (264,267.98) .. controls (264,257.31) and (272.73,248.66) .. (283.5,248.66) .. controls (294.27,248.66) and (303,257.31) .. (303,267.98) .. controls (303,278.65) and (294.27,287.29) .. (283.5,287.29) .. controls (272.73,287.29) and (264,278.65) .. (264,267.98) -- cycle ;
%Shape: Ellipse [id:dp16404939238349958]
\draw   (349,266.99) .. controls (349,256.32) and (357.73,247.67) .. (368.5,247.67) .. controls (379.27,247.67) and (388,256.32) .. (388,266.99) .. controls (388,277.66) and (379.27,286.3) .. (368.5,286.3) .. controls (357.73,286.3) and (349,277.66) .. (349,266.99) -- cycle ;
%Shape: Ellipse [id:dp7414805922975216]
\draw   (432,266.99) .. controls (432,256.32) and (440.73,247.67) .. (451.5,247.67) .. controls (462.27,247.67) and (471,256.32) .. (471,266.99) .. controls (471,277.66) and (462.27,286.3) .. (451.5,286.3) .. controls (440.73,286.3) and (432,277.66) .. (432,266.99) -- cycle ;
%Shape: Ellipse [id:dp25214142312855037]
\draw   (227,134.01) .. controls (227,126.77) and (232.93,120.89) .. (240.25,120.89) .. controls (247.57,120.89) and (253.5,126.77) .. (253.5,134.01) .. controls (253.5,141.26) and (247.57,147.14) .. (240.25,147.14) .. controls (232.93,147.14) and (227,141.26) .. (227,134.01) -- cycle ;
%Shape: Ellipse [id:dp7893844677118396]
\draw   (227,195.42) .. controls (227,188.18) and (232.93,182.3) .. (240.25,182.3) .. controls (247.57,182.3) and (253.5,188.18) .. (253.5,195.42) .. controls (253.5,202.67) and (247.57,208.55) .. (240.25,208.55) .. controls (232.93,208.55) and (227,202.67) .. (227,195.42) -- cycle ;
%Shape: Ellipse [id:dp926816506125107]
\draw   (313,134.01) .. controls (313,126.77) and (318.93,120.89) .. (326.25,120.89) .. controls (333.57,120.89) and (339.5,126.77) .. (339.5,134.01) .. controls (339.5,141.26) and (333.57,147.14) .. (326.25,147.14) .. controls (318.93,147.14) and (313,141.26) .. (313,134.01) -- cycle ;
%Shape: Ellipse [id:dp8935781498707076]
\draw   (313,195.42) .. controls (313,188.18) and (318.93,182.3) .. (326.25,182.3) .. controls (333.57,182.3) and (339.5,188.18) .. (339.5,195.42) .. controls (339.5,202.67) and (333.57,208.55) .. (326.25,208.55) .. controls (318.93,208.55) and (313,202.67) .. (313,195.42) -- cycle ;
%Shape: Ellipse [id:dp8314486092573847]
\draw   (396,134.01) .. controls (396,126.77) and (401.93,120.89) .. (409.25,120.89) .. controls (416.57,120.89) and (422.5,126.77) .. (422.5,134.01) .. controls (422.5,141.26) and (416.57,147.14) .. (409.25,147.14) .. controls (401.93,147.14) and (396,141.26) .. (396,134.01) -- cycle ;
%Shape: Ellipse [id:dp8639588436591459]
\draw   (397,195.42) .. controls (397,188.18) and (402.93,182.3) .. (410.25,182.3) .. controls (417.57,182.3) and (423.5,188.18) .. (423.5,195.42) .. controls (423.5,202.67) and (417.57,208.55) .. (410.25,208.55) .. controls (402.93,208.55) and (397,202.67) .. (397,195.42) -- cycle ;
%Straight Lines [id:da9324098020632179]
\draw    (223.5,19.86) -- (223.5,315.03) ;
%Straight Lines [id:da507557588103063]
\draw    (308.5,23.86) -- (308.5,319.03) ;
%Straight Lines [id:da03159447721048858]
\draw    (393.5,24.86) -- (393.5,320.03) ;

% Text Node
\draw (196,51.47) node [anchor=north west][inner sep=0.75pt]   [align=left] {7};
% Text Node
\draw (279,51.47) node [anchor=north west][inner sep=0.75pt]   [align=left] {5};
% Text Node
\draw (363,50.48) node [anchor=north west][inner sep=0.75pt]   [align=left] {4};
% Text Node
\draw (447,49.49) node [anchor=north west][inner sep=0.75pt]   [align=left] {6};
% Text Node
\draw (111,120.81) node [anchor=north west][inner sep=0.75pt]   [align=left] {2};
% Text Node
\draw (538,121.8) node [anchor=north west][inner sep=0.75pt]   [align=left] {3};
% Text Node
\draw (540,192.13) node [anchor=north west][inner sep=0.75pt]   [align=left] {4};
% Text Node
\draw (447,258.49) node [anchor=north west][inner sep=0.75pt]   [align=left] {9};
% Text Node
\draw (363,258.49) node [anchor=north west][inner sep=0.75pt]   [align=left] {3};
% Text Node
\draw (278,259.48) node [anchor=north west][inner sep=0.75pt]   [align=left] {8};
% Text Node
\draw (194,259.48) node [anchor=north west][inner sep=0.75pt]   [align=left] {5};
% Text Node
\draw (111,193.12) node [anchor=north west][inner sep=0.75pt]   [align=left] {3};
% Text Node
\draw (235,125.76) node [anchor=north west][inner sep=0.75pt]   [align=left] {2};
% Text Node
\draw (236,187.17) node [anchor=north west][inner sep=0.75pt]   [align=left] {3};
% Text Node
\draw (321,125.76) node [anchor=north west][inner sep=0.75pt]   [align=left] {1};
% Text Node
\draw (321,188.16) node [anchor=north west][inner sep=0.75pt]   [align=left] {2};
% Text Node
\draw (404,125.76) node [anchor=north west][inner sep=0.75pt]   [align=left] {2};
% Text Node
\draw (405,187.17) node [anchor=north west][inner sep=0.75pt]   [align=left] {1};
% Text Node
\draw (236,6.9) node [anchor=north west][inner sep=0.75pt]   [align=left] {Linea de Producción 1};
% Text Node
\draw (212,325.1) node [anchor=north west][inner sep=0.75pt]   [align=left] {Linea de Producción 2};

\end{tikzpicture}
\begin{center}
    Figura 60 - Diagrama ejercicio Lineas de producción
\end{center}
\\ \\
Los nodos que se encuentran al centro de ambas estaciones son los tiempos que tarda en pasar de una estación a otra. El nodo más cercano a la linea de producción es el tiempo que tarda en pasar de esa linea a la otra. \\
Los nodos que se encuentran a la izquierda de las lineas son los tiempos que tarda el producto en entrar en ellas y los de la derecha son los tiempos que tarda en salir.
Las lineas plasmadas verticalmente marcan las estaciones de trabajo.
Mencionado lo anterior procederemos a resolver un ejercicio.

\textbf{Problema.}

Encontrar el camino más optimo por el cual debe pasar un producto para su elaboración. Determinando cual es la suma total f *, cual es la linea más optima I* y el camino que debe seguir el producto dentro de las lineas.

\textbf{Desarrollo.}
\newline
Para dar solucion a este ejercicio debemos llenar dos tablas una para f * que nos permite saber la suma optima que se obtiene de pasar por las estaciones de trabajo y la de I* que nos permite saber la linea de producción final que se obtuvo. \\
\textbf{f *.} Para obtener f * construiremos una tabla en la cual iremos registrando la suma de los tiempos que se obtienen cada que el producto pasa por una estación de trabajo para ambas líneas.\\
\textbf{I *.} Para obtener I* haremos una tabla en la cual registraremos el numero de la linea que viene la suma, es decir al llegar a una estación el producto puede venir de la misma linea o bien de otra.\\
Comenzaremos con el desarrollo del ejercicio: \\\\
 Estación 1\\
- f{1} [1]. \\
Comenzaremos por analizar la linea 1 donde como podemos observar nuestro primer nodo tiene un valor de 2 y el siguiente un valor de 7 así que procedemos a realizar la suma y el resultado sera agregado a la tabla.\\
- f{2} [1]\\
Para comenzar en la linea 2 observamos que tenemos el valor de 2 seguido de 5 asi que los sumaremos y el resultado sera agregado a la tabla.
\begin{center}
   Linea 1.  $2 + 7 = 9$\\
   Linea 2.  $3 + 5 = 8$\\
    \begin{table}[!h]
        \centering

\begin{tabular}{|p{0.20\textwidth}|p{0.20\textwidth}|p{0.20\textwidth}|p{0.20\textwidth}|p{0.20\textwidth}|}
\hline
 \begin{center}
j
\end{center}
 & \begin{center}
1
\end{center}
 & \begin{center}
2
\end{center}
 & \begin{center}
3
\end{center}
 & \begin{center}
4
\end{center}
 \\
\hline
 \begin{center}
f1[n]
\end{center}
 & \begin{center}
9
\end{center}
 & \begin{center}
\end{center}
 & \begin{center}
\end{center}
 & \begin{center}
\end{center}
 \\
\hline
 \begin{center}
f2[n]
\end{center}
 & \begin{center}
8
\end{center}
 & \begin{center}
\end{center}
 & \begin{center}
\end{center}
 & \begin{center}
\end{center}
 \\
 \hline
\end{tabular}
        \caption{Estacion 1}
        \end{table}
\end{center}
 Estación 2\\
 Para la estación 2 haremos una suma del tiempo obtenido de la estacion 1 en la misma linea mas el tiempo del nodo de la estacion 2 y una suma del tiempo obtenido de la estacion 1 de la otra linea más el tiempo de cambio de linea más el tiempo del nodo de la estacion 2, haremos una comparacion entre ellos y colocaremos en la tabla el tiempo menor.\\
 Observaremos de que linea es donde viene el numero menor si de la actual o viene de un cambio y se colocara dentro del recuadro el numero de la linea que viene.\\ \\
- f{1}[2]. \\
Tiempo sobre linea 1 + tiempo de nodo y tiempo sobre linea 2 + tiempo de cambio + tiempo de nodo\\
\begin{center}
   L1.  $9 + 5 = 14$\\
   L2.  $8 + 3 + 5 = 16$\\
    $14 < 16 = 14$\\
    14 proviene de L1
\end{center}

- f{2}[2]\\
Tiempo sobre linea 1 + tiempo de cambio + tiempo de nodo y tiempo sobre linea 2 + tiempo de nodo.
\begin{center}
   L1.  $9 + 2 + 8 = 19$\\
   L2.  $8 + 8 = 16$\\
    $16 < 19 = 16$\\
    16 proviene de L2
\end{center}
\begin{center}
\begin{table}[!h]
        \centering

\begin{tabular}{|p{0.20\textwidth}|p{0.20\textwidth}|p{0.20\textwidth}|p{0.20\textwidth}|p{0.20\textwidth}|}
\hline
 \begin{center}
j
\end{center}
 & \begin{center}
1
\end{center}
 & \begin{center}
2
\end{center}
 & \begin{center}
3
\end{center}
 & \begin{center}
4
\end{center}
 \\
\hline
 \begin{center}
f1[n]
\end{center}
 & \begin{center}
9
\end{center}
 & \begin{center}
 \ \ 14 \ \boxed{1}
\end{center}
 & \begin{center}
\end{center}
 & \begin{center}
\end{center}
 \\
\hline
 \begin{center}
f2[n]
\end{center}
 & \begin{center}
8
\end{center}
 & \begin{center}
 \ \ 16 \ \boxed{2}
\end{center}
 & \begin{center}
\end{center}
 & \begin{center}
\end{center}
 \\
 \hline
\end{tabular}
        \caption{Estacion 2}
        \end{table}
\end{center}

 Estación 3\\
 Para la estación 3 realizaremos el mismo procedimiento que la estación anterior\\ \\
- f{1}[3]. \\
Tiempo sobre linea 1 + tiempo de nodo y tiempo sobre linea 2 + tiempo de cambio + tiempo de nodo\\
\begin{center}
   L1.  $14 + 4 = 18$\\
   L2.  $16 + 2 + 4 = 22$\\
    $18 < 22 = 18$\\
    18 proviene de L1
\end{center}

- f{2}[3]\\
Tiempo sobre linea 1 + tiempo de cambio + tiempo de nodo y tiempo sobre linea 2 + tiempo de nodo.
\begin{center}
   L1.  $14 + 1 + 3 = 18$\\
   L2.  $16 + 3 = 19$\\
    $18 < 19 = 18$\\
    18 proviene de L1
\end{center}
\begin{center}
\begin{table}[!h]
        \centering

\begin{tabular}{|p{0.20\textwidth}|p{0.20\textwidth}|p{0.20\textwidth}|p{0.20\textwidth}|p{0.20\textwidth}|}
\hline
 \begin{center}
j
\end{center}
 & \begin{center}
1
\end{center}
 & \begin{center}
2
\end{center}
 & \begin{center}
3
\end{center}
 & \begin{center}
4
\end{center}
 \\
\hline
 \begin{center}
f1[n]
\end{center}
 & \begin{center}
9
\end{center}
 & \begin{center}
 \ \ 14 \ \boxed{1}
\end{center}
 & \begin{center}
 \ \ \ 18 \boxed{1}
\end{center}
 & \begin{center}
\end{center}
 \\
\hline
 \begin{center}
f2[n]
\end{center}
 & \begin{center}
8
\end{center}
 & \begin{center}
 \ \ 16 \ \boxed{2}
\end{center}
 & \begin{center}
 \ \ \ 18 \boxed{1}
\end{center}
 & \begin{center}
\end{center}
 \\
 \hline
\end{tabular}
        \caption{Estacion 3}
        \end{table}
\end{center}
 Estación 4\\
 Para la estación 4 realizaremos el mismo procedimiento que la estación anterior\\ \\
- f{1}[4]. \\
Tiempo sobre linea 1 + tiempo de nodo y tiempo sobre linea 2 + tiempo de cambio + tiempo de nodo\\
\begin{center}
   L1.  $18 + 6 = 24$\\
   L2.  $18 + 1 + 6 = 25$\\
    $24 < 25 = 18$\\
    24 proviene de L1
\end{center}

- f{2}[4]\\
Tiempo sobre linea 1 + tiempo de cambio + tiempo de nodo y tiempo sobre linea 2 + tiempo de nodo.
\begin{center}
   L1.  $18 + 2 + 9 = 29$\\
   L2.  $18 + 9 = 27$\\
    $27 < 29 = 27$\\
    27 proviene de L2
\end{center}
\begin{center}
    \begin{table}[!h]
        \centering

\begin{tabular}{|p{0.20\textwidth}|p{0.20\textwidth}|p{0.20\textwidth}|p{0.20\textwidth}|p{0.20\textwidth}|}
\hline
 \begin{center}
j
\end{center}
 & \begin{center}
1
\end{center}
 & \begin{center}
2
\end{center}
 & \begin{center}
3
\end{center}
 & \begin{center}
4
\end{center}
 \\
\hline
 \begin{center}
f1[n]
\end{center}
 & \begin{center}
9
\end{center}
 & \begin{center}
 \ \ 14 \ \boxed{1}
\end{center}
 & \begin{center}
 \ \ \ 18 \boxed{1}
\end{center}
 & \begin{center}
 \ \ \ 24 \boxed{1}
\end{center}
 \\
\hline
 \begin{center}
f2[n]
\end{center}
 & \begin{center}
8
\end{center}
 & \begin{center}
 \ \ 16 \ \boxed{2}
\end{center}
 & \begin{center}
 \ \ \ 18 \boxed{1}
\end{center}
 & \begin{center}
 \ \ \ 27 \boxed{2}
\end{center}
 \\
 \hline
\end{tabular}
        \caption{Estacion 4}
        \end{table}
\end{center}
\\
Ya finalizamos nuestra tabla, lo siguiente es al resultado de la ultima estacion sumarle los tiempos de salida los cuales se encuentran en los nodos de la derecha del diagrama es decir:

\begin{center}
   f1[n] =  $24 + 3 = 27$\\
   \\
   f2[n] =  $27 + 4 = 31$\\
\end{center}
\\
Ahora bien al tener ambos resultados elegiremos el resultado menor y ese sera nuestro valor de f *, es decir \\
\begin{center}
   $f* = 27$
\end{center}

\\
Lo siguiente que debemos hacer es hacer la tabla para poder determinar el valor de I*. Esta tabla se forma con los numeros que fuimos poniendo dentro de los recuadros pequeños es decir de las lineas que provenian las sumas menores asi que nos queda de la siguiente manera:
\begin{center}
    \begin{table}[!h]
        \centering

\begin{tabular}{|p{0.25\textwidth}|p{0.25\textwidth}|p{0.25\textwidth}|p{0.25\textwidth}|}
\hline
 \begin{center}
j
\end{center}
 & \begin{center}
2
\end{center}
 & \begin{center}
3
\end{center}
 & \begin{center}
4
\end{center}
 \\
\hline
 \begin{center}
f1[j]
\end{center}
 & \begin{center}
1
\end{center}
 & \begin{center}
1
\end{center}
 & \begin{center}
1
\end{center}
 \\
\hline
 \begin{center}
f2[j]
\end{center}
 & \begin{center}
2
\end{center}
 & \begin{center}
1
\end{center}
 & \begin{center}
2
\end{center}
 \\
 \hline
\end{tabular}
        \caption{Tabla para I*}
        \end{table}
\end{center}
Teniendo nuestra tabla completa analizaremos la columnna de la ultima estación es decir la que contiene la estación numero 4 y nuestra tabla anterior (cuadro 4).
Como podemos observar en nuestra Tabla 4 el valor menor se encuentra en la linea 1 y en nuestra tabla 5 vemos que proviene de la linea 1 asi que este es el valor que determina la I*.
\begin{center}
   $I* = 1$
\end{center}
\\
Obtenidos los valores de I* y f* vamos a formar la ruta que permite el tiempo optimo . Analizaremos cada columna de la tabla 4 iniciando de derecha a izquierda comparando los valores de ambas lineas y determinando el menor, al saber cual es el menor iremos a la tabla 5 y encerraremos de que linea proviene ejemplo para la columna 4 el menor es el 24 que se encuentra en linea 1, vamos a tabla 5 y encerramos el numero 1 de la columna 4 y asi sucesivamente.
Finalmente nuestra tabla nos queda de esta manera:
\begin{center}

\begin{table}[!h]
        \centering

\begin{tabular}{|p{0.25\textwidth}|p{0.25\textwidth}|p{0.25\textwidth}|p{0.25\textwidth}|}
\hline
 \begin{center}
j
\end{center}
 & \begin{center}
2
\end{center}
 & \begin{center}
3
\end{center}
 & \begin{center}
4
\end{center}
 \\
\hline
 \begin{center}
f1[j]
\end{center}
 & \begin{center}



\tikzset{every picture/.style={line width=0.75pt}} %set default line width to 0.75pt

\begin{tikzpicture}[x=0.75pt,y=0.75pt,yscale=-1,xscale=1]
%uncomment if require: \path (0,48); %set diagram left start at 0, and has height of 48

%Shape: Circle [id:dp423740537088205]
\draw   (9,22.75) .. controls (9,13.5) and (16.5,6) .. (25.75,6) .. controls (35,6) and (42.5,13.5) .. (42.5,22.75) .. controls (42.5,32) and (35,39.5) .. (25.75,39.5) .. controls (16.5,39.5) and (9,32) .. (9,22.75) -- cycle ;

% Text Node
\draw (20,15) node [anchor=north west][inner sep=0.75pt]   [align=left] {1};


\end{tikzpicture}
\end{center}
 & \begin{center}



\tikzset{every picture/.style={line width=0.75pt}} %set default line width to 0.75pt

\begin{tikzpicture}[x=0.75pt,y=0.75pt,yscale=-1,xscale=1]
%uncomment if require: \path (0,48); %set diagram left start at 0, and has height of 48

%Shape: Circle [id:dp09457656318950414]
\draw   (9,22.75) .. controls (9,13.5) and (16.5,6) .. (25.75,6) .. controls (35,6) and (42.5,13.5) .. (42.5,22.75) .. controls (42.5,32) and (35,39.5) .. (25.75,39.5) .. controls (16.5,39.5) and (9,32) .. (9,22.75) -- cycle ;

% Text Node
\draw (20,15) node [anchor=north west][inner sep=0.75pt]   [align=left] {1};


\end{tikzpicture}
\end{center}
 & \begin{center}



\tikzset{every picture/.style={line width=0.75pt}} %set default line width to 0.75pt

\begin{tikzpicture}[x=0.75pt,y=0.75pt,yscale=-1,xscale=1]
%uncomment if require: \path (0,48); %set diagram left start at 0, and has height of 48

%Shape: Circle [id:dp5137991725581907]
\draw   (9,22.75) .. controls (9,13.5) and (16.5,6) .. (25.75,6) .. controls (35,6) and (42.5,13.5) .. (42.5,22.75) .. controls (42.5,32) and (35,39.5) .. (25.75,39.5) .. controls (16.5,39.5) and (9,32) .. (9,22.75) -- cycle ;

% Text Node
\draw (20,14) node [anchor=north west][inner sep=0.75pt]   [align=left] {1};


\end{tikzpicture}
\end{center}
 \\
\hline
 \begin{center}
f2[j]
\end{center}
 & \begin{center}
2
\end{center}
 & \begin{center}
1
\end{center}
 & \begin{center}
2
\end{center}
 \\
 \hline
\end{tabular}
        \caption{Tabla para I*}
        \end{table}
\end{center}
\newline



\newpage
Ya identificado de donde proceden los valores menores usaremos el diagrama principal. Nuestro tiempo de salida fue 3 esto quiere decir que proviene de la estacion 4 de la linea 1: \\
\begin{center}


\tikzset{every picture/.style={line width=0.75pt}} %set default line width to 0.75pt

\begin{tikzpicture}[x=0.75pt,y=0.75pt,yscale=-1,xscale=1]
%uncomment if require: \path (0,432); %set diagram left start at 0, and has height of 432

%Shape: Ellipse [id:dp18492110556143815]
\draw   (543,150.3) .. controls (543,139.63) and (551.73,130.98) .. (562.5,130.98) .. controls (573.27,130.98) and (582,139.63) .. (582,150.3) .. controls (582,160.97) and (573.27,169.61) .. (562.5,169.61) .. controls (551.73,169.61) and (543,160.97) .. (543,150.3) -- cycle ;
%Shape: Ellipse [id:dp42472144728140937]
\draw   (472,97.99) .. controls (472,87.32) and (480.73,78.68) .. (491.5,78.68) .. controls (502.27,78.68) and (511,87.32) .. (511,97.99) .. controls (511,108.66) and (502.27,117.31) .. (491.5,117.31) .. controls (480.73,117.31) and (472,108.66) .. (472,97.99) -- cycle ;
%Shape: Rectangle [id:dp5450541714180612]
\draw   (208,68.77) -- (523.5,68.77) -- (523.5,128.2) -- (208,128.2) -- cycle ;
%Straight Lines [id:da7137992761743133]
\draw    (510,110) -- (544.87,134.84) ;
\draw [shift={(546.5,136)}, rotate = 215.46] [color={rgb, 255:red, 0; green, 0; blue, 0 }  ][line width=0.75]    (10.93,-3.29) .. controls (6.95,-1.4) and (3.31,-0.3) .. (0,0) .. controls (3.31,0.3) and (6.95,1.4) .. (10.93,3.29)   ;

% Text Node
\draw (558,141.8) node [anchor=north west][inner sep=0.75pt]   [align=left] {3};
% Text Node
\draw (487,89.49) node [anchor=north west][inner sep=0.75pt]   [align=left] {6};
% Text Node
\draw (276,46.9) node [anchor=north west][inner sep=0.75pt]   [align=left] {Linea de Producción 1};


\end{tikzpicture}
\\ Figura 61 - Union Salida con Estacion 4
\end{center}
\\
El haber encerrado nuestras procedencias nos ayuda para conectar nuestra estacion actual con una anterior. Posicionados en la columna 4 buscamos el numero encerrado y nos damos cuenta que su procedencia viene por Linea 1 como se muestra a continuación.
\begin{center}


\tikzset{every picture/.style={line width=0.75pt}} %set default line width to 0.75pt

\begin{tikzpicture}[x=0.75pt,y=0.75pt,yscale=-1,xscale=1]
%uncomment if require: \path (0,432); %set diagram left start at 0, and has height of 432

%Shape: Ellipse [id:dp9306340350103373]
\draw   (543,150.3) .. controls (543,139.63) and (551.73,130.98) .. (562.5,130.98) .. controls (573.27,130.98) and (582,139.63) .. (582,150.3) .. controls (582,160.97) and (573.27,169.61) .. (562.5,169.61) .. controls (551.73,169.61) and (543,160.97) .. (543,150.3) -- cycle ;
%Shape: Ellipse [id:dp3704662324966781]
\draw   (472,97.99) .. controls (472,87.32) and (480.73,78.68) .. (491.5,78.68) .. controls (502.27,78.68) and (511,87.32) .. (511,97.99) .. controls (511,108.66) and (502.27,117.31) .. (491.5,117.31) .. controls (480.73,117.31) and (472,108.66) .. (472,97.99) -- cycle ;
%Shape: Rectangle [id:dp969276104491277]
\draw   (208,68.77) -- (523.5,68.77) -- (523.5,128.2) -- (208,128.2) -- cycle ;
%Straight Lines [id:da6090127514263177]
\draw    (510,110) -- (544.87,134.84) ;
\draw [shift={(546.5,136)}, rotate = 215.46] [color={rgb, 255:red, 0; green, 0; blue, 0 }  ][line width=0.75]    (10.93,-3.29) .. controls (6.95,-1.4) and (3.31,-0.3) .. (0,0) .. controls (3.31,0.3) and (6.95,1.4) .. (10.93,3.29)   ;
%Shape: Ellipse [id:dp40882552200266686]
\draw   (389,97.99) .. controls (389,87.32) and (397.73,78.68) .. (408.5,78.68) .. controls (419.27,78.68) and (428,87.32) .. (428,97.99) .. controls (428,108.66) and (419.27,117.31) .. (408.5,117.31) .. controls (397.73,117.31) and (389,108.66) .. (389,97.99) -- cycle ;
%Straight Lines [id:da14241157062841459]
\draw    (428,97.99) -- (470,97.99) ;
\draw [shift={(472,97.99)}, rotate = 180] [color={rgb, 255:red, 0; green, 0; blue, 0 }  ][line width=0.75]    (10.93,-3.29) .. controls (6.95,-1.4) and (3.31,-0.3) .. (0,0) .. controls (3.31,0.3) and (6.95,1.4) .. (10.93,3.29)   ;

% Text Node
\draw (558,141.8) node [anchor=north west][inner sep=0.75pt]   [align=left] {3};
% Text Node
\draw (487,89.49) node [anchor=north west][inner sep=0.75pt]   [align=left] {6};
% Text Node
\draw (276,46.9) node [anchor=north west][inner sep=0.75pt]   [align=left] {Linea de Producción 1};
% Text Node
\draw (403,90.48) node [anchor=north west][inner sep=0.75pt]   [align=left] {4};


\end{tikzpicture}
\\ Figura 62 - Procedencia Estacion 4
\end{center}
Haremos el mismo procedimiento pero ahora analizando la columna de la Estacion numero 3 donde de igual manera vemos que su procedencia viene por línea 1.
\begin{center}


\tikzset{every picture/.style={line width=0.75pt}} %set default line width to 0.75pt

\begin{tikzpicture}[x=0.75pt,y=0.75pt,yscale=-1,xscale=1]
%uncomment if require: \path (0,432); %set diagram left start at 0, and has height of 432

%Shape: Ellipse [id:dp9922104814788146]
\draw   (543,150.3) .. controls (543,139.63) and (551.73,130.98) .. (562.5,130.98) .. controls (573.27,130.98) and (582,139.63) .. (582,150.3) .. controls (582,160.97) and (573.27,169.61) .. (562.5,169.61) .. controls (551.73,169.61) and (543,160.97) .. (543,150.3) -- cycle ;
%Shape: Ellipse [id:dp5423475840783158]
\draw   (472,97.99) .. controls (472,87.32) and (480.73,78.68) .. (491.5,78.68) .. controls (502.27,78.68) and (511,87.32) .. (511,97.99) .. controls (511,108.66) and (502.27,117.31) .. (491.5,117.31) .. controls (480.73,117.31) and (472,108.66) .. (472,97.99) -- cycle ;
%Shape: Rectangle [id:dp9200014617886612]
\draw   (208,68.77) -- (523.5,68.77) -- (523.5,128.2) -- (208,128.2) -- cycle ;
%Straight Lines [id:da18056157949289986]
\draw    (510,110) -- (544.87,134.84) ;
\draw [shift={(546.5,136)}, rotate = 215.46] [color={rgb, 255:red, 0; green, 0; blue, 0 }  ][line width=0.75]    (10.93,-3.29) .. controls (6.95,-1.4) and (3.31,-0.3) .. (0,0) .. controls (3.31,0.3) and (6.95,1.4) .. (10.93,3.29)   ;
%Shape: Ellipse [id:dp6355578187212225]
\draw   (389,97.99) .. controls (389,87.32) and (397.73,78.68) .. (408.5,78.68) .. controls (419.27,78.68) and (428,87.32) .. (428,97.99) .. controls (428,108.66) and (419.27,117.31) .. (408.5,117.31) .. controls (397.73,117.31) and (389,108.66) .. (389,97.99) -- cycle ;
%Straight Lines [id:da5643904343905888]
\draw    (428,97.99) -- (470,97.99) ;
\draw [shift={(472,97.99)}, rotate = 180] [color={rgb, 255:red, 0; green, 0; blue, 0 }  ][line width=0.75]    (10.93,-3.29) .. controls (6.95,-1.4) and (3.31,-0.3) .. (0,0) .. controls (3.31,0.3) and (6.95,1.4) .. (10.93,3.29)   ;
%Shape: Ellipse [id:dp5320769227076223]
\draw   (304,99.97) .. controls (304,89.31) and (312.73,80.66) .. (323.5,80.66) .. controls (334.27,80.66) and (343,89.31) .. (343,99.97) .. controls (343,110.64) and (334.27,119.29) .. (323.5,119.29) .. controls (312.73,119.29) and (304,110.64) .. (304,99.97) -- cycle ;
%Straight Lines [id:da8435351623851575]
\draw    (343,99.97) -- (387,98.08) ;
\draw [shift={(389,97.99)}, rotate = 537.53] [color={rgb, 255:red, 0; green, 0; blue, 0 }  ][line width=0.75]    (10.93,-3.29) .. controls (6.95,-1.4) and (3.31,-0.3) .. (0,0) .. controls (3.31,0.3) and (6.95,1.4) .. (10.93,3.29)   ;

% Text Node
\draw (558,141.8) node [anchor=north west][inner sep=0.75pt]   [align=left] {3};
% Text Node
\draw (487,89.49) node [anchor=north west][inner sep=0.75pt]   [align=left] {6};
% Text Node
\draw (276,46.9) node [anchor=north west][inner sep=0.75pt]   [align=left] {Linea de Producción 1};
% Text Node
\draw (403,90.48) node [anchor=north west][inner sep=0.75pt]   [align=left] {4};
% Text Node
\draw (319,91.47) node [anchor=north west][inner sep=0.75pt]   [align=left] {5};


\end{tikzpicture}

    \\Figura 63 - Procedencia Estación 3
\end{center}

Ya nos encontramos en la Estación 2 asi que de igual forma observamos nuestra tabla 6 y vemos que nuestra estación 2 tiene procedencia por linea 1 quedando de la siguiente manera.
\begin{center}


\tikzset{every picture/.style={line width=0.75pt}} %set default line width to 0.75pt

\begin{tikzpicture}[x=0.75pt,y=0.75pt,yscale=-1,xscale=1]
%uncomment if require: \path (0,432); %set diagram left start at 0, and has height of 432

%Shape: Ellipse [id:dp9132166674990685]
\draw   (523,130.3) .. controls (523,119.63) and (531.73,110.98) .. (542.5,110.98) .. controls (553.27,110.98) and (562,119.63) .. (562,130.3) .. controls (562,140.97) and (553.27,149.61) .. (542.5,149.61) .. controls (531.73,149.61) and (523,140.97) .. (523,130.3) -- cycle ;
%Shape: Ellipse [id:dp6361295141470844]
\draw   (452,77.99) .. controls (452,67.32) and (460.73,58.68) .. (471.5,58.68) .. controls (482.27,58.68) and (491,67.32) .. (491,77.99) .. controls (491,88.66) and (482.27,97.31) .. (471.5,97.31) .. controls (460.73,97.31) and (452,88.66) .. (452,77.99) -- cycle ;
%Shape: Rectangle [id:dp6399454638978244]
\draw   (188,48.77) -- (503.5,48.77) -- (503.5,108.2) -- (188,108.2) -- cycle ;
%Straight Lines [id:da3262709601728562]
\draw    (490,90) -- (524.87,114.84) ;
\draw [shift={(526.5,116)}, rotate = 215.46] [color={rgb, 255:red, 0; green, 0; blue, 0 }  ][line width=0.75]    (10.93,-3.29) .. controls (6.95,-1.4) and (3.31,-0.3) .. (0,0) .. controls (3.31,0.3) and (6.95,1.4) .. (10.93,3.29)   ;
%Shape: Ellipse [id:dp6201215712670158]
\draw   (369,77.99) .. controls (369,67.32) and (377.73,58.68) .. (388.5,58.68) .. controls (399.27,58.68) and (408,67.32) .. (408,77.99) .. controls (408,88.66) and (399.27,97.31) .. (388.5,97.31) .. controls (377.73,97.31) and (369,88.66) .. (369,77.99) -- cycle ;
%Straight Lines [id:da0974685590807658]
\draw    (408,77.99) -- (450,77.99) ;
\draw [shift={(452,77.99)}, rotate = 180] [color={rgb, 255:red, 0; green, 0; blue, 0 }  ][line width=0.75]    (10.93,-3.29) .. controls (6.95,-1.4) and (3.31,-0.3) .. (0,0) .. controls (3.31,0.3) and (6.95,1.4) .. (10.93,3.29)   ;
%Shape: Ellipse [id:dp12245638194181518]
\draw   (284,79.97) .. controls (284,69.31) and (292.73,60.66) .. (303.5,60.66) .. controls (314.27,60.66) and (323,69.31) .. (323,79.97) .. controls (323,90.64) and (314.27,99.29) .. (303.5,99.29) .. controls (292.73,99.29) and (284,90.64) .. (284,79.97) -- cycle ;
%Straight Lines [id:da357458166287415]
\draw    (323,79.97) -- (367,78.08) ;
\draw [shift={(369,77.99)}, rotate = 537.53] [color={rgb, 255:red, 0; green, 0; blue, 0 }  ][line width=0.75]    (10.93,-3.29) .. controls (6.95,-1.4) and (3.31,-0.3) .. (0,0) .. controls (3.31,0.3) and (6.95,1.4) .. (10.93,3.29)   ;
%Shape: Ellipse [id:dp1764818898462046]
\draw   (200,79.97) .. controls (200,69.31) and (208.73,60.66) .. (219.5,60.66) .. controls (230.27,60.66) and (239,69.31) .. (239,79.97) .. controls (239,90.64) and (230.27,99.29) .. (219.5,99.29) .. controls (208.73,99.29) and (200,90.64) .. (200,79.97) -- cycle ;
%Straight Lines [id:da21095509748794727]
\draw    (239,79.97) -- (282,79.97) ;
\draw [shift={(284,79.97)}, rotate = 180] [color={rgb, 255:red, 0; green, 0; blue, 0 }  ][line width=0.75]    (10.93,-3.29) .. controls (6.95,-1.4) and (3.31,-0.3) .. (0,0) .. controls (3.31,0.3) and (6.95,1.4) .. (10.93,3.29)   ;

% Text Node
\draw (538,121.8) node [anchor=north west][inner sep=0.75pt]   [align=left] {3};
% Text Node
\draw (467,69.49) node [anchor=north west][inner sep=0.75pt]   [align=left] {6};
% Text Node
\draw (256,26.9) node [anchor=north west][inner sep=0.75pt]   [align=left] {Linea de Producción 1};
% Text Node
\draw (383,70.48) node [anchor=north west][inner sep=0.75pt]   [align=left] {4};
% Text Node
\draw (299,71.47) node [anchor=north west][inner sep=0.75pt]   [align=left] {5};
% Text Node
\draw (216,71.47) node [anchor=north west][inner sep=0.75pt]   [align=left] {7};


\end{tikzpicture}

    \\ Figura 64 - Procedencia Estación 2
\end{center}

Por ultimo ya posicionados en la estacion 1 solo tenemos una opción de procedencia que es la estación de entrada quedando de la siguiente manera.
\begin{center}


\tikzset{every picture/.style={line width=0.75pt}} %set default line width to 0.75pt

\begin{tikzpicture}[x=0.75pt,y=0.75pt,yscale=-1,xscale=1]
%uncomment if require: \path (0,432); %set diagram left start at 0, and has height of 432

%Shape: Ellipse [id:dp9132166674990685]
\draw   (523,130.3) .. controls (523,119.63) and (531.73,110.98) .. (542.5,110.98) .. controls (553.27,110.98) and (562,119.63) .. (562,130.3) .. controls (562,140.97) and (553.27,149.61) .. (542.5,149.61) .. controls (531.73,149.61) and (523,140.97) .. (523,130.3) -- cycle ;
%Shape: Ellipse [id:dp6361295141470844]
\draw   (452,77.99) .. controls (452,67.32) and (460.73,58.68) .. (471.5,58.68) .. controls (482.27,58.68) and (491,67.32) .. (491,77.99) .. controls (491,88.66) and (482.27,97.31) .. (471.5,97.31) .. controls (460.73,97.31) and (452,88.66) .. (452,77.99) -- cycle ;
%Shape: Rectangle [id:dp6399454638978244]
\draw   (188,48.77) -- (503.5,48.77) -- (503.5,108.2) -- (188,108.2) -- cycle ;
%Straight Lines [id:da3262709601728562]
\draw    (490,90) -- (524.87,114.84) ;
\draw [shift={(526.5,116)}, rotate = 215.46] [color={rgb, 255:red, 0; green, 0; blue, 0 }  ][line width=0.75]    (10.93,-3.29) .. controls (6.95,-1.4) and (3.31,-0.3) .. (0,0) .. controls (3.31,0.3) and (6.95,1.4) .. (10.93,3.29)   ;
%Shape: Ellipse [id:dp6201215712670158]
\draw   (369,77.99) .. controls (369,67.32) and (377.73,58.68) .. (388.5,58.68) .. controls (399.27,58.68) and (408,67.32) .. (408,77.99) .. controls (408,88.66) and (399.27,97.31) .. (388.5,97.31) .. controls (377.73,97.31) and (369,88.66) .. (369,77.99) -- cycle ;
%Straight Lines [id:da0974685590807658]
\draw    (408,77.99) -- (450,77.99) ;
\draw [shift={(452,77.99)}, rotate = 180] [color={rgb, 255:red, 0; green, 0; blue, 0 }  ][line width=0.75]    (10.93,-3.29) .. controls (6.95,-1.4) and (3.31,-0.3) .. (0,0) .. controls (3.31,0.3) and (6.95,1.4) .. (10.93,3.29)   ;
%Shape: Ellipse [id:dp12245638194181518]
\draw   (284,79.97) .. controls (284,69.31) and (292.73,60.66) .. (303.5,60.66) .. controls (314.27,60.66) and (323,69.31) .. (323,79.97) .. controls (323,90.64) and (314.27,99.29) .. (303.5,99.29) .. controls (292.73,99.29) and (284,90.64) .. (284,79.97) -- cycle ;
%Straight Lines [id:da357458166287415]
\draw    (323,79.97) -- (367,78.08) ;
\draw [shift={(369,77.99)}, rotate = 537.53] [color={rgb, 255:red, 0; green, 0; blue, 0 }  ][line width=0.75]    (10.93,-3.29) .. controls (6.95,-1.4) and (3.31,-0.3) .. (0,0) .. controls (3.31,0.3) and (6.95,1.4) .. (10.93,3.29)   ;
%Shape: Ellipse [id:dp1764818898462046]
\draw   (200,79.97) .. controls (200,69.31) and (208.73,60.66) .. (219.5,60.66) .. controls (230.27,60.66) and (239,69.31) .. (239,79.97) .. controls (239,90.64) and (230.27,99.29) .. (219.5,99.29) .. controls (208.73,99.29) and (200,90.64) .. (200,79.97) -- cycle ;
%Straight Lines [id:da21095509748794727]
\draw    (239,79.97) -- (282,79.97) ;
\draw [shift={(284,79.97)}, rotate = 180] [color={rgb, 255:red, 0; green, 0; blue, 0 }  ][line width=0.75]    (10.93,-3.29) .. controls (6.95,-1.4) and (3.31,-0.3) .. (0,0) .. controls (3.31,0.3) and (6.95,1.4) .. (10.93,3.29)   ;
%Shape: Ellipse [id:dp026562153011448375]
\draw   (116,149.31) .. controls (116,138.64) and (124.73,129.99) .. (135.5,129.99) .. controls (146.27,129.99) and (155,138.64) .. (155,149.31) .. controls (155,159.98) and (146.27,168.62) .. (135.5,168.62) .. controls (124.73,168.62) and (116,159.98) .. (116,149.31) -- cycle ;
%Straight Lines [id:da8774272747372955]
\draw    (147,131) -- (198.56,81.36) ;
\draw [shift={(200,79.97)}, rotate = 496.09] [color={rgb, 255:red, 0; green, 0; blue, 0 }  ][line width=0.75]    (10.93,-3.29) .. controls (6.95,-1.4) and (3.31,-0.3) .. (0,0) .. controls (3.31,0.3) and (6.95,1.4) .. (10.93,3.29)   ;

% Text Node
\draw (538,121.8) node [anchor=north west][inner sep=0.75pt]   [align=left] {3};
% Text Node
\draw (467,69.49) node [anchor=north west][inner sep=0.75pt]   [align=left] {6};
% Text Node
\draw (256,26.9) node [anchor=north west][inner sep=0.75pt]   [align=left] {Linea de Producción 1};
% Text Node
\draw (383,70.48) node [anchor=north west][inner sep=0.75pt]   [align=left] {4};
% Text Node
\draw (299,71.47) node [anchor=north west][inner sep=0.75pt]   [align=left] {5};
% Text Node
\draw (216,71.47) node [anchor=north west][inner sep=0.75pt]   [align=left] {7};
% Text Node
\draw (131,140.81) node [anchor=north west][inner sep=0.75pt]   [align=left] {2};


\end{tikzpicture}
    \\ Figura 65 - Procedencia Estacion 1
\end{center}
\\
En la siguiente figura se muestra finalmente como quedó nuestra ruta optima marcada de color rojo la cual tarda menos tiempo en producir un producto que en este caso solo recorre la Linea 1 y es esta la que nos da el tiempo menor.
\begin{center}


\tikzset{every picture/.style={line width=0.75pt}} %set default line width to 0.75pt

\begin{tikzpicture}[x=0.75pt,y=0.75pt,yscale=-1,xscale=1]
%uncomment if require: \path (0,432); %set diagram left start at 0, and has height of 432

%Shape: Rectangle [id:dp3882412922041636]
\draw   (168,28.77) -- (483.5,28.77) -- (483.5,88.2) -- (168,88.2) -- cycle ;
%Shape: Ellipse [id:dp8409179224209302]
\draw  [color={rgb, 255:red, 255; green, 0; blue, 0 }  ,draw opacity=1 ] (180,59.97) .. controls (180,49.31) and (188.73,40.66) .. (199.5,40.66) .. controls (210.27,40.66) and (219,49.31) .. (219,59.97) .. controls (219,70.64) and (210.27,79.29) .. (199.5,79.29) .. controls (188.73,79.29) and (180,70.64) .. (180,59.97) -- cycle ;
%Shape: Ellipse [id:dp6618697065833445]
\draw  [color={rgb, 255:red, 255; green, 0; blue, 0 }  ,draw opacity=1 ] (264,59.97) .. controls (264,49.31) and (272.73,40.66) .. (283.5,40.66) .. controls (294.27,40.66) and (303,49.31) .. (303,59.97) .. controls (303,70.64) and (294.27,79.29) .. (283.5,79.29) .. controls (272.73,79.29) and (264,70.64) .. (264,59.97) -- cycle ;
%Shape: Ellipse [id:dp5721952092295051]
\draw  [color={rgb, 255:red, 255; green, 0; blue, 0 }  ,draw opacity=1 ] (349,57.99) .. controls (349,47.32) and (357.73,38.68) .. (368.5,38.68) .. controls (379.27,38.68) and (388,47.32) .. (388,57.99) .. controls (388,68.66) and (379.27,77.31) .. (368.5,77.31) .. controls (357.73,77.31) and (349,68.66) .. (349,57.99) -- cycle ;
%Shape: Ellipse [id:dp6833723282837854]
\draw  [color={rgb, 255:red, 250; green, 0; blue, 0 }  ,draw opacity=1 ] (432,57.99) .. controls (432,47.32) and (440.73,38.68) .. (451.5,38.68) .. controls (462.27,38.68) and (471,47.32) .. (471,57.99) .. controls (471,68.66) and (462.27,77.31) .. (451.5,77.31) .. controls (440.73,77.31) and (432,68.66) .. (432,57.99) -- cycle ;
%Shape: Ellipse [id:dp8985181219604552]
\draw  [color={rgb, 255:red, 255; green, 0; blue, 0 }  ,draw opacity=1 ] (96,129.31) .. controls (96,118.64) and (104.73,109.99) .. (115.5,109.99) .. controls (126.27,109.99) and (135,118.64) .. (135,129.31) .. controls (135,139.98) and (126.27,148.62) .. (115.5,148.62) .. controls (104.73,148.62) and (96,139.98) .. (96,129.31) -- cycle ;
%Shape: Ellipse [id:dp00930473167103174]
\draw   (97,201.62) .. controls (97,190.95) and (105.73,182.3) .. (116.5,182.3) .. controls (127.27,182.3) and (136,190.95) .. (136,201.62) .. controls (136,212.28) and (127.27,220.93) .. (116.5,220.93) .. controls (105.73,220.93) and (97,212.28) .. (97,201.62) -- cycle ;
%Shape: Ellipse [id:dp5615897075603871]
\draw  [color={rgb, 255:red, 255; green, 0; blue, 0 }  ,draw opacity=1 ] (523,130.3) .. controls (523,119.63) and (531.73,110.98) .. (542.5,110.98) .. controls (553.27,110.98) and (562,119.63) .. (562,130.3) .. controls (562,140.97) and (553.27,149.61) .. (542.5,149.61) .. controls (531.73,149.61) and (523,140.97) .. (523,130.3) -- cycle ;
%Shape: Ellipse [id:dp9065681617862773]
\draw   (525,201.62) .. controls (525,190.95) and (533.73,182.3) .. (544.5,182.3) .. controls (555.27,182.3) and (564,190.95) .. (564,201.62) .. controls (564,212.28) and (555.27,220.93) .. (544.5,220.93) .. controls (533.73,220.93) and (525,212.28) .. (525,201.62) -- cycle ;
%Shape: Rectangle [id:dp909666943337301]
\draw   (168,236.78) -- (484.5,236.78) -- (484.5,296.21) -- (168,296.21) -- cycle ;
%Shape: Ellipse [id:dp9711948519929523]
\draw   (180,267.98) .. controls (180,257.31) and (188.73,248.66) .. (199.5,248.66) .. controls (210.27,248.66) and (219,257.31) .. (219,267.98) .. controls (219,278.65) and (210.27,287.29) .. (199.5,287.29) .. controls (188.73,287.29) and (180,278.65) .. (180,267.98) -- cycle ;
%Shape: Ellipse [id:dp3854952507607323]
\draw   (264,267.98) .. controls (264,257.31) and (272.73,248.66) .. (283.5,248.66) .. controls (294.27,248.66) and (303,257.31) .. (303,267.98) .. controls (303,278.65) and (294.27,287.29) .. (283.5,287.29) .. controls (272.73,287.29) and (264,278.65) .. (264,267.98) -- cycle ;
%Shape: Ellipse [id:dp16404939238349958]
\draw   (349,266.99) .. controls (349,256.32) and (357.73,247.67) .. (368.5,247.67) .. controls (379.27,247.67) and (388,256.32) .. (388,266.99) .. controls (388,277.66) and (379.27,286.3) .. (368.5,286.3) .. controls (357.73,286.3) and (349,277.66) .. (349,266.99) -- cycle ;
%Shape: Ellipse [id:dp7414805922975216]
\draw   (432,266.99) .. controls (432,256.32) and (440.73,247.67) .. (451.5,247.67) .. controls (462.27,247.67) and (471,256.32) .. (471,266.99) .. controls (471,277.66) and (462.27,286.3) .. (451.5,286.3) .. controls (440.73,286.3) and (432,277.66) .. (432,266.99) -- cycle ;
%Shape: Ellipse [id:dp25214142312855037]
\draw   (227,134.01) .. controls (227,126.77) and (232.93,120.89) .. (240.25,120.89) .. controls (247.57,120.89) and (253.5,126.77) .. (253.5,134.01) .. controls (253.5,141.26) and (247.57,147.14) .. (240.25,147.14) .. controls (232.93,147.14) and (227,141.26) .. (227,134.01) -- cycle ;
%Shape: Ellipse [id:dp7893844677118396]
\draw   (227,195.42) .. controls (227,188.18) and (232.93,182.3) .. (240.25,182.3) .. controls (247.57,182.3) and (253.5,188.18) .. (253.5,195.42) .. controls (253.5,202.67) and (247.57,208.55) .. (240.25,208.55) .. controls (232.93,208.55) and (227,202.67) .. (227,195.42) -- cycle ;
%Shape: Ellipse [id:dp926816506125107]
\draw   (313,134.01) .. controls (313,126.77) and (318.93,120.89) .. (326.25,120.89) .. controls (333.57,120.89) and (339.5,126.77) .. (339.5,134.01) .. controls (339.5,141.26) and (333.57,147.14) .. (326.25,147.14) .. controls (318.93,147.14) and (313,141.26) .. (313,134.01) -- cycle ;
%Shape: Ellipse [id:dp8935781498707076]
\draw   (313,195.42) .. controls (313,188.18) and (318.93,182.3) .. (326.25,182.3) .. controls (333.57,182.3) and (339.5,188.18) .. (339.5,195.42) .. controls (339.5,202.67) and (333.57,208.55) .. (326.25,208.55) .. controls (318.93,208.55) and (313,202.67) .. (313,195.42) -- cycle ;
%Shape: Ellipse [id:dp8314486092573847]
\draw   (396,134.01) .. controls (396,126.77) and (401.93,120.89) .. (409.25,120.89) .. controls (416.57,120.89) and (422.5,126.77) .. (422.5,134.01) .. controls (422.5,141.26) and (416.57,147.14) .. (409.25,147.14) .. controls (401.93,147.14) and (396,141.26) .. (396,134.01) -- cycle ;
%Shape: Ellipse [id:dp8639588436591459]
\draw   (397,195.42) .. controls (397,188.18) and (402.93,182.3) .. (410.25,182.3) .. controls (417.57,182.3) and (423.5,188.18) .. (423.5,195.42) .. controls (423.5,202.67) and (417.57,208.55) .. (410.25,208.55) .. controls (402.93,208.55) and (397,202.67) .. (397,195.42) -- cycle ;
%Straight Lines [id:da9324098020632179]
\draw    (223.5,19.86) -- (223.5,315.03) ;
%Straight Lines [id:da507557588103063]
\draw    (308.5,23.86) -- (308.5,319.03) ;
%Straight Lines [id:da03159447721048858]
\draw    (393.5,24.86) -- (393.5,320.03) ;
%Straight Lines [id:da8867835795288663]
\draw [color={rgb, 255:red, 255; green, 0; blue, 0 }  ,draw opacity=1 ]   (128,115) -- (178.63,61.43) ;
\draw [shift={(180,59.97)}, rotate = 493.38] [color={rgb, 255:red, 255; green, 0; blue, 0 }  ,draw opacity=1 ][line width=0.75]    (10.93,-3.29) .. controls (6.95,-1.4) and (3.31,-0.3) .. (0,0) .. controls (3.31,0.3) and (6.95,1.4) .. (10.93,3.29)   ;
%Straight Lines [id:da15783634325075702]
\draw [color={rgb, 255:red, 255; green, 0; blue, 0 }  ,draw opacity=1 ]   (219,59.97) -- (262,59.97) ;
\draw [shift={(264,59.97)}, rotate = 180] [color={rgb, 255:red, 255; green, 0; blue, 0 }  ,draw opacity=1 ][line width=0.75]    (10.93,-3.29) .. controls (6.95,-1.4) and (3.31,-0.3) .. (0,0) .. controls (3.31,0.3) and (6.95,1.4) .. (10.93,3.29)   ;
%Straight Lines [id:da6523795183891536]
\draw [color={rgb, 255:red, 255; green, 0; blue, 0 }  ,draw opacity=1 ]   (303,59.97) -- (347,58.08) ;
\draw [shift={(349,57.99)}, rotate = 537.53] [color={rgb, 255:red, 255; green, 0; blue, 0 }  ,draw opacity=1 ][line width=0.75]    (10.93,-3.29) .. controls (6.95,-1.4) and (3.31,-0.3) .. (0,0) .. controls (3.31,0.3) and (6.95,1.4) .. (10.93,3.29)   ;
%Straight Lines [id:da9758151143753224]
\draw [color={rgb, 255:red, 255; green, 6; blue, 37 }  ,draw opacity=1 ]   (388,57.99) -- (430,57.99) ;
\draw [shift={(432,57.99)}, rotate = 540] [color={rgb, 255:red, 255; green, 6; blue, 37 }  ,draw opacity=1 ][line width=0.75]    (10.93,-3.29) .. controls (6.95,-1.4) and (3.31,-0.3) .. (0,0) .. controls (3.31,0.3) and (6.95,1.4) .. (10.93,3.29)   ;
%Straight Lines [id:da13453787481760893]
\draw [color={rgb, 255:red, 255; green, 0; blue, 0 }  ,draw opacity=1 ]   (471,57.99) -- (529.07,114.6) ;
\draw [shift={(530.5,116)}, rotate = 224.27] [color={rgb, 255:red, 255; green, 0; blue, 0 }  ,draw opacity=1 ][line width=0.75]    (10.93,-3.29) .. controls (6.95,-1.4) and (3.31,-0.3) .. (0,0) .. controls (3.31,0.3) and (6.95,1.4) .. (10.93,3.29)   ;

% Text Node
\draw (196,51.47) node [anchor=north west][inner sep=0.75pt]   [align=left] {7};
% Text Node
\draw (279,51.47) node [anchor=north west][inner sep=0.75pt]   [align=left] {5};
% Text Node
\draw (363,50.48) node [anchor=north west][inner sep=0.75pt]   [align=left] {4};
% Text Node
\draw (447,49.49) node [anchor=north west][inner sep=0.75pt]   [align=left] {6};
% Text Node
\draw (111,120.81) node [anchor=north west][inner sep=0.75pt]   [align=left] {2};
% Text Node
\draw (538,121.8) node [anchor=north west][inner sep=0.75pt]   [align=left] {3};
% Text Node
\draw (540,192.13) node [anchor=north west][inner sep=0.75pt]   [align=left] {4};
% Text Node
\draw (447,258.49) node [anchor=north west][inner sep=0.75pt]   [align=left] {9};
% Text Node
\draw (363,258.49) node [anchor=north west][inner sep=0.75pt]   [align=left] {3};
% Text Node
\draw (278,259.48) node [anchor=north west][inner sep=0.75pt]   [align=left] {8};
% Text Node
\draw (194,259.48) node [anchor=north west][inner sep=0.75pt]   [align=left] {5};
% Text Node
\draw (111,193.12) node [anchor=north west][inner sep=0.75pt]   [align=left] {3};
% Text Node
\draw (235,125.76) node [anchor=north west][inner sep=0.75pt]   [align=left] {2};
% Text Node
\draw (236,187.17) node [anchor=north west][inner sep=0.75pt]   [align=left] {3};
% Text Node
\draw (321,125.76) node [anchor=north west][inner sep=0.75pt]   [align=left] {1};
% Text Node
\draw (321,188.16) node [anchor=north west][inner sep=0.75pt]   [align=left] {2};
% Text Node
\draw (404,125.76) node [anchor=north west][inner sep=0.75pt]   [align=left] {2};
% Text Node
\draw (405,187.17) node [anchor=north west][inner sep=0.75pt]   [align=left] {1};
% Text Node
\draw (236,6.9) node [anchor=north west][inner sep=0.75pt]   [align=left] {Linea de Producción 1};
% Text Node
\draw (212,325.1) node [anchor=north west][inner sep=0.75pt]   [align=left] {Linea de Producción 2};
\end{tikzpicture}
    \\ Figura 66 - Ruta optima
\end{center}
\\
A continuación se muestra la solución final del problema:
\begin{center}
    Linea 1, Estación 4\\
    Linea 1, Estación 3\\
    Linea 1, Estación 2\\
    Linea 1, Estación 1\\
\end{center}


\subsection{Algoritmo de Prim}
Este algoritmo se usa normalmente para ahorrar recursos, su aplicación más común es la implementación de cables de redes, de servidores, de postes de luz entre otros.\\

Sirve para poder hallar el “árbol recubridor mínimo”, en un grafo conexo no dirigido. Es capaz de encontrar un subconjunto de las aristas que formen un árbol que incluya todos los vértices del grafo inicial, donde el peso total de las aristas del árbol es el mínimo posible.\newline
\subsubsection{Definición.}\\\\
Sea V el conjunto de nodos de un grafo pesado no dirigido. El algoritmo de Prim comienza cuando se asigna a un conjunto U de nodos un nodo inicial Perteneciente a V, en el cual “crece” un árbol de expansión, arista por arista. En cada paso se localiza la arista más corta (u, v) que conecta a U con V-U, y después se agrega v, el vértice en V-U, a U. Este paso se repite hasta que V=U. El algoritmo de Prim es de O(N2), donde | V | = N.
\newline

\subsubsection{Funcionamiento.}\\
\begin{itemize}
    \item Se marca un vértice cualquiera. Será el vértice de partida.
    \item Se selecciona la arista de menor peso incidente en el vértice seleccionado anteriormente y se selecciona el otro vértice en el que incide dicha arista.
    \item Repetir el paso 2 siempre que la arista elegida enlace un vértice seleccionado y otro que no lo esté. Es decir, siempre que la arista elegida no cree ningún ciclo.
    \item El árbol de expansión mínima será encontrado cuando hayan sido seleccionados todos los vértices del grafo.
\end{itemize}

\subsubsection{Pseudocodigo Algoritmo de Prim}
El pseudocódigo para el algoritmo de prim muestra cómo creamos dos conjuntos de vértices U y VU. U contiene la lista de vértices que se han visitado y VU la lista de vértices que no. Uno por uno, movemos los vértices del conjunto VU al conjunto U conectando el borde de menor peso.\\
\begin{lstlisting}
1-- T = Conjunto vacio;
2-- U = { 1 };
3-- while (U != V)
4--     let (u, v) Borde de menor costo;
5--     T = T union {(u, v)}
6--     U = U union {v}
\end{lstlisting}
\newpage
\subsubsection{Ejemplo Algoritmo de Prim}
Encuentre un árbol de expansión de costo mínimo para el siguiente gráfico.
\begin{center}


\tikzset{every picture/.style={line width=0.75pt}} %set default line width to 0.75pt

\begin{tikzpicture}[x=0.75pt,y=0.75pt,yscale=-1,xscale=1]
%uncomment if require: \path (0,300); %set diagram left start at 0, and has height of 300

%Shape: Circle [id:dp6766302609341477]
\draw   (158,37.75) .. controls (158,29.05) and (165.05,22) .. (173.75,22) .. controls (182.45,22) and (189.5,29.05) .. (189.5,37.75) .. controls (189.5,46.45) and (182.45,53.5) .. (173.75,53.5) .. controls (165.05,53.5) and (158,46.45) .. (158,37.75) -- cycle ;
%Shape: Circle [id:dp5101690136650403]
\draw   (263,34.75) .. controls (263,26.05) and (270.05,19) .. (278.75,19) .. controls (287.45,19) and (294.5,26.05) .. (294.5,34.75) .. controls (294.5,43.45) and (287.45,50.5) .. (278.75,50.5) .. controls (270.05,50.5) and (263,43.45) .. (263,34.75) -- cycle ;
%Shape: Circle [id:dp10674224043865732]
\draw   (100,112.75) .. controls (100,104.05) and (107.05,97) .. (115.75,97) .. controls (124.45,97) and (131.5,104.05) .. (131.5,112.75) .. controls (131.5,121.45) and (124.45,128.5) .. (115.75,128.5) .. controls (107.05,128.5) and (100,121.45) .. (100,112.75) -- cycle ;
%Shape: Circle [id:dp7038574705264065]
\draw   (190,120.75) .. controls (190,112.05) and (197.05,105) .. (205.75,105) .. controls (214.45,105) and (221.5,112.05) .. (221.5,120.75) .. controls (221.5,129.45) and (214.45,136.5) .. (205.75,136.5) .. controls (197.05,136.5) and (190,129.45) .. (190,120.75) -- cycle ;
%Shape: Circle [id:dp17844965278694036]
\draw   (220,197.75) .. controls (220,189.05) and (227.05,182) .. (235.75,182) .. controls (244.45,182) and (251.5,189.05) .. (251.5,197.75) .. controls (251.5,206.45) and (244.45,213.5) .. (235.75,213.5) .. controls (227.05,213.5) and (220,206.45) .. (220,197.75) -- cycle ;
%Shape: Circle [id:dp808981532374941]
\draw   (287,114.75) .. controls (287,106.05) and (294.05,99) .. (302.75,99) .. controls (311.45,99) and (318.5,106.05) .. (318.5,114.75) .. controls (318.5,123.45) and (311.45,130.5) .. (302.75,130.5) .. controls (294.05,130.5) and (287,123.45) .. (287,114.75) -- cycle ;
%Shape: Circle [id:dp5491354594418891]
\draw   (360,57.75) .. controls (360,49.05) and (367.05,42) .. (375.75,42) .. controls (384.45,42) and (391.5,49.05) .. (391.5,57.75) .. controls (391.5,66.45) and (384.45,73.5) .. (375.75,73.5) .. controls (367.05,73.5) and (360,66.45) .. (360,57.75) -- cycle ;
%Shape: Circle [id:dp45125195270875396]
\draw   (414,120.75) .. controls (414,112.05) and (421.05,105) .. (429.75,105) .. controls (438.45,105) and (445.5,112.05) .. (445.5,120.75) .. controls (445.5,129.45) and (438.45,136.5) .. (429.75,136.5) .. controls (421.05,136.5) and (414,129.45) .. (414,120.75) -- cycle ;
%Shape: Circle [id:dp2680871908250422]
\draw   (349,172.75) .. controls (349,164.05) and (356.05,157) .. (364.75,157) .. controls (373.45,157) and (380.5,164.05) .. (380.5,172.75) .. controls (380.5,181.45) and (373.45,188.5) .. (364.75,188.5) .. controls (356.05,188.5) and (349,181.45) .. (349,172.75) -- cycle ;
%Straight Lines [id:da17366414148796783]
\draw    (123.5,97) -- (158,37.75) ;
%Straight Lines [id:da5985582199236872]
\draw    (115.75,128.5) -- (220,197.75) ;
%Straight Lines [id:da646861596943364]
\draw    (205.75,136.5) -- (235.75,182) ;
%Straight Lines [id:da7051650619293506]
\draw    (173.75,53.5) -- (205.75,105) ;
%Straight Lines [id:da8053951236580943]
\draw    (221.5,120.75) -- (287,114.75) ;
%Straight Lines [id:da16900806969521942]
\draw    (189.5,37.75) -- (263,34.75) ;
%Straight Lines [id:da07646583521054828]
\draw    (294.5,34.75) -- (360,57.75) ;
%Straight Lines [id:da2547460243893085]
\draw    (380.5,172.75) -- (429.75,136.5) ;
%Straight Lines [id:da03741063168893688]
\draw    (375.75,73.5) -- (318.5,114.75) ;
%Straight Lines [id:da516061144644659]
\draw    (302.75,130.5) -- (349,172.75) ;

% Text Node
\draw (110,103) node [anchor=north west][inner sep=0.75pt]   [align=left] {8};
% Text Node
\draw (168,29) node [anchor=north west][inner sep=0.75pt]   [align=left] {0};
% Text Node
\draw (201,112) node [anchor=north west][inner sep=0.75pt]   [align=left] {3};
% Text Node
\draw (231,189) node [anchor=north west][inner sep=0.75pt]   [align=left] {4};
% Text Node
\draw (273,24) node [anchor=north west][inner sep=0.75pt]   [align=left] {1};
% Text Node
\draw (370,48) node [anchor=north west][inner sep=0.75pt]   [align=left] {7};
% Text Node
\draw (297,105) node [anchor=north west][inner sep=0.75pt]   [align=left] {2};
% Text Node
\draw (360,165) node [anchor=north west][inner sep=0.75pt]   [align=left] {5};
% Text Node
\draw (425,113) node [anchor=north west][inner sep=0.75pt]   [align=left] {6};
% Text Node
\draw (395,135) node [anchor=north west][inner sep=0.75pt]   [align=left] {8};
% Text Node
\draw (326,136) node [anchor=north west][inner sep=0.75pt]   [align=left] {1};
% Text Node
\draw (331,78) node [anchor=north west][inner sep=0.75pt]   [align=left] {2};
% Text Node
\draw (324,28) node [anchor=north west][inner sep=0.75pt]   [align=left] {4};
% Text Node
\draw (222,19) node [anchor=north west][inner sep=0.75pt]   [align=left] {3};
% Text Node
\draw (193,68) node [anchor=north west][inner sep=0.75pt]   [align=left] {2};
% Text Node
\draw (123,53) node [anchor=north west][inner sep=0.75pt]   [align=left] {4};
% Text Node
\draw (147,161) node [anchor=north west][inner sep=0.75pt]   [align=left] {8};
% Text Node
\draw (220,145) node [anchor=north west][inner sep=0.75pt]   [align=left] {1};
% Text Node
\draw (250,101) node [anchor=north west][inner sep=0.75pt]   [align=left] {6};


\end{tikzpicture}
    \\ Figura 67 - Grafico 1
\end{center}
\textbf{Desarrollo.}
A continuacion procederemos a resolver el problema presentado. \\
- Paso 1 \\
Escogemos un nodo inicial: nodo 0 y lo marcamos como alcanzado (nodo verde) :
\begin{center}


\tikzset{every picture/.style={line width=0.75pt}} %set default line width to 0.75pt

\begin{tikzpicture}[x=0.75pt,y=0.75pt,yscale=-1,xscale=1]
%uncomment if require: \path (0,300); %set diagram left start at 0, and has height of 300

%Shape: Circle [id:dp2082955410555598]
\draw  [color={rgb, 255:red, 0; green, 255; blue, 24 }  ,draw opacity=1 ] (178,57.75) .. controls (178,49.05) and (185.05,42) .. (193.75,42) .. controls (202.45,42) and (209.5,49.05) .. (209.5,57.75) .. controls (209.5,66.45) and (202.45,73.5) .. (193.75,73.5) .. controls (185.05,73.5) and (178,66.45) .. (178,57.75) -- cycle ;
%Shape: Circle [id:dp9769973542724641]
\draw   (283,54.75) .. controls (283,46.05) and (290.05,39) .. (298.75,39) .. controls (307.45,39) and (314.5,46.05) .. (314.5,54.75) .. controls (314.5,63.45) and (307.45,70.5) .. (298.75,70.5) .. controls (290.05,70.5) and (283,63.45) .. (283,54.75) -- cycle ;
%Shape: Circle [id:dp42917390287144164]
\draw   (120,132.75) .. controls (120,124.05) and (127.05,117) .. (135.75,117) .. controls (144.45,117) and (151.5,124.05) .. (151.5,132.75) .. controls (151.5,141.45) and (144.45,148.5) .. (135.75,148.5) .. controls (127.05,148.5) and (120,141.45) .. (120,132.75) -- cycle ;
%Shape: Circle [id:dp6159057665761554]
\draw   (210,140.75) .. controls (210,132.05) and (217.05,125) .. (225.75,125) .. controls (234.45,125) and (241.5,132.05) .. (241.5,140.75) .. controls (241.5,149.45) and (234.45,156.5) .. (225.75,156.5) .. controls (217.05,156.5) and (210,149.45) .. (210,140.75) -- cycle ;
%Shape: Circle [id:dp22348472907541495]
\draw   (240,217.75) .. controls (240,209.05) and (247.05,202) .. (255.75,202) .. controls (264.45,202) and (271.5,209.05) .. (271.5,217.75) .. controls (271.5,226.45) and (264.45,233.5) .. (255.75,233.5) .. controls (247.05,233.5) and (240,226.45) .. (240,217.75) -- cycle ;
%Shape: Circle [id:dp7060340718103353]
\draw   (307,134.75) .. controls (307,126.05) and (314.05,119) .. (322.75,119) .. controls (331.45,119) and (338.5,126.05) .. (338.5,134.75) .. controls (338.5,143.45) and (331.45,150.5) .. (322.75,150.5) .. controls (314.05,150.5) and (307,143.45) .. (307,134.75) -- cycle ;
%Shape: Circle [id:dp6603614261945912]
\draw   (380,77.75) .. controls (380,69.05) and (387.05,62) .. (395.75,62) .. controls (404.45,62) and (411.5,69.05) .. (411.5,77.75) .. controls (411.5,86.45) and (404.45,93.5) .. (395.75,93.5) .. controls (387.05,93.5) and (380,86.45) .. (380,77.75) -- cycle ;
%Shape: Circle [id:dp3211777580644224]
\draw   (434,140.75) .. controls (434,132.05) and (441.05,125) .. (449.75,125) .. controls (458.45,125) and (465.5,132.05) .. (465.5,140.75) .. controls (465.5,149.45) and (458.45,156.5) .. (449.75,156.5) .. controls (441.05,156.5) and (434,149.45) .. (434,140.75) -- cycle ;
%Shape: Circle [id:dp6117446905390662]
\draw   (369,192.75) .. controls (369,184.05) and (376.05,177) .. (384.75,177) .. controls (393.45,177) and (400.5,184.05) .. (400.5,192.75) .. controls (400.5,201.45) and (393.45,208.5) .. (384.75,208.5) .. controls (376.05,208.5) and (369,201.45) .. (369,192.75) -- cycle ;
%Straight Lines [id:da5478157115253]
\draw    (143.5,117) -- (178,57.75) ;
%Straight Lines [id:da23537196819907802]
\draw    (135.75,148.5) -- (240,217.75) ;
%Straight Lines [id:da9608719672671411]
\draw    (225.75,156.5) -- (255.75,202) ;
%Straight Lines [id:da7708180010937593]
\draw    (193.75,73.5) -- (225.75,125) ;
%Straight Lines [id:da6352081979620849]
\draw    (241.5,140.75) -- (307,134.75) ;
%Straight Lines [id:da7706881326207031]
\draw    (209.5,57.75) -- (283,54.75) ;
%Straight Lines [id:da9532924386951052]
\draw    (314.5,54.75) -- (380,77.75) ;
%Straight Lines [id:da6736372249184637]
\draw    (400.5,192.75) -- (449.75,156.5) ;
%Straight Lines [id:da3770610359474238]
\draw    (395.75,93.5) -- (338.5,134.75) ;
%Straight Lines [id:da5936178392792015]
\draw    (322.75,150.5) -- (369,192.75) ;

% Text Node
\draw (130,123) node [anchor=north west][inner sep=0.75pt]   [align=left] {8};
% Text Node
\draw (188,49) node [anchor=north west][inner sep=0.75pt]   [align=left] {0};
% Text Node
\draw (221,132) node [anchor=north west][inner sep=0.75pt]   [align=left] {3};
% Text Node
\draw (251,209) node [anchor=north west][inner sep=0.75pt]   [align=left] {4};
% Text Node
\draw (293,44) node [anchor=north west][inner sep=0.75pt]   [align=left] {1};
% Text Node
\draw (390,68) node [anchor=north west][inner sep=0.75pt]   [align=left] {7};
% Text Node
\draw (317,125) node [anchor=north west][inner sep=0.75pt]   [align=left] {2};
% Text Node
\draw (380,185) node [anchor=north west][inner sep=0.75pt]   [align=left] {5};
% Text Node
\draw (445,133) node [anchor=north west][inner sep=0.75pt]   [align=left] {6};
% Text Node
\draw (415,155) node [anchor=north west][inner sep=0.75pt]   [align=left] {8};
% Text Node
\draw (346,156) node [anchor=north west][inner sep=0.75pt]   [align=left] {1};
% Text Node
\draw (351,98) node [anchor=north west][inner sep=0.75pt]   [align=left] {2};
% Text Node
\draw (344,48) node [anchor=north west][inner sep=0.75pt]   [align=left] {4};
% Text Node
\draw (242,39) node [anchor=north west][inner sep=0.75pt]   [align=left] {3};
% Text Node
\draw (213,88) node [anchor=north west][inner sep=0.75pt]   [align=left] {2};
% Text Node
\draw (143,73) node [anchor=north west][inner sep=0.75pt]   [align=left] {4};
% Text Node
\draw (167,181) node [anchor=north west][inner sep=0.75pt]   [align=left] {8};
% Text Node
\draw (240,165) node [anchor=north west][inner sep=0.75pt]   [align=left] {1};
% Text Node
\draw (270,121) node [anchor=north west][inner sep=0.75pt]   [align=left] {6};
\end{tikzpicture}
 \\ Figura 68 - Grafico nodo inicial
\end{center}
Todos los demás nodos están marcados como no alcanzados. \\
- Paso 2 \\
Encuentre una ventaja con un costo mínimo que conecte un nodo alcanzado a un nodo no alcanzado. Esta ventaja es: (0, 3 ) . \\
\\Agregue el borde (0,3) al MST y marque el nodo 3 como alcanzado
\begin{center}
    \tikzset{every picture/.style={line width=0.75pt}} %set default line width to 0.75pt

\begin{tikzpicture}[x=0.75pt,y=0.75pt,yscale=-1,xscale=1]
%uncomment if require: \path (0,300); %set diagram left start at 0, and has height of 300

%Shape: Circle [id:dp2082955410555598]
\draw  [color={rgb, 255:red, 0; green, 255; blue, 24 }  ,draw opacity=1 ] (178,57.75) .. controls (178,49.05) and (185.05,42) .. (193.75,42) .. controls (202.45,42) and (209.5,49.05) .. (209.5,57.75) .. controls (209.5,66.45) and (202.45,73.5) .. (193.75,73.5) .. controls (185.05,73.5) and (178,66.45) .. (178,57.75) -- cycle ;
%Shape: Circle [id:dp9769973542724641]
\draw   (283,54.75) .. controls (283,46.05) and (290.05,39) .. (298.75,39) .. controls (307.45,39) and (314.5,46.05) .. (314.5,54.75) .. controls (314.5,63.45) and (307.45,70.5) .. (298.75,70.5) .. controls (290.05,70.5) and (283,63.45) .. (283,54.75) -- cycle ;
%Shape: Circle [id:dp42917390287144164]
\draw   (120,132.75) .. controls (120,124.05) and (127.05,117) .. (135.75,117) .. controls (144.45,117) and (151.5,124.05) .. (151.5,132.75) .. controls (151.5,141.45) and (144.45,148.5) .. (135.75,148.5) .. controls (127.05,148.5) and (120,141.45) .. (120,132.75) -- cycle ;
%Shape: Circle [id:dp6159057665761554]
\draw  [color={rgb, 255:red, 21; green, 255; blue, 0 }  ,draw opacity=1 ] (210,140.75) .. controls (210,132.05) and (217.05,125) .. (225.75,125) .. controls (234.45,125) and (241.5,132.05) .. (241.5,140.75) .. controls (241.5,149.45) and (234.45,156.5) .. (225.75,156.5) .. controls (217.05,156.5) and (210,149.45) .. (210,140.75) -- cycle ;
%Shape: Circle [id:dp22348472907541495]
\draw   (240,217.75) .. controls (240,209.05) and (247.05,202) .. (255.75,202) .. controls (264.45,202) and (271.5,209.05) .. (271.5,217.75) .. controls (271.5,226.45) and (264.45,233.5) .. (255.75,233.5) .. controls (247.05,233.5) and (240,226.45) .. (240,217.75) -- cycle ;
%Shape: Circle [id:dp7060340718103353]
\draw   (307,134.75) .. controls (307,126.05) and (314.05,119) .. (322.75,119) .. controls (331.45,119) and (338.5,126.05) .. (338.5,134.75) .. controls (338.5,143.45) and (331.45,150.5) .. (322.75,150.5) .. controls (314.05,150.5) and (307,143.45) .. (307,134.75) -- cycle ;
%Shape: Circle [id:dp6603614261945912]
\draw   (380,77.75) .. controls (380,69.05) and (387.05,62) .. (395.75,62) .. controls (404.45,62) and (411.5,69.05) .. (411.5,77.75) .. controls (411.5,86.45) and (404.45,93.5) .. (395.75,93.5) .. controls (387.05,93.5) and (380,86.45) .. (380,77.75) -- cycle ;
%Shape: Circle [id:dp3211777580644224]
\draw   (434,140.75) .. controls (434,132.05) and (441.05,125) .. (449.75,125) .. controls (458.45,125) and (465.5,132.05) .. (465.5,140.75) .. controls (465.5,149.45) and (458.45,156.5) .. (449.75,156.5) .. controls (441.05,156.5) and (434,149.45) .. (434,140.75) -- cycle ;
%Shape: Circle [id:dp6117446905390662]
\draw   (369,192.75) .. controls (369,184.05) and (376.05,177) .. (384.75,177) .. controls (393.45,177) and (400.5,184.05) .. (400.5,192.75) .. controls (400.5,201.45) and (393.45,208.5) .. (384.75,208.5) .. controls (376.05,208.5) and (369,201.45) .. (369,192.75) -- cycle ;
%Straight Lines [id:da5478157115253]
\draw    (143.5,117) -- (178,57.75) ;
%Straight Lines [id:da23537196819907802]
\draw    (135.75,148.5) -- (240,217.75) ;
%Straight Lines [id:da9608719672671411]
\draw    (225.75,156.5) -- (255.75,202) ;
%Straight Lines [id:da7708180010937593]
\draw [color={rgb, 255:red, 0; green, 77; blue, 255 }  ,draw opacity=1 ]   (193.75,73.5) -- (225.75,125) ;
%Straight Lines [id:da6352081979620849]
\draw    (241.5,140.75) -- (307,134.75) ;
%Straight Lines [id:da7706881326207031]
\draw    (209.5,57.75) -- (283,54.75) ;
%Straight Lines [id:da9532924386951052]
\draw    (314.5,54.75) -- (380,77.75) ;
%Straight Lines [id:da6736372249184637]
\draw    (400.5,192.75) -- (449.75,156.5) ;
%Straight Lines [id:da3770610359474238]
\draw    (395.75,93.5) -- (338.5,134.75) ;
%Straight Lines [id:da5936178392792015]
\draw    (322.75,150.5) -- (369,192.75) ;

% Text Node
\draw (130,123) node [anchor=north west][inner sep=0.75pt]   [align=left] {8};
% Text Node
\draw (188,49) node [anchor=north west][inner sep=0.75pt]   [align=left] {0};
% Text Node
\draw (221,132) node [anchor=north west][inner sep=0.75pt]   [align=left] {3};
% Text Node
\draw (251,209) node [anchor=north west][inner sep=0.75pt]   [align=left] {4};
% Text Node
\draw (293,44) node [anchor=north west][inner sep=0.75pt]   [align=left] {1};
% Text Node
\draw (390,68) node [anchor=north west][inner sep=0.75pt]   [align=left] {7};
% Text Node
\draw (317,125) node [anchor=north west][inner sep=0.75pt]   [align=left] {2};
% Text Node
\draw (380,185) node [anchor=north west][inner sep=0.75pt]   [align=left] {5};
% Text Node
\draw (445,133) node [anchor=north west][inner sep=0.75pt]   [align=left] {6};
% Text Node
\draw (415,155) node [anchor=north west][inner sep=0.75pt]   [align=left] {8};
% Text Node
\draw (346,156) node [anchor=north west][inner sep=0.75pt]   [align=left] {1};
% Text Node
\draw (351,98) node [anchor=north west][inner sep=0.75pt]   [align=left] {2};
% Text Node
\draw (344,48) node [anchor=north west][inner sep=0.75pt]   [align=left] {4};
% Text Node
\draw (242,39) node [anchor=north west][inner sep=0.75pt]   [align=left] {3};
% Text Node
\draw (213,88) node [anchor=north west][inner sep=0.75pt]   [align=left] {2};
% Text Node
\draw (143,73) node [anchor=north west][inner sep=0.75pt]   [align=left] {4};
% Text Node
\draw (167,181) node [anchor=north west][inner sep=0.75pt]   [align=left] {8};
% Text Node
\draw (240,165) node [anchor=north west][inner sep=0.75pt]   [align=left] {1};
% Text Node
\draw (270,121) node [anchor=north west][inner sep=0.75pt]   [align=left] {6};


\end{tikzpicture}
    \\ Figura 69 - Grafico unión nodos (0,3)


\end{center}

- Paso 3 \\
Encuentre una ventaja con un costo mínimo que conecte un nodo alcanzado a un nodo no alcanzado. Esta ventaja es: (3, 4 ).  \\
Agregue el borde (3,4) al MST y marque el nodo 4 como alcanzado :

\begin{center}
    \tikzset{every picture/.style={line width=0.75pt}} %set default line width to 0.75pt

\begin{tikzpicture}[x=0.75pt,y=0.75pt,yscale=-1,xscale=1]
%uncomment if require: \path (0,300); %set diagram left start at 0, and has height of 300

%Shape: Circle [id:dp2082955410555598]
\draw  [color={rgb, 255:red, 0; green, 255; blue, 24 }  ,draw opacity=1 ] (178,57.75) .. controls (178,49.05) and (185.05,42) .. (193.75,42) .. controls (202.45,42) and (209.5,49.05) .. (209.5,57.75) .. controls (209.5,66.45) and (202.45,73.5) .. (193.75,73.5) .. controls (185.05,73.5) and (178,66.45) .. (178,57.75) -- cycle ;
%Shape: Circle [id:dp9769973542724641]
\draw   (283,54.75) .. controls (283,46.05) and (290.05,39) .. (298.75,39) .. controls (307.45,39) and (314.5,46.05) .. (314.5,54.75) .. controls (314.5,63.45) and (307.45,70.5) .. (298.75,70.5) .. controls (290.05,70.5) and (283,63.45) .. (283,54.75) -- cycle ;
%Shape: Circle [id:dp42917390287144164]
\draw   (120,132.75) .. controls (120,124.05) and (127.05,117) .. (135.75,117) .. controls (144.45,117) and (151.5,124.05) .. (151.5,132.75) .. controls (151.5,141.45) and (144.45,148.5) .. (135.75,148.5) .. controls (127.05,148.5) and (120,141.45) .. (120,132.75) -- cycle ;
%Shape: Circle [id:dp6159057665761554]
\draw  [color={rgb, 255:red, 21; green, 255; blue, 0 }  ,draw opacity=1 ] (210,140.75) .. controls (210,132.05) and (217.05,125) .. (225.75,125) .. controls (234.45,125) and (241.5,132.05) .. (241.5,140.75) .. controls (241.5,149.45) and (234.45,156.5) .. (225.75,156.5) .. controls (217.05,156.5) and (210,149.45) .. (210,140.75) -- cycle ;
%Shape: Circle [id:dp22348472907541495]
\draw  [color={rgb, 255:red, 0; green, 255; blue, 0 }  ,draw opacity=1 ] (240,217.75) .. controls (240,209.05) and (247.05,202) .. (255.75,202) .. controls (264.45,202) and (271.5,209.05) .. (271.5,217.75) .. controls (271.5,226.45) and (264.45,233.5) .. (255.75,233.5) .. controls (247.05,233.5) and (240,226.45) .. (240,217.75) -- cycle ;
%Shape: Circle [id:dp7060340718103353]
\draw   (307,134.75) .. controls (307,126.05) and (314.05,119) .. (322.75,119) .. controls (331.45,119) and (338.5,126.05) .. (338.5,134.75) .. controls (338.5,143.45) and (331.45,150.5) .. (322.75,150.5) .. controls (314.05,150.5) and (307,143.45) .. (307,134.75) -- cycle ;
%Shape: Circle [id:dp6603614261945912]
\draw   (380,77.75) .. controls (380,69.05) and (387.05,62) .. (395.75,62) .. controls (404.45,62) and (411.5,69.05) .. (411.5,77.75) .. controls (411.5,86.45) and (404.45,93.5) .. (395.75,93.5) .. controls (387.05,93.5) and (380,86.45) .. (380,77.75) -- cycle ;
%Shape: Circle [id:dp3211777580644224]
\draw   (434,140.75) .. controls (434,132.05) and (441.05,125) .. (449.75,125) .. controls (458.45,125) and (465.5,132.05) .. (465.5,140.75) .. controls (465.5,149.45) and (458.45,156.5) .. (449.75,156.5) .. controls (441.05,156.5) and (434,149.45) .. (434,140.75) -- cycle ;
%Shape: Circle [id:dp6117446905390662]
\draw   (369,192.75) .. controls (369,184.05) and (376.05,177) .. (384.75,177) .. controls (393.45,177) and (400.5,184.05) .. (400.5,192.75) .. controls (400.5,201.45) and (393.45,208.5) .. (384.75,208.5) .. controls (376.05,208.5) and (369,201.45) .. (369,192.75) -- cycle ;
%Straight Lines [id:da5478157115253]
\draw    (143.5,117) -- (178,57.75) ;
%Straight Lines [id:da23537196819907802]
\draw    (135.75,148.5) -- (240,217.75) ;
%Straight Lines [id:da9608719672671411]
\draw [color={rgb, 255:red, 0; green, 77; blue, 255 }  ,draw opacity=1 ]   (225.75,156.5) -- (255.75,202) ;
%Straight Lines [id:da7708180010937593]
\draw [color={rgb, 255:red, 0; green, 77; blue, 255 }  ,draw opacity=1 ]   (193.75,73.5) -- (225.75,125) ;
%Straight Lines [id:da6352081979620849]
\draw    (241.5,140.75) -- (307,134.75) ;
%Straight Lines [id:da7706881326207031]
\draw    (209.5,57.75) -- (283,54.75) ;
%Straight Lines [id:da9532924386951052]
\draw    (314.5,54.75) -- (380,77.75) ;
%Straight Lines [id:da6736372249184637]
\draw    (400.5,192.75) -- (449.75,156.5) ;
%Straight Lines [id:da3770610359474238]
\draw    (395.75,93.5) -- (338.5,134.75) ;
%Straight Lines [id:da5936178392792015]
\draw    (322.75,150.5) -- (369,192.75) ;

% Text Node
\draw (130,123) node [anchor=north west][inner sep=0.75pt]   [align=left] {8};
% Text Node
\draw (188,49) node [anchor=north west][inner sep=0.75pt]   [align=left] {0};
% Text Node
\draw (221,132) node [anchor=north west][inner sep=0.75pt]   [align=left] {3};
% Text Node
\draw (251,209) node [anchor=north west][inner sep=0.75pt]   [align=left] {4};
% Text Node
\draw (293,44) node [anchor=north west][inner sep=0.75pt]   [align=left] {1};
% Text Node
\draw (390,68) node [anchor=north west][inner sep=0.75pt]   [align=left] {7};
% Text Node
\draw (317,125) node [anchor=north west][inner sep=0.75pt]   [align=left] {2};
% Text Node
\draw (380,185) node [anchor=north west][inner sep=0.75pt]   [align=left] {5};
% Text Node
\draw (445,133) node [anchor=north west][inner sep=0.75pt]   [align=left] {6};
% Text Node
\draw (415,155) node [anchor=north west][inner sep=0.75pt]   [align=left] {8};
% Text Node
\draw (346,156) node [anchor=north west][inner sep=0.75pt]   [align=left] {1};
% Text Node
\draw (351,98) node [anchor=north west][inner sep=0.75pt]   [align=left] {2};
% Text Node
\draw (344,48) node [anchor=north west][inner sep=0.75pt]   [align=left] {4};
% Text Node
\draw (242,39) node [anchor=north west][inner sep=0.75pt]   [align=left] {3};
% Text Node
\draw (213,88) node [anchor=north west][inner sep=0.75pt]   [align=left] {2};
% Text Node
\draw (143,73) node [anchor=north west][inner sep=0.75pt]   [align=left] {4};
% Text Node
\draw (167,181) node [anchor=north west][inner sep=0.75pt]   [align=left] {8};
% Text Node
\draw (240,165) node [anchor=north west][inner sep=0.75pt]   [align=left] {1};
% Text Node
\draw (270,121) node [anchor=north west][inner sep=0.75pt]   [align=left] {6};


\end{tikzpicture}

    \\ Figura 70 - Grafico unión nodo (3,4)


\end{center}
- Paso 4 \\
Encuentre una ventaja con un costo mínimo que conecte un nodo alcanzado a un nodo no alcanzado. Esta ventaja es: (0, 1 ).\\
Agregue el borde (0,1) al MST y marque el nodo 1 como alcanzado.
\begin{center}


\tikzset{every picture/.style={line width=0.75pt}} %set default line width to 0.75pt

\begin{tikzpicture}[x=0.75pt,y=0.75pt,yscale=-1,xscale=1]
%uncomment if require: \path (0,300); %set diagram left start at 0, and has height of 300

%Shape: Circle [id:dp2082955410555598]
\draw  [color={rgb, 255:red, 0; green, 255; blue, 24 }  ,draw opacity=1 ] (178,57.75) .. controls (178,49.05) and (185.05,42) .. (193.75,42) .. controls (202.45,42) and (209.5,49.05) .. (209.5,57.75) .. controls (209.5,66.45) and (202.45,73.5) .. (193.75,73.5) .. controls (185.05,73.5) and (178,66.45) .. (178,57.75) -- cycle ;
%Shape: Circle [id:dp9769973542724641]
\draw  [color={rgb, 255:red, 0; green, 255; blue, 0 }  ,draw opacity=1 ] (283,54.75) .. controls (283,46.05) and (290.05,39) .. (298.75,39) .. controls (307.45,39) and (314.5,46.05) .. (314.5,54.75) .. controls (314.5,63.45) and (307.45,70.5) .. (298.75,70.5) .. controls (290.05,70.5) and (283,63.45) .. (283,54.75) -- cycle ;
%Shape: Circle [id:dp42917390287144164]
\draw   (120,132.75) .. controls (120,124.05) and (127.05,117) .. (135.75,117) .. controls (144.45,117) and (151.5,124.05) .. (151.5,132.75) .. controls (151.5,141.45) and (144.45,148.5) .. (135.75,148.5) .. controls (127.05,148.5) and (120,141.45) .. (120,132.75) -- cycle ;
%Shape: Circle [id:dp6159057665761554]
\draw  [color={rgb, 255:red, 21; green, 255; blue, 0 }  ,draw opacity=1 ] (210,140.75) .. controls (210,132.05) and (217.05,125) .. (225.75,125) .. controls (234.45,125) and (241.5,132.05) .. (241.5,140.75) .. controls (241.5,149.45) and (234.45,156.5) .. (225.75,156.5) .. controls (217.05,156.5) and (210,149.45) .. (210,140.75) -- cycle ;
%Shape: Circle [id:dp22348472907541495]
\draw  [color={rgb, 255:red, 0; green, 255; blue, 0 }  ,draw opacity=1 ] (240,217.75) .. controls (240,209.05) and (247.05,202) .. (255.75,202) .. controls (264.45,202) and (271.5,209.05) .. (271.5,217.75) .. controls (271.5,226.45) and (264.45,233.5) .. (255.75,233.5) .. controls (247.05,233.5) and (240,226.45) .. (240,217.75) -- cycle ;
%Shape: Circle [id:dp7060340718103353]
\draw   (307,134.75) .. controls (307,126.05) and (314.05,119) .. (322.75,119) .. controls (331.45,119) and (338.5,126.05) .. (338.5,134.75) .. controls (338.5,143.45) and (331.45,150.5) .. (322.75,150.5) .. controls (314.05,150.5) and (307,143.45) .. (307,134.75) -- cycle ;
%Shape: Circle [id:dp6603614261945912]
\draw   (380,77.75) .. controls (380,69.05) and (387.05,62) .. (395.75,62) .. controls (404.45,62) and (411.5,69.05) .. (411.5,77.75) .. controls (411.5,86.45) and (404.45,93.5) .. (395.75,93.5) .. controls (387.05,93.5) and (380,86.45) .. (380,77.75) -- cycle ;
%Shape: Circle [id:dp3211777580644224]
\draw   (434,140.75) .. controls (434,132.05) and (441.05,125) .. (449.75,125) .. controls (458.45,125) and (465.5,132.05) .. (465.5,140.75) .. controls (465.5,149.45) and (458.45,156.5) .. (449.75,156.5) .. controls (441.05,156.5) and (434,149.45) .. (434,140.75) -- cycle ;
%Shape: Circle [id:dp6117446905390662]
\draw   (369,192.75) .. controls (369,184.05) and (376.05,177) .. (384.75,177) .. controls (393.45,177) and (400.5,184.05) .. (400.5,192.75) .. controls (400.5,201.45) and (393.45,208.5) .. (384.75,208.5) .. controls (376.05,208.5) and (369,201.45) .. (369,192.75) -- cycle ;
%Straight Lines [id:da5478157115253]
\draw    (143.5,117) -- (178,57.75) ;
%Straight Lines [id:da23537196819907802]
\draw    (135.75,148.5) -- (240,217.75) ;
%Straight Lines [id:da9608719672671411]
\draw [color={rgb, 255:red, 0; green, 77; blue, 255 }  ,draw opacity=1 ]   (225.75,156.5) -- (255.75,202) ;
%Straight Lines [id:da7708180010937593]
\draw [color={rgb, 255:red, 0; green, 77; blue, 255 }  ,draw opacity=1 ]   (193.75,73.5) -- (225.75,125) ;
%Straight Lines [id:da6352081979620849]
\draw    (241.5,140.75) -- (307,134.75) ;
%Straight Lines [id:da7706881326207031]
\draw [color={rgb, 255:red, 0; green, 77; blue, 255 }  ,draw opacity=1 ]   (209.5,57.75) -- (283,54.75) ;
%Straight Lines [id:da9532924386951052]
\draw    (314.5,54.75) -- (380,77.75) ;
%Straight Lines [id:da6736372249184637]
\draw    (400.5,192.75) -- (449.75,156.5) ;
%Straight Lines [id:da3770610359474238]
\draw    (395.75,93.5) -- (338.5,134.75) ;
%Straight Lines [id:da5936178392792015]
\draw    (322.75,150.5) -- (369,192.75) ;

% Text Node
\draw (130,123) node [anchor=north west][inner sep=0.75pt]   [align=left] {8};
% Text Node
\draw (188,49) node [anchor=north west][inner sep=0.75pt]   [align=left] {0};
% Text Node
\draw (221,132) node [anchor=north west][inner sep=0.75pt]   [align=left] {3};
% Text Node
\draw (251,209) node [anchor=north west][inner sep=0.75pt]   [align=left] {4};
% Text Node
\draw (293,44) node [anchor=north west][inner sep=0.75pt]   [align=left] {1};
% Text Node
\draw (390,68) node [anchor=north west][inner sep=0.75pt]   [align=left] {7};
% Text Node
\draw (317,125) node [anchor=north west][inner sep=0.75pt]   [align=left] {2};
% Text Node
\draw (380,185) node [anchor=north west][inner sep=0.75pt]   [align=left] {5};
% Text Node
\draw (445,133) node [anchor=north west][inner sep=0.75pt]   [align=left] {6};
% Text Node
\draw (415,155) node [anchor=north west][inner sep=0.75pt]   [align=left] {8};
% Text Node
\draw (346,156) node [anchor=north west][inner sep=0.75pt]   [align=left] {1};
% Text Node
\draw (351,98) node [anchor=north west][inner sep=0.75pt]   [align=left] {2};
% Text Node
\draw (344,48) node [anchor=north west][inner sep=0.75pt]   [align=left] {4};
% Text Node
\draw (242,39) node [anchor=north west][inner sep=0.75pt]   [align=left] {3};
% Text Node
\draw (213,88) node [anchor=north west][inner sep=0.75pt]   [align=left] {2};
% Text Node
\draw (143,73) node [anchor=north west][inner sep=0.75pt]   [align=left] {4};
% Text Node
\draw (167,181) node [anchor=north west][inner sep=0.75pt]   [align=left] {8};
% Text Node
\draw (240,165) node [anchor=north west][inner sep=0.75pt]   [align=left] {1};
% Text Node
\draw (270,121) node [anchor=north west][inner sep=0.75pt]   [align=left] {6};


\end{tikzpicture}

    \\ Figura 71 - Grafico unión nodos (0,1)
\end{center}
\\
- Paso 5\\
Aceleraremos un poco el proceso así que. El siguiente borde agregado es: (1,7).
\begin{center}

\tikzset{every picture/.style={line width=0.75pt}} %set default line width to 0.75pt

\begin{tikzpicture}[x=0.75pt,y=0.75pt,yscale=-1,xscale=1]
%uncomment if require: \path (0,300); %set diagram left start at 0, and has height of 300

%Shape: Circle [id:dp2082955410555598]
\draw  [color={rgb, 255:red, 0; green, 255; blue, 24 }  ,draw opacity=1 ] (178,57.75) .. controls (178,49.05) and (185.05,42) .. (193.75,42) .. controls (202.45,42) and (209.5,49.05) .. (209.5,57.75) .. controls (209.5,66.45) and (202.45,73.5) .. (193.75,73.5) .. controls (185.05,73.5) and (178,66.45) .. (178,57.75) -- cycle ;
%Shape: Circle [id:dp9769973542724641]
\draw  [color={rgb, 255:red, 0; green, 255; blue, 0 }  ,draw opacity=1 ] (283,54.75) .. controls (283,46.05) and (290.05,39) .. (298.75,39) .. controls (307.45,39) and (314.5,46.05) .. (314.5,54.75) .. controls (314.5,63.45) and (307.45,70.5) .. (298.75,70.5) .. controls (290.05,70.5) and (283,63.45) .. (283,54.75) -- cycle ;
%Shape: Circle [id:dp42917390287144164]
\draw   (120,132.75) .. controls (120,124.05) and (127.05,117) .. (135.75,117) .. controls (144.45,117) and (151.5,124.05) .. (151.5,132.75) .. controls (151.5,141.45) and (144.45,148.5) .. (135.75,148.5) .. controls (127.05,148.5) and (120,141.45) .. (120,132.75) -- cycle ;
%Shape: Circle [id:dp6159057665761554]
\draw  [color={rgb, 255:red, 21; green, 255; blue, 0 }  ,draw opacity=1 ] (210,140.75) .. controls (210,132.05) and (217.05,125) .. (225.75,125) .. controls (234.45,125) and (241.5,132.05) .. (241.5,140.75) .. controls (241.5,149.45) and (234.45,156.5) .. (225.75,156.5) .. controls (217.05,156.5) and (210,149.45) .. (210,140.75) -- cycle ;
%Shape: Circle [id:dp22348472907541495]
\draw  [color={rgb, 255:red, 0; green, 255; blue, 0 }  ,draw opacity=1 ] (240,217.75) .. controls (240,209.05) and (247.05,202) .. (255.75,202) .. controls (264.45,202) and (271.5,209.05) .. (271.5,217.75) .. controls (271.5,226.45) and (264.45,233.5) .. (255.75,233.5) .. controls (247.05,233.5) and (240,226.45) .. (240,217.75) -- cycle ;
%Shape: Circle [id:dp7060340718103353]
\draw   (307,134.75) .. controls (307,126.05) and (314.05,119) .. (322.75,119) .. controls (331.45,119) and (338.5,126.05) .. (338.5,134.75) .. controls (338.5,143.45) and (331.45,150.5) .. (322.75,150.5) .. controls (314.05,150.5) and (307,143.45) .. (307,134.75) -- cycle ;
%Shape: Circle [id:dp6603614261945912]
\draw  [color={rgb, 255:red, 0; green, 255; blue, 0 }  ,draw opacity=1 ] (380,77.75) .. controls (380,69.05) and (387.05,62) .. (395.75,62) .. controls (404.45,62) and (411.5,69.05) .. (411.5,77.75) .. controls (411.5,86.45) and (404.45,93.5) .. (395.75,93.5) .. controls (387.05,93.5) and (380,86.45) .. (380,77.75) -- cycle ;
%Shape: Circle [id:dp3211777580644224]
\draw   (434,140.75) .. controls (434,132.05) and (441.05,125) .. (449.75,125) .. controls (458.45,125) and (465.5,132.05) .. (465.5,140.75) .. controls (465.5,149.45) and (458.45,156.5) .. (449.75,156.5) .. controls (441.05,156.5) and (434,149.45) .. (434,140.75) -- cycle ;
%Shape: Circle [id:dp6117446905390662]
\draw   (369,192.75) .. controls (369,184.05) and (376.05,177) .. (384.75,177) .. controls (393.45,177) and (400.5,184.05) .. (400.5,192.75) .. controls (400.5,201.45) and (393.45,208.5) .. (384.75,208.5) .. controls (376.05,208.5) and (369,201.45) .. (369,192.75) -- cycle ;
%Straight Lines [id:da5478157115253]
\draw    (143.5,117) -- (178,57.75) ;
%Straight Lines [id:da23537196819907802]
\draw    (135.75,148.5) -- (240,217.75) ;
%Straight Lines [id:da9608719672671411]
\draw [color={rgb, 255:red, 0; green, 77; blue, 255 }  ,draw opacity=1 ]   (225.75,156.5) -- (255.75,202) ;
%Straight Lines [id:da7708180010937593]
\draw [color={rgb, 255:red, 0; green, 77; blue, 255 }  ,draw opacity=1 ]   (193.75,73.5) -- (225.75,125) ;
%Straight Lines [id:da6352081979620849]
\draw    (241.5,140.75) -- (307,134.75) ;
%Straight Lines [id:da7706881326207031]
\draw [color={rgb, 255:red, 0; green, 77; blue, 255 }  ,draw opacity=1 ]   (209.5,57.75) -- (283,54.75) ;
%Straight Lines [id:da9532924386951052]
\draw [color={rgb, 255:red, 0; green, 77; blue, 255 }  ,draw opacity=1 ]   (314.5,54.75) -- (380,77.75) ;
%Straight Lines [id:da6736372249184637]
\draw    (400.5,192.75) -- (449.75,156.5) ;
%Straight Lines [id:da3770610359474238]
\draw    (395.75,93.5) -- (338.5,134.75) ;
%Straight Lines [id:da5936178392792015]
\draw    (322.75,150.5) -- (369,192.75) ;

% Text Node
\draw (130,123) node [anchor=north west][inner sep=0.75pt]   [align=left] {8};
% Text Node
\draw (188,49) node [anchor=north west][inner sep=0.75pt]   [align=left] {0};
% Text Node
\draw (221,132) node [anchor=north west][inner sep=0.75pt]   [align=left] {3};
% Text Node
\draw (251,209) node [anchor=north west][inner sep=0.75pt]   [align=left] {4};
% Text Node
\draw (293,44) node [anchor=north west][inner sep=0.75pt]   [align=left] {1};
% Text Node
\draw (390,68) node [anchor=north west][inner sep=0.75pt]   [align=left] {7};
% Text Node
\draw (317,125) node [anchor=north west][inner sep=0.75pt]   [align=left] {2};
% Text Node
\draw (380,185) node [anchor=north west][inner sep=0.75pt]   [align=left] {5};
% Text Node
\draw (445,133) node [anchor=north west][inner sep=0.75pt]   [align=left] {6};
% Text Node
\draw (415,155) node [anchor=north west][inner sep=0.75pt]   [align=left] {8};
% Text Node
\draw (346,156) node [anchor=north west][inner sep=0.75pt]   [align=left] {1};
% Text Node
\draw (351,98) node [anchor=north west][inner sep=0.75pt]   [align=left] {2};
% Text Node
\draw (344,48) node [anchor=north west][inner sep=0.75pt]   [align=left] {4};
% Text Node
\draw (242,39) node [anchor=north west][inner sep=0.75pt]   [align=left] {3};
% Text Node
\draw (213,88) node [anchor=north west][inner sep=0.75pt]   [align=left] {2};
% Text Node
\draw (143,73) node [anchor=north west][inner sep=0.75pt]   [align=left] {4};
% Text Node
\draw (167,181) node [anchor=north west][inner sep=0.75pt]   [align=left] {8};
% Text Node
\draw (240,165) node [anchor=north west][inner sep=0.75pt]   [align=left] {1};
% Text Node
\draw (270,121) node [anchor=north west][inner sep=0.75pt]   [align=left] {6};


\end{tikzpicture}

    \\ Figura 72 - Grafico unión nodos (1,7)
\end{center}

- Paso 6 \\
El siguiente borde agregado es: (7,2).\\
\begin{center}
\tikzset{every picture/.style={line width=0.75pt}} %set default line width to 0.75pt

\begin{tikzpicture}[x=0.75pt,y=0.75pt,yscale=-1,xscale=1]
%uncomment if require: \path (0,300); %set diagram left start at 0, and has height of 300

%Shape: Circle [id:dp2082955410555598]
\draw  [color={rgb, 255:red, 0; green, 255; blue, 24 }  ,draw opacity=1 ] (178,57.75) .. controls (178,49.05) and (185.05,42) .. (193.75,42) .. controls (202.45,42) and (209.5,49.05) .. (209.5,57.75) .. controls (209.5,66.45) and (202.45,73.5) .. (193.75,73.5) .. controls (185.05,73.5) and (178,66.45) .. (178,57.75) -- cycle ;
%Shape: Circle [id:dp9769973542724641]
\draw  [color={rgb, 255:red, 0; green, 255; blue, 0 }  ,draw opacity=1 ] (283,54.75) .. controls (283,46.05) and (290.05,39) .. (298.75,39) .. controls (307.45,39) and (314.5,46.05) .. (314.5,54.75) .. controls (314.5,63.45) and (307.45,70.5) .. (298.75,70.5) .. controls (290.05,70.5) and (283,63.45) .. (283,54.75) -- cycle ;
%Shape: Circle [id:dp42917390287144164]
\draw   (120,132.75) .. controls (120,124.05) and (127.05,117) .. (135.75,117) .. controls (144.45,117) and (151.5,124.05) .. (151.5,132.75) .. controls (151.5,141.45) and (144.45,148.5) .. (135.75,148.5) .. controls (127.05,148.5) and (120,141.45) .. (120,132.75) -- cycle ;
%Shape: Circle [id:dp6159057665761554]
\draw  [color={rgb, 255:red, 21; green, 255; blue, 0 }  ,draw opacity=1 ] (210,140.75) .. controls (210,132.05) and (217.05,125) .. (225.75,125) .. controls (234.45,125) and (241.5,132.05) .. (241.5,140.75) .. controls (241.5,149.45) and (234.45,156.5) .. (225.75,156.5) .. controls (217.05,156.5) and (210,149.45) .. (210,140.75) -- cycle ;
%Shape: Circle [id:dp22348472907541495]
\draw  [color={rgb, 255:red, 0; green, 255; blue, 0 }  ,draw opacity=1 ] (240,217.75) .. controls (240,209.05) and (247.05,202) .. (255.75,202) .. controls (264.45,202) and (271.5,209.05) .. (271.5,217.75) .. controls (271.5,226.45) and (264.45,233.5) .. (255.75,233.5) .. controls (247.05,233.5) and (240,226.45) .. (240,217.75) -- cycle ;
%Shape: Circle [id:dp7060340718103353]
\draw  [color={rgb, 255:red, 0; green, 255; blue, 0 }  ,draw opacity=1 ] (307,134.75) .. controls (307,126.05) and (314.05,119) .. (322.75,119) .. controls (331.45,119) and (338.5,126.05) .. (338.5,134.75) .. controls (338.5,143.45) and (331.45,150.5) .. (322.75,150.5) .. controls (314.05,150.5) and (307,143.45) .. (307,134.75) -- cycle ;
%Shape: Circle [id:dp6603614261945912]
\draw  [color={rgb, 255:red, 0; green, 255; blue, 0 }  ,draw opacity=1 ] (380,77.75) .. controls (380,69.05) and (387.05,62) .. (395.75,62) .. controls (404.45,62) and (411.5,69.05) .. (411.5,77.75) .. controls (411.5,86.45) and (404.45,93.5) .. (395.75,93.5) .. controls (387.05,93.5) and (380,86.45) .. (380,77.75) -- cycle ;
%Shape: Circle [id:dp3211777580644224]
\draw   (434,140.75) .. controls (434,132.05) and (441.05,125) .. (449.75,125) .. controls (458.45,125) and (465.5,132.05) .. (465.5,140.75) .. controls (465.5,149.45) and (458.45,156.5) .. (449.75,156.5) .. controls (441.05,156.5) and (434,149.45) .. (434,140.75) -- cycle ;
%Shape: Circle [id:dp6117446905390662]
\draw   (369,192.75) .. controls (369,184.05) and (376.05,177) .. (384.75,177) .. controls (393.45,177) and (400.5,184.05) .. (400.5,192.75) .. controls (400.5,201.45) and (393.45,208.5) .. (384.75,208.5) .. controls (376.05,208.5) and (369,201.45) .. (369,192.75) -- cycle ;
%Straight Lines [id:da5478157115253]
\draw    (143.5,117) -- (178,57.75) ;
%Straight Lines [id:da23537196819907802]
\draw    (135.75,148.5) -- (240,217.75) ;
%Straight Lines [id:da9608719672671411]
\draw [color={rgb, 255:red, 0; green, 77; blue, 255 }  ,draw opacity=1 ]   (225.75,156.5) -- (255.75,202) ;
%Straight Lines [id:da7708180010937593]
\draw [color={rgb, 255:red, 0; green, 77; blue, 255 }  ,draw opacity=1 ]   (193.75,73.5) -- (225.75,125) ;
%Straight Lines [id:da6352081979620849]
\draw    (241.5,140.75) -- (307,134.75) ;
%Straight Lines [id:da7706881326207031]
\draw [color={rgb, 255:red, 0; green, 77; blue, 255 }  ,draw opacity=1 ]   (209.5,57.75) -- (283,54.75) ;
%Straight Lines [id:da9532924386951052]
\draw [color={rgb, 255:red, 0; green, 77; blue, 255 }  ,draw opacity=1 ]   (314.5,54.75) -- (380,77.75) ;
%Straight Lines [id:da6736372249184637]
\draw    (400.5,192.75) -- (449.75,156.5) ;
%Straight Lines [id:da3770610359474238]
\draw [color={rgb, 255:red, 0; green, 77; blue, 255 }  ,draw opacity=1 ]   (395.75,93.5) -- (338.5,134.75) ;
%Straight Lines [id:da5936178392792015]
\draw    (322.75,150.5) -- (369,192.75) ;

% Text Node
\draw (130,123) node [anchor=north west][inner sep=0.75pt]   [align=left] {8};
% Text Node
\draw (188,49) node [anchor=north west][inner sep=0.75pt]   [align=left] {0};
% Text Node
\draw (221,132) node [anchor=north west][inner sep=0.75pt]   [align=left] {3};
% Text Node
\draw (251,209) node [anchor=north west][inner sep=0.75pt]   [align=left] {4};
% Text Node
\draw (293,44) node [anchor=north west][inner sep=0.75pt]   [align=left] {1};
% Text Node
\draw (390,68) node [anchor=north west][inner sep=0.75pt]   [align=left] {7};
% Text Node
\draw (317,125) node [anchor=north west][inner sep=0.75pt]   [align=left] {2};
% Text Node
\draw (380,185) node [anchor=north west][inner sep=0.75pt]   [align=left] {5};
% Text Node
\draw (445,133) node [anchor=north west][inner sep=0.75pt]   [align=left] {6};
% Text Node
\draw (415,155) node [anchor=north west][inner sep=0.75pt]   [align=left] {8};
% Text Node
\draw (346,156) node [anchor=north west][inner sep=0.75pt]   [align=left] {1};
% Text Node
\draw (351,98) node [anchor=north west][inner sep=0.75pt]   [align=left] {2};
% Text Node
\draw (344,48) node [anchor=north west][inner sep=0.75pt]   [align=left] {4};
% Text Node
\draw (242,39) node [anchor=north west][inner sep=0.75pt]   [align=left] {3};
% Text Node
\draw (213,88) node [anchor=north west][inner sep=0.75pt]   [align=left] {2};
% Text Node
\draw (143,73) node [anchor=north west][inner sep=0.75pt]   [align=left] {4};
% Text Node
\draw (167,181) node [anchor=north west][inner sep=0.75pt]   [align=left] {8};
% Text Node
\draw (240,165) node [anchor=north west][inner sep=0.75pt]   [align=left] {1};
% Text Node
\draw (270,121) node [anchor=north west][inner sep=0.75pt]   [align=left] {6};


\end{tikzpicture}
    \\Figura 73 - Grafico unión nodos (7,2)
\end{center}
\newpage
- Paso 7 \\
El siguiente borde agregado es: (2,5)
\begin{center}


\tikzset{every picture/.style={line width=0.75pt}} %set default line width to 0.75pt

\begin{tikzpicture}[x=0.75pt,y=0.75pt,yscale=-1,xscale=1]
%uncomment if require: \path (0,300); %set diagram left start at 0, and has height of 300

%Shape: Circle [id:dp2082955410555598]
\draw  [color={rgb, 255:red, 0; green, 255; blue, 24 }  ,draw opacity=1 ] (178,57.75) .. controls (178,49.05) and (185.05,42) .. (193.75,42) .. controls (202.45,42) and (209.5,49.05) .. (209.5,57.75) .. controls (209.5,66.45) and (202.45,73.5) .. (193.75,73.5) .. controls (185.05,73.5) and (178,66.45) .. (178,57.75) -- cycle ;
%Shape: Circle [id:dp9769973542724641]
\draw  [color={rgb, 255:red, 0; green, 255; blue, 0 }  ,draw opacity=1 ] (283,54.75) .. controls (283,46.05) and (290.05,39) .. (298.75,39) .. controls (307.45,39) and (314.5,46.05) .. (314.5,54.75) .. controls (314.5,63.45) and (307.45,70.5) .. (298.75,70.5) .. controls (290.05,70.5) and (283,63.45) .. (283,54.75) -- cycle ;
%Shape: Circle [id:dp42917390287144164]
\draw   (120,132.75) .. controls (120,124.05) and (127.05,117) .. (135.75,117) .. controls (144.45,117) and (151.5,124.05) .. (151.5,132.75) .. controls (151.5,141.45) and (144.45,148.5) .. (135.75,148.5) .. controls (127.05,148.5) and (120,141.45) .. (120,132.75) -- cycle ;
%Shape: Circle [id:dp6159057665761554]
\draw  [color={rgb, 255:red, 21; green, 255; blue, 0 }  ,draw opacity=1 ] (210,140.75) .. controls (210,132.05) and (217.05,125) .. (225.75,125) .. controls (234.45,125) and (241.5,132.05) .. (241.5,140.75) .. controls (241.5,149.45) and (234.45,156.5) .. (225.75,156.5) .. controls (217.05,156.5) and (210,149.45) .. (210,140.75) -- cycle ;
%Shape: Circle [id:dp22348472907541495]
\draw  [color={rgb, 255:red, 0; green, 255; blue, 0 }  ,draw opacity=1 ] (240,217.75) .. controls (240,209.05) and (247.05,202) .. (255.75,202) .. controls (264.45,202) and (271.5,209.05) .. (271.5,217.75) .. controls (271.5,226.45) and (264.45,233.5) .. (255.75,233.5) .. controls (247.05,233.5) and (240,226.45) .. (240,217.75) -- cycle ;
%Shape: Circle [id:dp7060340718103353]
\draw  [color={rgb, 255:red, 0; green, 255; blue, 0 }  ,draw opacity=1 ] (307,134.75) .. controls (307,126.05) and (314.05,119) .. (322.75,119) .. controls (331.45,119) and (338.5,126.05) .. (338.5,134.75) .. controls (338.5,143.45) and (331.45,150.5) .. (322.75,150.5) .. controls (314.05,150.5) and (307,143.45) .. (307,134.75) -- cycle ;
%Shape: Circle [id:dp6603614261945912]
\draw  [color={rgb, 255:red, 0; green, 255; blue, 0 }  ,draw opacity=1 ] (380,77.75) .. controls (380,69.05) and (387.05,62) .. (395.75,62) .. controls (404.45,62) and (411.5,69.05) .. (411.5,77.75) .. controls (411.5,86.45) and (404.45,93.5) .. (395.75,93.5) .. controls (387.05,93.5) and (380,86.45) .. (380,77.75) -- cycle ;
%Shape: Circle [id:dp3211777580644224]
\draw   (434,140.75) .. controls (434,132.05) and (441.05,125) .. (449.75,125) .. controls (458.45,125) and (465.5,132.05) .. (465.5,140.75) .. controls (465.5,149.45) and (458.45,156.5) .. (449.75,156.5) .. controls (441.05,156.5) and (434,149.45) .. (434,140.75) -- cycle ;
%Shape: Circle [id:dp6117446905390662]
\draw  [color={rgb, 255:red, 0; green, 255; blue, 0 }  ,draw opacity=1 ] (369,192.75) .. controls (369,184.05) and (376.05,177) .. (384.75,177) .. controls (393.45,177) and (400.5,184.05) .. (400.5,192.75) .. controls (400.5,201.45) and (393.45,208.5) .. (384.75,208.5) .. controls (376.05,208.5) and (369,201.45) .. (369,192.75) -- cycle ;
%Straight Lines [id:da5478157115253]
\draw    (143.5,117) -- (178,57.75) ;
%Straight Lines [id:da23537196819907802]
\draw    (135.75,148.5) -- (240,217.75) ;
%Straight Lines [id:da9608719672671411]
\draw [color={rgb, 255:red, 0; green, 77; blue, 255 }  ,draw opacity=1 ]   (225.75,156.5) -- (255.75,202) ;
%Straight Lines [id:da7708180010937593]
\draw [color={rgb, 255:red, 0; green, 77; blue, 255 }  ,draw opacity=1 ]   (193.75,73.5) -- (225.75,125) ;
%Straight Lines [id:da6352081979620849]
\draw    (241.5,140.75) -- (307,134.75) ;
%Straight Lines [id:da7706881326207031]
\draw [color={rgb, 255:red, 0; green, 77; blue, 255 }  ,draw opacity=1 ]   (209.5,57.75) -- (283,54.75) ;
%Straight Lines [id:da9532924386951052]
\draw [color={rgb, 255:red, 0; green, 77; blue, 255 }  ,draw opacity=1 ]   (314.5,54.75) -- (380,77.75) ;
%Straight Lines [id:da6736372249184637]
\draw    (400.5,192.75) -- (449.75,156.5) ;
%Straight Lines [id:da3770610359474238]
\draw [color={rgb, 255:red, 0; green, 77; blue, 255 }  ,draw opacity=1 ]   (395.75,93.5) -- (338.5,134.75) ;
%Straight Lines [id:da5936178392792015]
\draw [color={rgb, 255:red, 0; green, 77; blue, 255 }  ,draw opacity=1 ]   (322.75,150.5) -- (369,192.75) ;

% Text Node
\draw (130,123) node [anchor=north west][inner sep=0.75pt]   [align=left] {8};
% Text Node
\draw (188,49) node [anchor=north west][inner sep=0.75pt]   [align=left] {0};
% Text Node
\draw (221,132) node [anchor=north west][inner sep=0.75pt]   [align=left] {3};
% Text Node
\draw (251,209) node [anchor=north west][inner sep=0.75pt]   [align=left] {4};
% Text Node
\draw (293,44) node [anchor=north west][inner sep=0.75pt]   [align=left] {1};
% Text Node
\draw (390,68) node [anchor=north west][inner sep=0.75pt]   [align=left] {7};
% Text Node
\draw (317,125) node [anchor=north west][inner sep=0.75pt]   [align=left] {2};
% Text Node
\draw (380,185) node [anchor=north west][inner sep=0.75pt]   [align=left] {5};
% Text Node
\draw (445,133) node [anchor=north west][inner sep=0.75pt]   [align=left] {6};
% Text Node
\draw (415,155) node [anchor=north west][inner sep=0.75pt]   [align=left] {8};
% Text Node
\draw (346,156) node [anchor=north west][inner sep=0.75pt]   [align=left] {1};
% Text Node
\draw (351,98) node [anchor=north west][inner sep=0.75pt]   [align=left] {2};
% Text Node
\draw (344,48) node [anchor=north west][inner sep=0.75pt]   [align=left] {4};
% Text Node
\draw (242,39) node [anchor=north west][inner sep=0.75pt]   [align=left] {3};
% Text Node
\draw (213,88) node [anchor=north west][inner sep=0.75pt]   [align=left] {2};
% Text Node
\draw (143,73) node [anchor=north west][inner sep=0.75pt]   [align=left] {4};
% Text Node
\draw (167,181) node [anchor=north west][inner sep=0.75pt]   [align=left] {8};
% Text Node
\draw (240,165) node [anchor=north west][inner sep=0.75pt]   [align=left] {1};
% Text Node
\draw (270,121) node [anchor=north west][inner sep=0.75pt]   [align=left] {6};


\end{tikzpicture}

    \\ Figura 74 - Grafico unión nodos (2,5)
\end{center}
- Paso 8 \\
El siguiente borde agregado es: (0,8).
\begin{center}


\tikzset{every picture/.style={line width=0.75pt}} %set default line width to 0.75pt

\begin{tikzpicture}[x=0.75pt,y=0.75pt,yscale=-1,xscale=1]
%uncomment if require: \path (0,300); %set diagram left start at 0, and has height of 300

%Shape: Circle [id:dp2082955410555598]
\draw  [color={rgb, 255:red, 0; green, 255; blue, 24 }  ,draw opacity=1 ] (178,57.75) .. controls (178,49.05) and (185.05,42) .. (193.75,42) .. controls (202.45,42) and (209.5,49.05) .. (209.5,57.75) .. controls (209.5,66.45) and (202.45,73.5) .. (193.75,73.5) .. controls (185.05,73.5) and (178,66.45) .. (178,57.75) -- cycle ;
%Shape: Circle [id:dp9769973542724641]
\draw  [color={rgb, 255:red, 0; green, 255; blue, 0 }  ,draw opacity=1 ] (283,54.75) .. controls (283,46.05) and (290.05,39) .. (298.75,39) .. controls (307.45,39) and (314.5,46.05) .. (314.5,54.75) .. controls (314.5,63.45) and (307.45,70.5) .. (298.75,70.5) .. controls (290.05,70.5) and (283,63.45) .. (283,54.75) -- cycle ;
%Shape: Circle [id:dp42917390287144164]
\draw  [color={rgb, 255:red, 0; green, 255; blue, 0 }  ,draw opacity=1 ] (120,132.75) .. controls (120,124.05) and (127.05,117) .. (135.75,117) .. controls (144.45,117) and (151.5,124.05) .. (151.5,132.75) .. controls (151.5,141.45) and (144.45,148.5) .. (135.75,148.5) .. controls (127.05,148.5) and (120,141.45) .. (120,132.75) -- cycle ;
%Shape: Circle [id:dp6159057665761554]
\draw  [color={rgb, 255:red, 21; green, 255; blue, 0 }  ,draw opacity=1 ] (210,140.75) .. controls (210,132.05) and (217.05,125) .. (225.75,125) .. controls (234.45,125) and (241.5,132.05) .. (241.5,140.75) .. controls (241.5,149.45) and (234.45,156.5) .. (225.75,156.5) .. controls (217.05,156.5) and (210,149.45) .. (210,140.75) -- cycle ;
%Shape: Circle [id:dp22348472907541495]
\draw  [color={rgb, 255:red, 0; green, 255; blue, 0 }  ,draw opacity=1 ] (240,217.75) .. controls (240,209.05) and (247.05,202) .. (255.75,202) .. controls (264.45,202) and (271.5,209.05) .. (271.5,217.75) .. controls (271.5,226.45) and (264.45,233.5) .. (255.75,233.5) .. controls (247.05,233.5) and (240,226.45) .. (240,217.75) -- cycle ;
%Shape: Circle [id:dp7060340718103353]
\draw  [color={rgb, 255:red, 0; green, 255; blue, 0 }  ,draw opacity=1 ] (307,134.75) .. controls (307,126.05) and (314.05,119) .. (322.75,119) .. controls (331.45,119) and (338.5,126.05) .. (338.5,134.75) .. controls (338.5,143.45) and (331.45,150.5) .. (322.75,150.5) .. controls (314.05,150.5) and (307,143.45) .. (307,134.75) -- cycle ;
%Shape: Circle [id:dp6603614261945912]
\draw  [color={rgb, 255:red, 0; green, 255; blue, 0 }  ,draw opacity=1 ] (380,77.75) .. controls (380,69.05) and (387.05,62) .. (395.75,62) .. controls (404.45,62) and (411.5,69.05) .. (411.5,77.75) .. controls (411.5,86.45) and (404.45,93.5) .. (395.75,93.5) .. controls (387.05,93.5) and (380,86.45) .. (380,77.75) -- cycle ;
%Shape: Circle [id:dp3211777580644224]
\draw   (434,140.75) .. controls (434,132.05) and (441.05,125) .. (449.75,125) .. controls (458.45,125) and (465.5,132.05) .. (465.5,140.75) .. controls (465.5,149.45) and (458.45,156.5) .. (449.75,156.5) .. controls (441.05,156.5) and (434,149.45) .. (434,140.75) -- cycle ;
%Shape: Circle [id:dp6117446905390662]
\draw  [color={rgb, 255:red, 0; green, 255; blue, 0 }  ,draw opacity=1 ] (369,192.75) .. controls (369,184.05) and (376.05,177) .. (384.75,177) .. controls (393.45,177) and (400.5,184.05) .. (400.5,192.75) .. controls (400.5,201.45) and (393.45,208.5) .. (384.75,208.5) .. controls (376.05,208.5) and (369,201.45) .. (369,192.75) -- cycle ;
%Straight Lines [id:da5478157115253]
\draw [color={rgb, 255:red, 0; green, 77; blue, 255 }  ,draw opacity=1 ]   (143.5,117) -- (178,57.75) ;
%Straight Lines [id:da23537196819907802]
\draw    (135.75,148.5) -- (240,217.75) ;
%Straight Lines [id:da9608719672671411]
\draw [color={rgb, 255:red, 0; green, 77; blue, 255 }  ,draw opacity=1 ]   (225.75,156.5) -- (255.75,202) ;
%Straight Lines [id:da7708180010937593]
\draw [color={rgb, 255:red, 0; green, 77; blue, 255 }  ,draw opacity=1 ]   (193.75,73.5) -- (225.75,125) ;
%Straight Lines [id:da6352081979620849]
\draw    (241.5,140.75) -- (307,134.75) ;
%Straight Lines [id:da7706881326207031]
\draw [color={rgb, 255:red, 0; green, 77; blue, 255 }  ,draw opacity=1 ]   (209.5,57.75) -- (283,54.75) ;
%Straight Lines [id:da9532924386951052]
\draw [color={rgb, 255:red, 0; green, 77; blue, 255 }  ,draw opacity=1 ]   (314.5,54.75) -- (380,77.75) ;
%Straight Lines [id:da6736372249184637]
\draw    (400.5,192.75) -- (449.75,156.5) ;
%Straight Lines [id:da3770610359474238]
\draw [color={rgb, 255:red, 0; green, 77; blue, 255 }  ,draw opacity=1 ]   (395.75,93.5) -- (338.5,134.75) ;
%Straight Lines [id:da5936178392792015]
\draw [color={rgb, 255:red, 0; green, 77; blue, 255 }  ,draw opacity=1 ]   (322.75,150.5) -- (369,192.75) ;

% Text Node
\draw (130,123) node [anchor=north west][inner sep=0.75pt]   [align=left] {8};
% Text Node
\draw (188,49) node [anchor=north west][inner sep=0.75pt]   [align=left] {0};
% Text Node
\draw (221,132) node [anchor=north west][inner sep=0.75pt]   [align=left] {3};
% Text Node
\draw (251,209) node [anchor=north west][inner sep=0.75pt]   [align=left] {4};
% Text Node
\draw (293,44) node [anchor=north west][inner sep=0.75pt]   [align=left] {1};
% Text Node
\draw (390,68) node [anchor=north west][inner sep=0.75pt]   [align=left] {7};
% Text Node
\draw (317,125) node [anchor=north west][inner sep=0.75pt]   [align=left] {2};
% Text Node
\draw (380,185) node [anchor=north west][inner sep=0.75pt]   [align=left] {5};
% Text Node
\draw (445,133) node [anchor=north west][inner sep=0.75pt]   [align=left] {6};
% Text Node
\draw (415,155) node [anchor=north west][inner sep=0.75pt]   [align=left] {8};
% Text Node
\draw (346,156) node [anchor=north west][inner sep=0.75pt]   [align=left] {1};
% Text Node
\draw (351,98) node [anchor=north west][inner sep=0.75pt]   [align=left] {2};
% Text Node
\draw (344,48) node [anchor=north west][inner sep=0.75pt]   [align=left] {4};
% Text Node
\draw (242,39) node [anchor=north west][inner sep=0.75pt]   [align=left] {3};
% Text Node
\draw (213,88) node [anchor=north west][inner sep=0.75pt]   [align=left] {2};
% Text Node
\draw (143,73) node [anchor=north west][inner sep=0.75pt]   [align=left] {4};
% Text Node
\draw (167,181) node [anchor=north west][inner sep=0.75pt]   [align=left] {8};
% Text Node
\draw (240,165) node [anchor=north west][inner sep=0.75pt]   [align=left] {1};
% Text Node
\draw (270,121) node [anchor=north west][inner sep=0.75pt]   [align=left] {6};


\end{tikzpicture}
    \\Figura 75 - Grafico unión nodos (0,8)
\end{center}
- Paso 9 \\
El siguiente borde agregado es: (5,6).
\begin{center}


\tikzset{every picture/.style={line width=0.75pt}} %set default line width to 0.75pt

\begin{tikzpicture}[x=0.75pt,y=0.75pt,yscale=-1,xscale=1]
%uncomment if require: \path (0,300); %set diagram left start at 0, and has height of 300

%Shape: Circle [id:dp2082955410555598]
\draw  [color={rgb, 255:red, 0; green, 255; blue, 24 }  ,draw opacity=1 ] (178,57.75) .. controls (178,49.05) and (185.05,42) .. (193.75,42) .. controls (202.45,42) and (209.5,49.05) .. (209.5,57.75) .. controls (209.5,66.45) and (202.45,73.5) .. (193.75,73.5) .. controls (185.05,73.5) and (178,66.45) .. (178,57.75) -- cycle ;
%Shape: Circle [id:dp9769973542724641]
\draw  [color={rgb, 255:red, 0; green, 255; blue, 0 }  ,draw opacity=1 ] (283,54.75) .. controls (283,46.05) and (290.05,39) .. (298.75,39) .. controls (307.45,39) and (314.5,46.05) .. (314.5,54.75) .. controls (314.5,63.45) and (307.45,70.5) .. (298.75,70.5) .. controls (290.05,70.5) and (283,63.45) .. (283,54.75) -- cycle ;
%Shape: Circle [id:dp42917390287144164]
\draw  [color={rgb, 255:red, 0; green, 255; blue, 0 }  ,draw opacity=1 ] (120,132.75) .. controls (120,124.05) and (127.05,117) .. (135.75,117) .. controls (144.45,117) and (151.5,124.05) .. (151.5,132.75) .. controls (151.5,141.45) and (144.45,148.5) .. (135.75,148.5) .. controls (127.05,148.5) and (120,141.45) .. (120,132.75) -- cycle ;
%Shape: Circle [id:dp6159057665761554]
\draw  [color={rgb, 255:red, 21; green, 255; blue, 0 }  ,draw opacity=1 ] (210,140.75) .. controls (210,132.05) and (217.05,125) .. (225.75,125) .. controls (234.45,125) and (241.5,132.05) .. (241.5,140.75) .. controls (241.5,149.45) and (234.45,156.5) .. (225.75,156.5) .. controls (217.05,156.5) and (210,149.45) .. (210,140.75) -- cycle ;
%Shape: Circle [id:dp22348472907541495]
\draw  [color={rgb, 255:red, 0; green, 255; blue, 0 }  ,draw opacity=1 ] (240,217.75) .. controls (240,209.05) and (247.05,202) .. (255.75,202) .. controls (264.45,202) and (271.5,209.05) .. (271.5,217.75) .. controls (271.5,226.45) and (264.45,233.5) .. (255.75,233.5) .. controls (247.05,233.5) and (240,226.45) .. (240,217.75) -- cycle ;
%Shape: Circle [id:dp7060340718103353]
\draw  [color={rgb, 255:red, 0; green, 255; blue, 0 }  ,draw opacity=1 ] (307,134.75) .. controls (307,126.05) and (314.05,119) .. (322.75,119) .. controls (331.45,119) and (338.5,126.05) .. (338.5,134.75) .. controls (338.5,143.45) and (331.45,150.5) .. (322.75,150.5) .. controls (314.05,150.5) and (307,143.45) .. (307,134.75) -- cycle ;
%Shape: Circle [id:dp6603614261945912]
\draw  [color={rgb, 255:red, 0; green, 255; blue, 0 }  ,draw opacity=1 ] (380,77.75) .. controls (380,69.05) and (387.05,62) .. (395.75,62) .. controls (404.45,62) and (411.5,69.05) .. (411.5,77.75) .. controls (411.5,86.45) and (404.45,93.5) .. (395.75,93.5) .. controls (387.05,93.5) and (380,86.45) .. (380,77.75) -- cycle ;
%Shape: Circle [id:dp3211777580644224]
\draw  [color={rgb, 255:red, 0; green, 255; blue, 0 }  ,draw opacity=1 ] (434,140.75) .. controls (434,132.05) and (441.05,125) .. (449.75,125) .. controls (458.45,125) and (465.5,132.05) .. (465.5,140.75) .. controls (465.5,149.45) and (458.45,156.5) .. (449.75,156.5) .. controls (441.05,156.5) and (434,149.45) .. (434,140.75) -- cycle ;
%Shape: Circle [id:dp6117446905390662]
\draw  [color={rgb, 255:red, 0; green, 255; blue, 0 }  ,draw opacity=1 ] (369,192.75) .. controls (369,184.05) and (376.05,177) .. (384.75,177) .. controls (393.45,177) and (400.5,184.05) .. (400.5,192.75) .. controls (400.5,201.45) and (393.45,208.5) .. (384.75,208.5) .. controls (376.05,208.5) and (369,201.45) .. (369,192.75) -- cycle ;
%Straight Lines [id:da5478157115253]
\draw [color={rgb, 255:red, 0; green, 77; blue, 255 }  ,draw opacity=1 ]   (143.5,117) -- (178,57.75) ;
%Straight Lines [id:da23537196819907802]
\draw    (135.75,148.5) -- (240,217.75) ;
%Straight Lines [id:da9608719672671411]
\draw [color={rgb, 255:red, 0; green, 77; blue, 255 }  ,draw opacity=1 ]   (225.75,156.5) -- (255.75,202) ;
%Straight Lines [id:da7708180010937593]
\draw [color={rgb, 255:red, 0; green, 77; blue, 255 }  ,draw opacity=1 ]   (193.75,73.5) -- (225.75,125) ;
%Straight Lines [id:da6352081979620849]
\draw    (241.5,140.75) -- (307,134.75) ;
%Straight Lines [id:da7706881326207031]
\draw [color={rgb, 255:red, 0; green, 77; blue, 255 }  ,draw opacity=1 ]   (209.5,57.75) -- (283,54.75) ;
%Straight Lines [id:da9532924386951052]
\draw [color={rgb, 255:red, 0; green, 77; blue, 255 }  ,draw opacity=1 ]   (314.5,54.75) -- (380,77.75) ;
%Straight Lines [id:da6736372249184637]
\draw [color={rgb, 255:red, 0; green, 77; blue, 255 }  ,draw opacity=1 ]   (400.5,192.75) -- (449.75,156.5) ;
%Straight Lines [id:da3770610359474238]
\draw [color={rgb, 255:red, 0; green, 77; blue, 255 }  ,draw opacity=1 ]   (395.75,93.5) -- (338.5,134.75) ;
%Straight Lines [id:da5936178392792015]
\draw [color={rgb, 255:red, 0; green, 77; blue, 255 }  ,draw opacity=1 ]   (322.75,150.5) -- (369,192.75) ;

% Text Node
\draw (130,123) node [anchor=north west][inner sep=0.75pt]   [align=left] {8};
% Text Node
\draw (188,49) node [anchor=north west][inner sep=0.75pt]   [align=left] {0};
% Text Node
\draw (221,132) node [anchor=north west][inner sep=0.75pt]   [align=left] {3};
% Text Node
\draw (251,209) node [anchor=north west][inner sep=0.75pt]   [align=left] {4};
% Text Node
\draw (293,44) node [anchor=north west][inner sep=0.75pt]   [align=left] {1};
% Text Node
\draw (390,68) node [anchor=north west][inner sep=0.75pt]   [align=left] {7};
% Text Node
\draw (317,125) node [anchor=north west][inner sep=0.75pt]   [align=left] {2};
% Text Node
\draw (380,185) node [anchor=north west][inner sep=0.75pt]   [align=left] {5};
% Text Node
\draw (445,133) node [anchor=north west][inner sep=0.75pt]   [align=left] {6};
% Text Node
\draw (415,155) node [anchor=north west][inner sep=0.75pt]   [align=left] {8};
% Text Node
\draw (346,156) node [anchor=north west][inner sep=0.75pt]   [align=left] {1};
% Text Node
\draw (351,98) node [anchor=north west][inner sep=0.75pt]   [align=left] {2};
% Text Node
\draw (344,48) node [anchor=north west][inner sep=0.75pt]   [align=left] {4};
% Text Node
\draw (242,39) node [anchor=north west][inner sep=0.75pt]   [align=left] {3};
% Text Node
\draw (213,88) node [anchor=north west][inner sep=0.75pt]   [align=left] {2};
% Text Node
\draw (143,73) node [anchor=north west][inner sep=0.75pt]   [align=left] {4};
% Text Node
\draw (167,181) node [anchor=north west][inner sep=0.75pt]   [align=left] {8};
% Text Node
\draw (240,165) node [anchor=north west][inner sep=0.75pt]   [align=left] {1};
% Text Node
\draw (270,121) node [anchor=north west][inner sep=0.75pt]   [align=left] {6};


\end{tikzpicture}

    Figura 76 - Grafico unión nodos (5,6)
\end{center}
Como podemos observar ahora se alcanzan todos los nodos del grafico cumpliendo el objetivo del algoritmo de Prim. \\
El costo mínimo del árbol de expansión es:\\
\begin{center}


\tikzset{every picture/.style={line width=0.75pt}} %set default line width to 0.75pt

\begin{tikzpicture}[x=0.75pt,y=0.75pt,yscale=-1,xscale=1]
%uncomment if require: \path (0,300); %set diagram left start at 0, and has height of 300

%Shape: Circle [id:dp2082955410555598]
\draw  [color={rgb, 255:red, 0; green, 255; blue, 24 }  ,draw opacity=1 ] (178,57.75) .. controls (178,49.05) and (185.05,42) .. (193.75,42) .. controls (202.45,42) and (209.5,49.05) .. (209.5,57.75) .. controls (209.5,66.45) and (202.45,73.5) .. (193.75,73.5) .. controls (185.05,73.5) and (178,66.45) .. (178,57.75) -- cycle ;
%Shape: Circle [id:dp9769973542724641]
\draw  [color={rgb, 255:red, 0; green, 255; blue, 0 }  ,draw opacity=1 ] (283,54.75) .. controls (283,46.05) and (290.05,39) .. (298.75,39) .. controls (307.45,39) and (314.5,46.05) .. (314.5,54.75) .. controls (314.5,63.45) and (307.45,70.5) .. (298.75,70.5) .. controls (290.05,70.5) and (283,63.45) .. (283,54.75) -- cycle ;
%Shape: Circle [id:dp42917390287144164]
\draw  [color={rgb, 255:red, 0; green, 255; blue, 0 }  ,draw opacity=1 ] (120,132.75) .. controls (120,124.05) and (127.05,117) .. (135.75,117) .. controls (144.45,117) and (151.5,124.05) .. (151.5,132.75) .. controls (151.5,141.45) and (144.45,148.5) .. (135.75,148.5) .. controls (127.05,148.5) and (120,141.45) .. (120,132.75) -- cycle ;
%Shape: Circle [id:dp6159057665761554]
\draw  [color={rgb, 255:red, 21; green, 255; blue, 0 }  ,draw opacity=1 ] (210,140.75) .. controls (210,132.05) and (217.05,125) .. (225.75,125) .. controls (234.45,125) and (241.5,132.05) .. (241.5,140.75) .. controls (241.5,149.45) and (234.45,156.5) .. (225.75,156.5) .. controls (217.05,156.5) and (210,149.45) .. (210,140.75) -- cycle ;
%Shape: Circle [id:dp22348472907541495]
\draw  [color={rgb, 255:red, 0; green, 255; blue, 0 }  ,draw opacity=1 ] (240,217.75) .. controls (240,209.05) and (247.05,202) .. (255.75,202) .. controls (264.45,202) and (271.5,209.05) .. (271.5,217.75) .. controls (271.5,226.45) and (264.45,233.5) .. (255.75,233.5) .. controls (247.05,233.5) and (240,226.45) .. (240,217.75) -- cycle ;
%Shape: Circle [id:dp7060340718103353]
\draw  [color={rgb, 255:red, 0; green, 255; blue, 0 }  ,draw opacity=1 ] (307,134.75) .. controls (307,126.05) and (314.05,119) .. (322.75,119) .. controls (331.45,119) and (338.5,126.05) .. (338.5,134.75) .. controls (338.5,143.45) and (331.45,150.5) .. (322.75,150.5) .. controls (314.05,150.5) and (307,143.45) .. (307,134.75) -- cycle ;
%Shape: Circle [id:dp6603614261945912]
\draw  [color={rgb, 255:red, 0; green, 255; blue, 0 }  ,draw opacity=1 ] (380,77.75) .. controls (380,69.05) and (387.05,62) .. (395.75,62) .. controls (404.45,62) and (411.5,69.05) .. (411.5,77.75) .. controls (411.5,86.45) and (404.45,93.5) .. (395.75,93.5) .. controls (387.05,93.5) and (380,86.45) .. (380,77.75) -- cycle ;
%Shape: Circle [id:dp3211777580644224]
\draw  [color={rgb, 255:red, 0; green, 255; blue, 0 }  ,draw opacity=1 ] (434,140.75) .. controls (434,132.05) and (441.05,125) .. (449.75,125) .. controls (458.45,125) and (465.5,132.05) .. (465.5,140.75) .. controls (465.5,149.45) and (458.45,156.5) .. (449.75,156.5) .. controls (441.05,156.5) and (434,149.45) .. (434,140.75) -- cycle ;
%Shape: Circle [id:dp6117446905390662]
\draw  [color={rgb, 255:red, 0; green, 255; blue, 0 }  ,draw opacity=1 ] (369,192.75) .. controls (369,184.05) and (376.05,177) .. (384.75,177) .. controls (393.45,177) and (400.5,184.05) .. (400.5,192.75) .. controls (400.5,201.45) and (393.45,208.5) .. (384.75,208.5) .. controls (376.05,208.5) and (369,201.45) .. (369,192.75) -- cycle ;
%Straight Lines [id:da5478157115253]
\draw [color={rgb, 255:red, 0; green, 77; blue, 255 }  ,draw opacity=1 ]   (143.5,117) -- (178,57.75) ;
%Straight Lines [id:da9608719672671411]
\draw [color={rgb, 255:red, 0; green, 77; blue, 255 }  ,draw opacity=1 ]   (225.75,156.5) -- (255.75,202) ;
%Straight Lines [id:da7708180010937593]
\draw [color={rgb, 255:red, 0; green, 77; blue, 255 }  ,draw opacity=1 ]   (193.75,73.5) -- (225.75,125) ;
%Straight Lines [id:da7706881326207031]
\draw [color={rgb, 255:red, 0; green, 77; blue, 255 }  ,draw opacity=1 ]   (209.5,57.75) -- (283,54.75) ;
%Straight Lines [id:da9532924386951052]
\draw [color={rgb, 255:red, 0; green, 77; blue, 255 }  ,draw opacity=1 ]   (314.5,54.75) -- (380,77.75) ;
%Straight Lines [id:da6736372249184637]
\draw [color={rgb, 255:red, 0; green, 77; blue, 255 }  ,draw opacity=1 ]   (400.5,192.75) -- (449.75,156.5) ;
%Straight Lines [id:da3770610359474238]
\draw [color={rgb, 255:red, 0; green, 77; blue, 255 }  ,draw opacity=1 ]   (395.75,93.5) -- (338.5,134.75) ;
%Straight Lines [id:da5936178392792015]
\draw [color={rgb, 255:red, 0; green, 77; blue, 255 }  ,draw opacity=1 ]   (322.75,150.5) -- (369,192.75) ;

% Text Node
\draw (130,123) node [anchor=north west][inner sep=0.75pt]   [align=left] {8};
% Text Node
\draw (188,49) node [anchor=north west][inner sep=0.75pt]   [align=left] {0};
% Text Node
\draw (221,132) node [anchor=north west][inner sep=0.75pt]   [align=left] {3};
% Text Node
\draw (251,209) node [anchor=north west][inner sep=0.75pt]   [align=left] {4};
% Text Node
\draw (293,44) node [anchor=north west][inner sep=0.75pt]   [align=left] {1};
% Text Node
\draw (390,68) node [anchor=north west][inner sep=0.75pt]   [align=left] {7};
% Text Node
\draw (317,125) node [anchor=north west][inner sep=0.75pt]   [align=left] {2};
% Text Node
\draw (380,185) node [anchor=north west][inner sep=0.75pt]   [align=left] {5};
% Text Node
\draw (445,133) node [anchor=north west][inner sep=0.75pt]   [align=left] {6};
% Text Node
\draw (415,155) node [anchor=north west][inner sep=0.75pt]   [align=left] {8};
% Text Node
\draw (346,156) node [anchor=north west][inner sep=0.75pt]   [align=left] {1};
% Text Node
\draw (351,98) node [anchor=north west][inner sep=0.75pt]   [align=left] {2};
% Text Node
\draw (344,48) node [anchor=north west][inner sep=0.75pt]   [align=left] {4};
% Text Node
\draw (242,39) node [anchor=north west][inner sep=0.75pt]   [align=left] {3};
% Text Node
\draw (213,88) node [anchor=north west][inner sep=0.75pt]   [align=left] {2};
% Text Node
\draw (143,73) node [anchor=north west][inner sep=0.75pt]   [align=left] {4};
% Text Node
\draw (240,165) node [anchor=north west][inner sep=0.75pt]   [align=left] {1};


\end{tikzpicture}

     \\Figura 77 - Arbol de Expansion Mínimo.
\end{center}

\subsection{Algoritmo de Kruskal.}
Es un algoritmo de la teoría de grafos para encontrar un árbol recubridor mínimo en un grafo conexo y ponderado. Es decir, busca un subconjunto de aristas que, formando un árbol, incluyen todos los vértices y donde el valor total de todas las aristas del árbol es el mínimo. Si el grafo no es conexo, entonces busca un bosque expandido mínimo (un árbol expandido mínimo para cada componente conexa). \\
El algoritmo de Kruskal, dado un grafo conexo, no dirigido y ponderado, encuentra un árbol de expansión mínima. Es decir, es capaz de encontrar un subconjunto de las aristas que formen un árbol que incluya todos los vértices del grafo inicial, donde el peso total de las aristas del árbol es el mínimo posible.\\

\subsection{Funcionamiento}
\begin{itemize}
    \item Se crea un bosque B (un conjunto de árboles), donde cada vértice del grafo es un árbol separado
    \item Se crea un conjunto C que contenga a todas las aristas del grafo
    \item Mientras C es no vacío
    \item Eliminar una arista de peso mínimo de C
    \item Si esa arista conecta dos árboles diferentes se añade al bosque, combinando los dos árboles en un solo árbol
    \item En caso contrario, se desecha la arista
Al acabar el algoritmo, el bosque tiene un solo componente, el cual forma un árbol de expansión mínimo del grafo.
\end{itemize}

\subsection{Pseudocodigo algoritmo kruskal}
\begin{lstlisting}
1- kruskal (G: grafo, n: #nodos)
2- Construir una cola de prioridad cp con los arcos del grafo G
3- Inicializar componente conexa
4- F = Conjunto vacio
5- while !vacia(cp) && |F| < n-1
6- 	arista = obtenerMin(cp)
7-	borrarMin(cp)
8-	u = componente(de(arista))
9-	v = componente(a(arista))
10-	if numeroComponente(u)!= numeroComponente(v)
11-		añadir arista arista a F
12-		unirComponentes(u,v)
\end{lstlisting}

\subsection{Orden de Complejidad Kruskal}
El algoritmo de kruskal se ejecuta sobre estructuras de datos simples.\\

Al momento de ordenar las aristas del grafo por su peso se usa una ordenación de comparación
llamada (Comparison sort) la cual tiene una complejidad de orden O(n log n) lo cual permite
que el paso "eliminar una arista de peso mínimo de C" se ejecute en tiempo constante.\\

Se usa una estructura de datos sobre conjuntos disjuntos al momento de controlar qué vertices
están en qué componentes lo cual tiene una complejidad de orden O(n) ya que por cada arista
hay dos operaciones de busqueda y posiblemente una unión de conjuntos.\\

Incluso una estructura de datos sobre conjuntos disjuntos simple con uniones por rangos puede
ejecutar las operaciones mencionadas en O(m log n). \\

Sabiendo esta informacion llegamos a la conclusión que la complejidad total es del
orden de \\
\begin{center}
    $Kruskal \in \Theta(nlogn)$
\end{center}




\newpage
\section{Bibliograf\'ia}

[1]D. Programming, "Dynamic Programming | Top-Down and Bottom-Up approach | Tabulation V/S Memoization", CodesDope, 2020. [Online]. Available: https://www.codesdope.com/course/algorithms-dynamic-programming/.\newline

[2]Ingenieria.unam.mx, 2020. [Online]. Available: \newline https://www.ingenieria.unam.mx/sistemas/PDF/Avisos/Seminarios\newline/SeminarioV/Sesion6_IdaliaFlores_20abr15.pdf.\newline

[3]"Tecnicas de diseño", Desarrolloweb.com, 2020. [Online]. Available:\\ https://desarrolloweb.com/articulos/2183.php. \newline

[4]"Números de Fibonacci ⋆ Quantdare", Quantdare, 2020. [Online]. Available: https://quantdare.com/numeros-de-fibonacci/.\newline

[5] Pisinger, D. (2003). Where are the hard knapsack problems?. Technical Report 2003/8, DIKU, University of Copenhagen, Denmark.\newline

[6] "Algoritmo de Prim", Estructura de Datos II, 2020. [Online]. Available: https://estructurasite.wordpress.com/algoritmo-de-prim/.\\

[7] "Algoritmo de Prim - Complejidad Algorítmica", Sites.google.com, 2020. [Online]. Available: https://sites.google.com/site/complejidadalgoritmicaes/prim.\\

[8] Mathcs.emory.edu, 2020. [Online]. Available: http://www.mathcs.emory.edu/\\cheung/Courses/171/Syllabus/11-Graph/prim2.html.\\

[9]"Algoritmo de Kruskal - EcuRed", Ecured.cu, 2020. [Online]. Available: https://www.ecured.cu/Algoritmo_de_Kruskal.\\

[10]"Algoritmo de Kruskal - Complejidad Algorítmica", Sites.google.com, 2020. [Online]. Available: https://sites.google.com/site/complejidadalgoritmicaes/kruskal.



\end{document}
